\documentclass[fleqn]{article}
\usepackage[margin = 0.6in, paper = a4paper]{geometry}
\usepackage{xcolor}
\usepackage[fleqn]{amsmath}
\usepackage{amssymb}
\usepackage{amsthm}
\usepackage{bookmark}
\usepackage{hyperref}
\hypersetup{
  pdfauthor = {Kiprian Berbatov},
  pdftitle = {IDEC Documentation},
  pdfsubject = {Discrete calculus},
  pdfkeywords = {discrete, microstructure, calculus, geometry},
  pdfcreator = {pdflatex},
  pdfproducer = {Latex2e with hyperref},
  colorlinks = true,
  linkcolor = blue,
  citecolor = green,
  urlcolor = cyan,
  bookmarksnumbered = true
}
\usepackage{cleveref}
\usepackage[most]{tcolorbox}
\newcommand\coloredcomponent[2]
{
  \tcolorboxenvironment{#1}
  {
    breakable,
    enhanced,
    colback = white,
    colframe = #2,
    boxrule = 1pt,
    left = 2pt,
    right = 2pt,
    top = 2pt,
    bottom = 2pt,
    sharp corners,
    before skip = \topsep,
    after skip = \topsep,
  }
}

\counterwithin{equation}{section}
\theoremstyle{definition}

\newtheorem{theorem}{Theorem}[section]
\coloredcomponent{theorem}{blue}

\newtheorem{proposition}[theorem]{Proposition}
\coloredcomponent{proposition}{blue}

\newtheorem{lemma}[theorem]{Lemma}
\coloredcomponent{lemma}{blue}

\newtheorem{corollary}[theorem]{Corollary}
\coloredcomponent{corollary}{blue}

\newtheorem{hypothesis}[theorem]{Hypothesis}
\coloredcomponent{hypothesis}{red}

\newtheorem{notation}[theorem]{Notation}
\coloredcomponent{notation}{green}

\newtheorem{definition}[theorem]{Definition}
\coloredcomponent{definition}{green}

\newtheorem{discussion}[theorem]{}
\coloredcomponent{discussion}{yellow}

\newtheorem{example}[theorem]{Example}
\coloredcomponent{example}{purple}

\newtheorem{remark}[theorem]{Remark}
\coloredcomponent{remark}{orange}

\coloredcomponent{proof}{cyan}

\renewcommand{\thefootnote}{\arabic{footnote}}

\newcommand{\N}{\mathbb{N}}
\newcommand{\Z}{\mathbb{Z}}
\newcommand{\Q}{\mathbb{Q}}
\newcommand{\R}{\mathbb{R}}
\renewcommand{\S}{\mathbb{S}}

\newcommand{\set}[2]{\left\{ #1 \mid #2 \right\}}
\newcommand{\restrict}[2]{\left. #1 \right|_{#2}}

\newcommand{\norm}[1]{\left\lVert#1\right\rVert}
\newcommand{\abs}[1]{\left\lvert#1\right\rvert}
\newcommand{\inner}[2]{\langle#1,#2\rangle}

\newcommand{\linearspan}{\mathop{\rm span}\nolimits}
\newcommand{\Ker}{\mathop{\rm Ker}\nolimits}
\renewcommand{\Im}{\mathop{\rm Im}\nolimits}
\newcommand{\Hom}{\mathop{\rm Hom}\nolimits}
\newcommand{\id}{\mathop{\rm id}\nolimits}
\newcommand{\tr}{\mathop{\rm tr}\nolimits}
\newcommand{\sym}{\mathop{\rm sym}\nolimits}

\newcommand{\grad}{\mathop{\rm grad}\nolimits}
\renewcommand{\div}{\mathop{\rm div}\nolimits}

\newcommand{\Aff}{\mathop{\rm Aff}\nolimits}
\newcommand{\Con}{\mathop{\rm Con}\nolimits}

\newcommand{\Cl}{\mathop{\rm Cl}\nolimits}
\newcommand{\Link}{\mathop{\rm Link}\nolimits}

\newcommand{\sgn}{\mathop{\rm sgn}\nolimits}
\newcommand{\OR}{\mathop{\rm or}\nolimits}
\newcommand{\vol}{\mathop{\rm vol}\nolimits}

\title{IDEC documentation}
\author{Kiprian Berbatov}
\date{June 11, 2024}

\begin{document}

\maketitle

\section{Jagged arrays}

\begin{definition}
  \label{idec:jagged:definition}
  Let $n$ be a positive integer and $T$ be a type (e.g., integers)
  A \textbf{jagged array}
  of order $n$ is defined recursively as follows:
  \begin{enumerate}
    \item
      jagged array of order $0$ is an object of type $T$;
    \item
      for $n > 0$, jagged array of order $n$ is a finite (possibly empty)
      sequence of jagged arrays of order $n - 1$.
  \end{enumerate}
\end{definition}

\begin{example}
  \leavevmode
  \begin{itemize}
    \item
      $()$ and $(0, 1, 2)$ are jagged arrays of order $1$ (ordinary arrays);
    \item
      $((0, 1, 2), (3), ())$ is a jagged array of order $2$
      (note that its last element is the empty array of order $1$);
    \item
      $(((0, 1, 2), (3)),())$ is a jagged array of order $3$
      (note that its last element is the empty array of order $2$);
  \end{itemize}
\end{example}

\begin{definition}
  Let $n > 0$.
  A \hyperref[idec:jagged:definition]{jagged array}
  of order $n$ is represented in memory by a structure of $n + 1$
  fields as follows:
  an integer $a_0$,
  $n - 1$ pointers $a_1, ..., a_{n - 1}$ to integer
  (any of them points to the  $0$-th element of an array of integers),
  a pointer $a_n$ to the $0$-th element of an array of type $T$.
  
  The fields $a_0, ..., a_{n - 1}$ represent the internal structure,
  while $a_n$ holds the values.
  More precisely, let $x$ be a jagged array of order $n$.
  $a_0$ is the number of its elements of order $n - 1$.
  For $i = 0, ..., a_0 - 1$,
  $a_1[i]$ represents the number of elements of order $n - 2$ in $x[i]$.
  There are totally $b_2 = a_1[0] + ... a_1[a_0 - 1]$ elements of order $n - 2$.
  For $i = 0, ..., a_0 - 1$, $j  = 0, ..., a_1[i] - 1$,
  $a_2[a_1[0] + ... + a_1[i - 1] + j]$
  is the number of its elements of order $n - 3$ in $x[i][j]$.
  Continue in the same manner for higher orders.
  Finally, $a_n$ represents the flattened version of $x$.
\end{definition}

\begin{example}
  Consider the \hyperref[idec:jagged:definition]{jagged array} of order $3$
  \begin{equation}
    x = (((0, 1, 2), (3, 4)), (), (()), ((5, 6), ())).
  \end{equation}
  Then $x$ has the following representation:
  \begin{equation}
    \begin{split}
      & x.a0 = 4 \\
      & x.a1 = (2, 0, 1, 2) \\
      & x.a2 = (3, 2, 0, 2, 0) \\
      & x.a3 = (0, 1, 2, 3, 4 , 5, 6)
    \end{split}
  \end{equation}
\end{example}

\section{Meshes}

\begin{definition}
  \label{idec:mesh:definition}
  Let $d$ be a natural number.
  A \textbf{mesh}
  of dimension $d$ is a finite set of polytopes of dimension at
  most $d$ such that:
  \begin{itemize}
    \item
      if $X$ is an element of $M$, then all subfaces of $X$ are also in $M$;
    \item
      the intersection of elements of $M$ is a finite (possibly empty) union of
      elements of $M$.
  \end{itemize}
  For an integer $p \in [0, d]$, the set of elements (polytopes) of dimension
  $p$ in $M$ is denoted by $M_p$.
\end{definition}

\begin{definition}
  A \hyperref[idec:mesh:definition]{mesh}
  is represented in memory by its topology and geometry.
  The topology is represented using:
  \begin{itemize}
    \item
      integer \emph{dim} ($d$) storing mesh's dimension;
    \item
      array \emph{cn} storing the number of $p$-cells for $p = 0, ..., d$; 
    \item
      a jagged array \emph{cf} of order $4$ storing its topology.
      More precisely, if $0 < p \leq d$, $0 \leq q < p$,
      $0 \leq i < {\rm cn}[p]$, then ${\rm cf}[p][q][i]$ stores the indices of
      all $q$-dimensional subfaces of the $i$-th $p$-cell.
  \end{itemize}
  For a mesh of flat polytopes, the topology is represented by:
  \begin{itemize}
    \item
      integer \emph{dim\_embedded} ($d$) storing mesh's embedding dimension;
    \item
      array of floating point numbers \emph{coord} of length
      ${\rm cn}[0] * {\rm dim\_embedded}$ storing coordinates of the nodes
      (${\rm cn}[0]$ nodes of dimension $\R^{{\rm cn}[0]}$).
  \end{itemize}
\end{definition}

\begin{example}
  Consider two triangles dividing a square.
  Representation of the topology: ${\rm dim} = 2$, ${\rm cn} = \{4, 5, 2\}$,
  \begin{equation}
    {\rm cf} =
    (
      (
        ((0, 1), (1, 2), (2, 3), (3, 0), (0, 2)),
      ),
      (
        ((0, 1, 2), (0, 1, 3)),
        ((0, 1, 4), (2, 3, 4))
      )
    ).
  \end{equation}
  Representation of the geometry: ${\rm dim\_embedded} = 2$,
  ${\rm coord} = (-1, -1, 1, -1, 1, 1, 1, -1)$
  (the latter represents the nodes with coordinates
  $(-1, -1), (1, -1), (1, 1), (1, -1)$ respectively).
\end{example}

\section{Relative orientation on meshes}

\begin{theorem}
  Let
    $d \in \N$,
    $M$ be a \hyperref[idec:mesh:definition]{mesh} of dimension $d$,
    $p \in \{2, ..., d\}$,
    $a_p \in M_p$,
    $c_{p - 2} \in M_{p - 2}$,
    $a_p \succ c_{p - 2}$.
  Then there exist exactly two $(p - 1)$-cells
  $b_{p - 1}, b_{p - 1}' \in M_{p - 1}$
  that are between $a_p$ $c_{p - 2}$, i.e.,
  \begin{equation}
    a_p \succ b_{p - 1} \succ c_{p - 2}\ \text{and}\
    a_p \succ b_{p - 1}'' \succ c_{p - 2}.
  \end{equation}
\end{theorem}

\begin{definition}
  \label{idec:relative_orientation:definition}
  Let
    $d \in \N$,
    $M$ be a \hyperref[idec:mesh:definition]{mesh} of dimension $d$,
    $R$ be a commutative ring with unity.
  A family of maps
  \begin{equation}
    \{\varepsilon_p \colon M_p \times M_{p - 1} \to \{-1, 0, 1\}\}_{p = 1}^d
  \end{equation}
  is called a \textbf{relative orientation}
  on $M$ if the following conditions are satisfied:
  \begin{enumerate}
    \item
      for any $p \in \{1, ..., d\}$, $a_p \in M_p$, $b_{p - 1} \in M_{p - 1}$,
      \begin{equation}
        a_p \succ b_{p - 1} \Leftrightarrow \varepsilon(a_p, b_{p - 1}) \neq 0;
      \end{equation}
    \item
      for any edge $a_1$ with endpoints the nodes $b_0$ and $c_0$,
      \begin{equation}
        \varepsilon(a_1, b_0) + \varepsilon(a_1, c_0) = 0;
      \end{equation}
    \item
      for any
        $p \in \{2, ..., d\}$,
        $a_p \in M_p$,
        $c_{p - 2} \in M_{p - 2}$ with $a_p \succ c_{p - 2}$,
      let $b_{p - 1}$ and $b_{p - 1}'$ be the two
      cells between $a_p$ and $c_{p - 2}$.
      Then
      \begin{equation}
          \varepsilon(a_p, b_{p - 1}) \varepsilon(b_{p - 1}, c_{p - 2}) 
        + \varepsilon(a_p, b_{p - 1}') \varepsilon(b_{p - 1}', c_{p - 2}) = 0.
      \end{equation}
  \end{enumerate}
\end{definition}

\begin{remark}
  Note that the last condition in the above definition can be written as
  \begin{equation}
    \sum _{b_{p - 1} \in (c_{p - 2}, a_p)}
    \varepsilon(a_p, b_{p - 1}) \varepsilon(b_{p - 1}, c_{p - 2}) = 0.
  \end{equation}
\end{remark}

\begin{theorem}
  Let
    $d \in \N$,
    $M$ be a \hyperref[idec:mesh:definition]{mesh} of dimension $d$,
    $R$ be a commutative ring with unity.
  Then there exists a relative orientation on $M$.
\end{theorem}

\section{Chains and boundary operator on meshes}

\begin{definition}
  \label{idec:mesh_chain:definition}
  Let
    $d \in \N$,
    $M$ be a \hyperref[idec:mesh:definition]{mesh} of dimension $d$,
    $p \in \N,\ p \in [0, d]$,
    $R$ be a commutative ring with unity (for instance, $R = \R$).
  The space $C_p(M, R)$ of \textbf{$p$-chains}
  on $M$ with coefficients in $R$ is the free $R$-module
  (vector space over $R$ when $R$ is a field, e.g., when $R = \R$)
  generated by $M_p$:
  \begin{equation}
    C_p(M, R) := {\rm Free}_R(M_p).
  \end{equation}
  In other words, the elements of $C_p(M, R)$ are the formal linear combinations
  of cells in $M_p$ in coefficients in $R$.
  An element $c_p$ of $C_p(M, R)$ has the form
  \begin{equation}
    c_p := \lambda_0 c(p_, h_0) + ... + \lambda_{n - 1} c(p, h_{n - 1}),
  \end{equation}
  where for $i = 0, ..., n - 1$, $\lambda_i \in R$ and $c(p, h_i) \in M_p$.
\end{definition}

\begin{definition}
  Let
    $d \in \N$,
    $M$ be a \hyperref[idec:mesh:definition]{mesh} of dimension $d$,
    $R$ be a commutative ring with unity.
  The space $C_\bullet(M, R)$ of all \textbf{chains} on $M$ is the direct sum
  \begin{equation}
    C_\bullet(M, R) := \bigoplus_{p = 0}^d
    \hyperref[idec:mesh_chain:definition]{C_p(M, R)}.
  \end{equation}
\end{definition}

\begin{definition}
  Let
    $d \in \N$,
    $M$ be a \hyperref[idec:mesh:definition]{mesh} of dimension $d$,
    $R$ be a commutative ring with unity,
    $\varepsilon$ be a relative orientation on $M$,
    $p \in \{1, ..., d\}$,
  The \textbf{boundary operator on $p$-cells} $\partial_p$ is the map
  \begin{equation}
     \partial_p \colon C_p(M, R) \to C_{p - 1}(M, R)
  \end{equation}
  defined for a basis cochain $c_p \in M_p$ by
  \begin{equation}
    \partial_p a_p
    := \sum_{b_{p - 1} \preceq a_p} \varepsilon(a_p, b_{p - 1}) b_{p - 1}.
  \end{equation}
  and extended on $C_p M$ by linearity.

  The full \textbf{boundary operator} $\partial$ on $M$,
  \begin{equation}
    \partial \colon C_\bullet(M, R) \to C_\bullet(M, R),
  \end{equation}
  is the sum of all boundary operators on $p$-cells.
  In other words, for any $p \in \{1, ..., d\}$ $\sigma_p \in C_p(M, R)$,
  \begin{equation}
    \partial \sigma_p := \partial_p \sigma_p,
  \end{equation}
  and $\partial$ is extended by linearity on all cells
  (it returns zero when acting on $0$-cells).
\end{definition}

\begin{proposition}
  Let
    $d \in \N$,
    $M$ be a \hyperref[idec:mesh:definition]{mesh} of dimension $d$,
    $R$ be a commutative ring with unity,
    $\varepsilon$ be a relative orientation on $M$.
  Then the algebra $(C_\bullet M, \partial)$ is a chain complex, i.e.,
  \begin{equation}
    \partial^2 = 0.
  \end{equation}
\end{proposition}

\begin{proof}
  It is enough to prove that for any $p \in \{0, ..., d\}$, $c_p \in M_d$,
  \begin{equation}
    \partial^2 c_p = 0. 
  \end{equation}
  The proposition is trivially true for $p = 0$ and $p = 1$
  becuase $\partial_0 = 0$.
  Assume that $p \geq 2$.
  Then
  \begin{equation}
    \begin{split}
      \partial^2 a_p
      & = \partial_{p - 1} (\partial_p a_p) \\
      & = \partial_{p - 1}
      \left(
        \sum_{b_{p - 1} \prec a_p} \varepsilon(a_p, b_{p - 1}) b_{p - 1}
      \right) \\
      & =
      \sum_{b_{p - 1} \prec a_p}
        \sum_{c_{p - 2} \prec b_{p - 1}}
            \varepsilon(a_p, b_{p - 1})
            \varepsilon(b_{p - 1}, c_{p - 2})
            c_{p - 2} \\
      & =
      \sum_{c_{p - 2} \prec a_p}
        \left(
          \sum _{b_{p - 1} \in (c_{p - 2}, a_p)}
            \varepsilon(a_p, b_{p - 1}) \varepsilon(b_{p - 1}, c_{p - 2})
        \right)
        c_{p - 2} \\
      & = 0
    \end{split}
  \end{equation}
  (the last equation follows from the last condition in the definition of
  \hyperref[idec:relative_orientation:definition]{relative orientation}).
\end{proof}

\begin{proposition}
  Let
    $d \in \N$,
    $M$ be a \hyperref[idec:mesh:definition]{mesh} of dimension $d$,
    $R$ be a commutative ring with unity,
    $\varepsilon$ and $\varepsilon'$ be a relative orientations on $M$
      with corresponding boundary operators
      $\partial$ and $\partial'$ respectively.
  Then
  \begin{equation}
    (C_\bullet(M, R), \partial) \cong (C_\bullet(M, R), \partial')
  \end{equation}
  ($\cong$ is understood as isomorphism of chain complexes).
\end{proposition}

\begin{remark}
  The above proposition says that the boundary operator is esentially unique,
  i.e., up to isomorphism it does not depend on the chosen relative orientation.
  This motivates the notion of ``\emph{the} boundary operator'' on a mesh.
  Nevertheless, this does not exclude special choices of relative orientations
  in some particular cases like compatibly orientable meshes or regular grids.
\end{remark}

\section{Cochains and coboundary operator on meshes}

\begin{definition}
  Let
    $d \in \N$,
    $M$ be a \hyperref[idec:mesh:definition]{mesh} of dimensions $d$,
    $R$ be a commutative ring with unity.
  The space $C^\bullet(M, R)$ of \textbf{cochains} is the dual to the
  \hyperref[idec:mesh_chain:definition]{space of chains}, i.e.,
  \begin{equation}
    C^\bullet(M, R) := (C_\bullet(M, R))^* = \Hom_R(C_\bullet(M, R), R)
  \end{equation}
  If $p \in \{0, .., d\}$ then the space of \textbf{$p$-cochains} $C^p(M, R)$ is
  \begin{equation}
    C^p(M, R) := (C_p(M, R))^* = \Hom_R(C_p(M, R), R).
  \end{equation}
  We have the decomposition
  \begin{equation}
    C^\bullet(M, R) = \bigoplus_{p = 0}^d C^p(M, R).
  \end{equation}
\end{definition}

\begin{definition}
  Let
    $d \in \N$,
    $M$ be a \hyperref[idec:mesh:definition]{mesh} of dimension $d$,
    $R$ be a commutative ring with unity,
    $\partial$ be a boundary operator on $M$.
  Then the corresponding \textbf{coboundary operator} on $M$ $\delta$ is the
  dual of $\partial$, i.e.,
  \begin{equation}
    \delta = \partial^* \colon C^\bullet(M, R) \to C^\bullet(M, R).
  \end{equation}
  In other words, for any cochain $\pi^\bullet \in C^\bullet(M, R)$ and any
  chain $\rho_\bullet \in C_\bullet(M, R)$,
  \begin{equation}
    (\delta \pi^\bullet) \rho_\bullet := \pi^\bullet(\partial \rho_\bullet).
  \end{equation}
  If $p \in \{0, ..., d - 1\}$, the \textbf{coboundary operator on $p$-cochains}
  $\delta_p$ is defined as the dual of $\partial_{p + 1}$.
  In other words,
  \begin{equation}
    \delta_p = \partial_{p + 1}^* \colon C^p(M, R) \to C^{p + 1}(M, R).
  \end{equation}
  If $\pi^p \in C^p(M, R)$, $\rho_{p + 1} \in C_{p + 1}(M, R)$, then
  \begin{equation}
    (\delta_p \pi^p) \rho_{p + 1} = \pi^p(\partial_{p + 1} \rho_{p + 1}).
  \end{equation}
\end{definition}

\begin{proposition}
  Let
    $d \in \N$,
    $M$ be a \hyperref[idec:mesh:definition]{mesh} of dimension $d$,
    $R$ be a commutative ring with unity,
    $\partial$ be a boundary operator on $M$,
    $\delta$ be the corresponding coboundary operator.
  Then
  \begin{equation}
    (C^\bullet(M, R), \delta)
  \end{equation}
  is a cochain complex.
\end{proposition}

\begin{proof}
  Let $\pi^\bullet \in C^\bullet(M, R)$, $\rho_\bullet \in C_\bullet(M, R)$.
  Then
  \begin{equation}
    (\delta^2 \pi^\bullet) \rho_\bullet
    = (\delta \pi^\bullet) (\partial \rho_\bullet)
    = \pi^\bullet (\partial (\partial \rho_\bullet))
    = \pi^\bullet (0)
    = 0.
  \end{equation}
  Since $\pi^\bullet$ and $\rho_\bullet$ were arbitrary,
  \begin{equation}
    \delta^2 = 0.
  \end{equation}
\end{proof}

\section{Combinatorial differential forms and Forman decomposition}

\begin{definition}
  Let
    $d \in \N$,
    $M$ be a \hyperref[idec:mesh:definition]{mesh} of dimension $d$,
    $0 \leq p_f \leq d$.
  A \textbf{combinatorial differential form} of dimension $p_f$
  (called $p_f$-form for short) on $M$ is a linear map
  \begin{equation}
    \omega \colon C_\bullet M \to C_\bullet M
  \end{equation}
  such that for any $p \in [p_f, d]$ and any $c_p \in M_p$,
  $\omega(c_p)$ is a linear combination of the $(p - p_f)$-subfaces of $c_p$.

  The space of all $p_f$-forms on $M$ is denoted by $\Omega^{p_f} M$.
  The space of all combinatorial differential forms is the direct sum
  \begin{equation}
    \Omega^\bullet M := \bigoplus_{p_f = 0}^{d} \Omega^{p_f} M.
  \end{equation}
\end{definition}

\begin{definition}
  Let
    $d \in \N$,
    $M$ be a \hyperref[idec:mesh:definition]{mesh} of dimension $d$
    $\partial$ be a boundary operator on $M$.
  The \textbf{discrete differential} on $M$ is the linear map
  \begin{equation}
    D \colon \Omega^\bullet \to \Omega^\bullet
  \end{equation}
  which maps a $p$-form $\omega$ to a $(p + 1)$-form by the formula
  \begin{equation}
    D \omega := \omega \circ \partial - (-1)^p \partial \circ \omega.
  \end{equation}
\end{definition}

\begin{proposition}
  Let
    $d \in \N$,
    $M$ be a \hyperref[idec:mesh:definition]{mesh} of dimension $d$,
    $\partial$ be a boundary operator on $M$m
    $D$ be the discrete differential on $M$.
  Then $(\Omega^\bullet M, D)$ is a cochain complex, i.e.,
  \begin{equation}
    D^2 = 0.
  \end{equation}
\end{proposition}

\begin{proof}
  A straightforward computation using the fact that $\partial^2 = 0$.
  Indeed, let $\omega \in \Omega^p M$.
  Then
  \begin{equation}
    \begin{split}
      D^2(\omega)
      & = D(D \omega) \\
      & = D(\omega \circ \partial - (-1)^p \partial \circ \omega) \\
      & =
      \omega \circ \partial \circ \partial
        - (-1)^{p + 1} \partial \circ \omega \circ \partial
        - (-1)^p \partial \circ \omega \circ \partial
        - (-1)^p (-1)^{p + 1} \partial \circ \partial \circ \omega \\
      & =
      0 
      - (-1)^{p + 1}\partial \circ \omega \circ \partial
      + (-1)^{p + 1}\partial \circ \omega \circ \partial
      + 0 \\
      & = 0.
    \end{split}
  \end{equation}
\end{proof}

\begin{definition}
  Let
    $d \in \N$,
    $M$ be a \hyperref[idec:mesh:definition]{mesh} of dimension $d$.
  Consider a mesh $K$ constructed as follows.
  The nodes of $K$ are the centroids of the cells of $M$.
  (In general, the topology of $K$ can always be constructed while the geometry
  is tricky.
  For simplicity we may assume that all the cells of $M$ are convex,
  although for non-simplicial or non-brick meshes in dimensions $3$ and above
  the resulting mesh may contain non-flat polytopes.)

  For $p_f \in [0, d]$, a $p_f$-cell of $K$ is constructed as follows.
  Let $p \in [p_f, d]$, $s = p - p_f$ consider two cells
  \begin{equation}
    c(p, i) \succeq c(s, l)
  \end{equation}
  Then a $p_f$-cell is such a pair $(c(p, i), c(s, l))$.
  If $q_f \in [0, p_f]$, $q \in [q_f, d]$, $r = q - q_f$,
  \begin{equation}
    c(q, j) \succeq c(r, k),
  \end{equation}
  then $(c(q, j), c(r, k))$ is a subface of $(c(p, i), c(s, l))$ if
  \begin{equation}
    c(p, i) \succeq c(q, j) \succeq c(r, k) \succeq c(s, l).
  \end{equation}
  The constructed space is a mesh which we call the \textbf{Forman subdivision}.
\end{definition}

\begin{definition}
  Let
    $d \in \N$,
    $M$ be a \hyperref[idec:mesh:definition]{mesh} of dimension $d$,
    $K$ be the Forman subdivision of $M$,
    $\varepsilon_M$ be the relative orientation on $M$.
  We construct the relative orientation $\varepsilon_K$ as follows.
  Let 
    $p_f \in [1, d]$,
    $p \in [p_f, d]$,
    $s = p - p_f$,
    $c_K(p_f, i_f)$ be a $p_f$-cell on $K$,
  \begin{equation}
    c_K(p_f, i_f) = (c(p, i), c(s, l))\
    \text{for some $c(p, i) \in C_p M$ and $c(s, l) \in C_s M$}.
  \end{equation}
  Let $c_K(p_f - 1, j_f)$ be a hyperface of $c_K(p_f, i_f)$.
  Then there exist $q, r \in \N$ such that
  $p \geq q \geq r \geq s$ and $q - r = p_f - 1$,
  such that
  \begin{equation}
    c_K(p_f - 1, j_f) = (c(q, j), c(r, k))\
    \text{for some $c(q, j) \in C_p M$ and $c(r, k) \in C_r M$}.
  \end{equation}
  There are two possibilities for $q$ and $r$:
  $(q, r) = (p - 1, s)$ or $(q, r) = (p, s + 1)$.
  \begin{enumerate}
    \item
      If $(q, r) = (p - 1, s)$, then
      \begin{equation}
        c_K(p_f - 1, j_f) = (c(p - 1, j), c(s, l)),\
        \text{where $c(p, i) \succ c(p - 1, j) \succeq c(s, l)$}.
      \end{equation}
      In this case
      \begin{equation}
        \varepsilon_K(c_K(p_f, i_f), c_K(p_f - 1, j_f))
        = \varepsilon_M(c_M(p, i), c_M(p - 1, j)).
      \end{equation}
    \item
      If $(q, r) = (p, s + 1)$, then
      \begin{equation}
        c_K(p_f - 1, j_f) = (c(p, i), c(s + 1, k)),\
        \text{where $c(p, i) \succeq c(s + 1, k) \succ c(s, l)$}.
      \end{equation}
      In this case
      \begin{equation}
        \varepsilon_K(c_K(p_f, i_f), c_K(p_f - 1, j_f))
        = (-1)^{p_f} \varepsilon_M(c_M(s + 1, k), c_M(s, l)).
      \end{equation}
  \end{enumerate}
\end{definition}

\begin{theorem}
  Let
    $d \in \N$,
    $M$ be a \hyperref[idec:mesh:definition]{mesh} of dimension $d$,
    $\varepsilon_M$ be a relative orientation on $M$
      with corresponding boundary operator $\partial_M$
      and discrete differential $D_M$.
  Let $K$ be the Forman subdivision of $M$,
  $\varepsilon_K$ be the orientation on $K$ constructed above,
  $d_K$ be the corresponding coboundary operator on $K$.
  Then
  \begin{equation}
    (\Omega^p M, D_M) \cong (C^p K, d_K).
  \end{equation}
  with the isomorphism being the mapping of the basis forms to basis cochains
  introduced in the construction of $K$.
\end{theorem}

\begin{definition}
  Let $d \in \N$, $P$ be a polytope of dimension $d$.
  We say that $P$ is a ($d$-dimensional) \textbf{quasi-cube} if the mesh
  topology of $P$ is isomorphic to the mesh topology of the $d$-cube.
\end{definition}

\begin{example}
  All $0$D and $1$D polytopes (points and segments respectively)
  are quasi-cubes.
  In $2$D quasi-cubes are quadrilaterals.
  In $3$D quasi-cubes are shapes
  with $8$ vertices, $12$ edges and $6$ faces (hexahedrons).
\end{example}

\begin{definition}
  Let $M$ be a \hyperref[idec:mesh:definition]{mesh}.
  We say that $M$ is a \textbf{quasi-cubical mesh}
  if all cells of $M$ are quasi-cubes.
\end{definition}

\begin{definition}
  Let
    $(P, \preceq)$ be a partially ordered set,
    $a, b \in P$ with $a \preceq b$.
  The \textbf{closed interval} $[a, b]$ is defined as
  \begin{equation}
    [a, b] := \set{x \in P}{a \preceq x\ \&\ x \preceq b}.
  \end{equation}
\end{definition}

\begin{definition}
  We say that a mesh $M$ is \textbf{interval-simplicial} if for any two cells
  $a, b \in M$ with $a \preceq b$, the interval $[a, b]$ is an abstract simplex. 
\end{definition}

\begin{proposition}
  Let
    $M$ be a \hyperref[idec:mesh:definition]{mesh}
    and $K$ be its Forman decomposition.
  Then $M$ is interval-simplicial if and only if $K$ is quasi-cubical.
\end{proposition}

\begin{proposition}
  Let $M$ be a \hyperref[idec:mesh:definition]{mesh} of dimension at most $2$.
  Then $M$ is interval-simplicial.
  In particular, its Forman decomposition is quasi-cubical.
\end{proposition}

\begin{definition}
  Let $d \in \N$ and $P$ be a polytope of dimension $d$.
  We say that $P$ is a \textbf{simple polytope}
  if any of its nodes is connected to exactly $d$ edges.
\end{definition}

\begin{proposition}
  Let $M$ be a \hyperref[idec:mesh:definition]{mesh} of dimension $3$.
  Then $M$ is interval-simplicial if and only if all $3$-cells of $M$
  are simple polytopes.
\end{proposition}

\section{Metric-dependent calculus on quasi-cubical meshes}

\begin{discussion}
  As we saw, the Forman decomposition of an interval-simplicial mesh leads to a
  quasi-cubical mesh.
  Interval-simplicial meshes are not that uncommon:
  \begin{enumerate}
    \item
      all meshes of dimension at most $2$ are interval-simplicial;
    \item
      all $3$D meshes of simple polytopes are interval-simplicial;
    \item
      all simplicial and quasi-cubical meshes are interval-simplicial;
    \item
      the product of interval-simplicial meshes is an interval-simplicial mesh.
  \end{enumerate}
  For that reason we will build our calculus on qusi-cubical meshes thought as
  the Forman decomposition of an interval-simplicial mesh.
\end{discussion}

\begin{definition}
  Let
    $d \in \N$,
    $K$ be a quasi-cubical mesh of dimension $d$,
    $a_d \in M_d$,
    $p \in \{0, ..., d\}$,
    $b_p \in M_p$,
    $c_{d - p} \in M_{d - p}$.
  We say that $b_p$ and $c_{d - p}$ are \textbf{topologically orthogonal}
  with respect to $a_d$ if $b_p, c_{d - p} \preceq a_d$, and the intersection of
  $b_p$ with $c_{d - p}$ is a node in $a_d$.
  In this case we write
  \begin{equation}
    b_p \oplus c_{d - p} = a_p\
    \text{and}\
    b_p \perp c_{d - p}.
  \end{equation}
\end{definition}

\begin{definition}
  Let
    $d \in \N$,
    $K$ be a quasi-cubical mesh of dimension $d$,
    $p \in \{0, ..., d\}$.
  The \textbf{inner product of $p$-cochains} is defined as follows.
  If $b_p \in K_p$, then
  \begin{equation}
    \inner{b^p}{b^p}_p :=
    \frac{1}{2^d \mu(b_p)} \sum_{c_{d - p} \perp b_p} \mu(c_{d - p}).
  \end{equation}
  The inner product between two different basis $p$-cochains
  is defined to be zero.
  Extending by bilinearity, we define the inner product of arbitrary
  $p$-cochains (since we already have it for pairs of basis $p$-cochains).
\end{definition}

\begin{example}
  Let
    $d \in \N$,
    $h \in \R^+$
    $K$ a cubical grid of size $h$ with dimension $d$,
    $p \in \{0, ..., d\}$,
    $b_p$ be an internal $p$-cell in $K$.
  Then
  \begin{equation}
    \inner{b^p}{b^p}_p = h^{d - 2 p}.
  \end{equation}
\end{example}

\begin{definition}
  Let
    $d \in \N$,
    $K$ be a quasi-cubical mesh of dimension $d$,
    $p \in \{1, ..., d\}$.
  The \textbf{adjoint coboundary operator on $p$-cochain}
  $\delta_p^\star \colon C^p K \to C^{p - 1} K$
  is defined as the adjoint of $\delta_{p - 1}$
  with respect to the inner product of cochains.
  In other words, for any $a^p \in C^p K,\ b^{p - 1} \in C^{p - 1}$,
  \begin{equation}
    \inner{\delta_p^\star a^p}{b^{p - 1}}_{p - 1}
    = \inner{a^p}{\delta_{p - 1} b^{p - 1}}_p.
  \end{equation}
\end{definition}

\begin{proposition}
  Let
    $d \in \N$,
    $K$ be a quasi-cubical mesh of dimension $d$,
    $p \in \{1, ..., d\}$.
  Then for any $a^p \in C^p K$,
  \begin{equation}
    \delta^\star_p a^p =
    \sum_{b_{p - 1} \prec a_p}
      \frac
      {\inner{a^p}{a^p}}
      {\inner{b^{p - 1}}{b^{p - 1}}}
      \varepsilon(a_p, b_{p - 1})
      b^{p - 1}
  \end{equation}
\end{proposition}

\begin{corollary}
  On a regular grid
  \begin{equation}
    \delta^\star_p a^p =
    \frac{1}{h^2}
    \sum_{b_{p - 1} \prec a_p}
      \varepsilon(a_p, b_{p - 1})
      b^{p - 1}
  \end{equation}
\end{corollary}

\section{Approximating vector fields with 1-cochains}

\begin{definition}
  Let $m, n \in \N$, $A$ be a real $m \times n$ matrix.
  An $n \times m$ matrix $B$ is called \textbf{Moore-Penrose inverse} or
  \textbf{pseudo-inverse} if
  \begin{subequations}
    \begin{align}
      A B A & = A, \\
      B A B & = B, \\
      (A B)^T & = A B, \\
      (B A)^T & = B A.
    \end{align}
  \end{subequations}
\end{definition}

\begin{theorem}
  Let $m, n \in \N$, $A$ be a real $m \times n$ matrix.
  Then $A$ has a unique Moore-Penrose inverse, denoted by $A^*$.
\end{theorem}

\begin{remark}
  If the matrix $A$ is of full rank there exists a closed formula for $A^*$.
  \begin{enumerate}
    \item
      If $A$ is a square matrix with full rank, i.e., an invertible one, then
      $A^* = A^{-1}$.
    \item
      If $m > n$ and $A$ is an $m \times n$ matrix of full rank, then its
      columns are linearly independent which means that $A^T A$ is
      symmetric and positive definite and hence invertible.
      (Its inverse $(A^T A)^{-1}$ is also symmetric and positive definite.)
      It is then easy to check that
      \begin{equation}
        B := (A^T A)^{-1} A^T
      \end{equation}
      is the Moore-Penrose inverse of $A$.
      Indeed, obviously $B$ is a left inverse of $A$, and 
      \begin{subequations}
        \begin{alignat}{2}
          & A B A && = A (B A) = A I_n = A, \\
          & B A B && = (B A) B = I_n B = B, \\
          & (A B)^T && = (A (A^T A)^{-1} A^T)^T
            = (A^T)^T ((A^T A)^{-1})^T A^T = A (A^T A)^{-1} A^T = A B, \\
          & (B A)^T && = I_n^T = I_n = B A.
        \end{alignat}
      \end{subequations}
    \item
      If $m < n$ and $A$ is an $m \times n$ matrix of full rank, then an
      analogous reasoning to the previous point shows that
      \begin{equation}
        A^* = A^T (A A^T)^{-1}.
      \end{equation}
  \end{enumerate}
\end{remark}

\begin{definition}
  Let $d\in \N$ and $M$ be a mesh of dimension $d$.
  We say that $M$ is \textbf{flat} if the following conditions are satisfied:
  \begin{enumerate}
    \item
      it is pure, i.e., all of its cells lie within some $d$-cell;
    \item
      all its cells are flat (but possibly degenerate) polytopes;
    \item
      it can be embedded in $\R^d$.
  \end{enumerate}
\end{definition}

\begin{definition}
  Let:
  \begin{enumerate}
    \item
      $d \in \N,\ d \geq 1$;
    \item
      $K$ be a flat mesh of dimension $d$
      (with a chosen embeddeding in $\R^d$);
    \item
      $\epsilon$ be a relative orientation on $K$;
    \item
      $\mathcal{N}_i$ be a node in $K$ connected to $n > 0$ edges;
    \item
      $\mathcal{E}_{j_0}, ..., \mathcal{E}_{j_{n - 1}}$ be all the edges
      containing $\mathcal{N}_i$ as a node;
    \item
      $\mathcal{N}_{i_0}, ..., \mathcal{N}_{i_{n - 1}}$ be the other than
      other $\mathcal{N}_i$ of
      $\mathcal{E}_{j_0}, ..., \mathcal{E}_{j_{n - 1}}$ respectively.
  \end{enumerate}
  We define the \textbf{node matrix} $\mathcal{L}_{\mathcal{N}_i}$ of
  $\mathcal{N}_i$ by
  \begin{equation}
    \mathcal{L}_{\mathcal{N}_i} :=
    \begin{pmatrix}
      x_{i_0, 0} - x_{i, 0} & \cdots & x_{i_0, d - 1} - x_{i, d - 1} \\
      \vdots & \ddots & \vdots \\
      x_{i_{n - 1}, 0} - x_{i, 0} & \cdots & x_{i_{n - 1}, d - 1} - x_{i, d - 1}
    \end{pmatrix}
    \in \R^{n \times d}.
  \end{equation}
\end{definition}

\begin{definition}
  Let $d \in \N\ \text{with}\ d \geq 1$, $K$ be a mesh of dimension $d$,
  $\pi^1 \in C^1 K$, $\mathcal{N}_i \in K_0$.
  Define the \textbf{neighbor representation} $\widehat{\pi^1}_{\mathcal{N}_i}$
  of $\pi^1$ at $\mathcal{N}_i$ by
  \begin{equation}
    \widehat{\pi^1}_{\mathcal{N}_i} :=
    ( \epsilon_K(\mathcal{E}_{j_0}, \mathcal{N}_{i_0}) \pi^1(\mathcal{E}_{j_0}),
      \cdots,
        \epsilon_K(\mathcal{E}_{j_{k - 1}}, \mathcal{N}_{i_{k - 1}})
        \pi^1(\mathcal{E}_{j_{k - 1}})
    ) \in \R^n.
  \end{equation}
  Define the \textbf{1-cochain embedding} $\overline{\pi^1}(x_i)$
  of $\pi^1$ at $\mathcal{N}_i$ by
  \begin{equation}
    \overline{\pi^1}(x_i) :=
    \left(\mathcal{L}_{\mathcal{N}_i}\right)^* \widehat{\pi^1}_{\mathcal{N}_i}
    \in \R^d.
  \end{equation}
\end{definition}

\begin{example}
  Let $h \in \R^+$ and $K$ be a regular subdivision with size $h$
  of some interval,
  all the edges in $K$ are oriented from left to right,
  $\pi^1 \in C^1 K$.
  \begin{enumerate}
    \item
      Consider an interior point $\mathcal{N}_i$ with neighboring edges
      $\mathcal{E}_{i - 1}$ and $\mathcal{E}_i$ and corresponing neighboring
      nodes $\mathcal{N}_{i - 1}$ and $\mathcal{N}_{i + 1}$.
      Then
      \begin{equation}
        \mathcal{L}_{\mathcal{N}_i} =
        \begin{pmatrix}
          -h \\
          h
        \end{pmatrix}
        \Rightarrow
        (\mathcal{L}_{\mathcal{N}_i})^* =
          \frac{1}{2 h}
          \begin{pmatrix}
            -1 & 1
          \end{pmatrix}
      \end{equation}
      and
      \begin{equation}
        \widehat{\pi^1}_{\mathcal{N}_i} =
        \begin{pmatrix}
          - \pi^1 \mathcal{E}_{i - 1} \\
          \pi^1 \mathcal{E}_i
        \end{pmatrix}
        \Rightarrow
        \overline{\pi^1}(x_i) =
          \frac{1}{2 h}
          \begin{pmatrix}
            -1 & 1
          \end{pmatrix}
          \begin{pmatrix}
            - \pi^1 \mathcal{E}_{i - 1} \\
            \pi^1 \mathcal{E}_i
          \end{pmatrix}
        = \frac{1}{2 h}
          \left(\pi^1 \mathcal{E}_{i - 1} + \pi^1 \mathcal{E}_i\right).
      \end{equation}
  \end{enumerate}
\end{example}

\begin{definition}
  \label{idec/vector_field_to_1_cochain/definition}
  Let $K$ be an embedded flat mesh and $u$ be a vector field on $|K|$.
  Define the approximation $\underline{u} \in C^1 K$ as follows.
  Let $\mathcal{E}_i$ be an edge with endpoints $\mathcal{N}_{i_0}$ and
  $\mathcal{N}_{i_1}$, oriented from $\mathcal{N}_{i_0}$ to $\mathcal{N}_{i_1}$.
  Then
  \begin{equation}
    \underline{u}(\mathcal{E}_i) :=
      \frac{1}{2}
      ( - (\mathcal{L}_{\mathcal{N}_{i_0}} u(x_{i_0}))_0
        + (\mathcal{L}_{\mathcal{N}_{i_1}} u(x_{i_1}))_1
      ).
  \end{equation}
\end{definition}

\begin{example}
  With the mesh of the previous example, we have
  \begin{equation}
    \label{idec/vector_field_to_1_cochain/1d_example:exact_value}
    \underline{u}(\mathcal{E}_i)
    = \frac{1}{2} (h u(x_i) + h u(x_{i + 1}))
    = h \frac{u(x_i) + u(x_{i + 1})}{2}.
  \end{equation}
  Let's calculate the consecutive application of approximation and embedding
  (and vice versa).
  \begin{equation}
    \underline{\left(\overline{\pi^1}\right)}(\mathcal{E}_i)
    = h
      \left(
        \frac{\overline{\pi^1}(x_i) + \overline{\pi^1}(x_{i + 1})}{2}
      \right)
    = \frac{h}{2}
      \frac{1}{2 h}
      ((\pi^1 \mathcal{E}_{i - 1} + \pi^1 \mathcal{E}_i)
       + (\pi^1 \mathcal{E}_i + \pi^1 \mathcal{E}_{i + 1}))
    = \frac
    {\pi^1 \mathcal{E}_{i - 1} + 2 \pi^1 \mathcal{E}_i
      + \pi^1 \mathcal{E}_{i + 1}}
    {4}.
  \end{equation}
  \begin{equation}
    \overline{\left(\underline{u}\right)}(x_i)
    = \frac{1}{2 h}
      \left(
        \underline{u} \mathcal{E}_{i - 1} + \underline{u} \mathcal{E}_i
      \right)
    = \frac{1}{2 h}
      \frac{h}{2}
      ((u(x_{i - 1}) + u(x_i)) + (u(x_i) + u(x_{i + 1})))
    = \frac{u(x_{i - 1}) + 2 u(x_i) + u(x_{i + 1})}{4}.
  \end{equation}
  In both cases of composition of embedding and approximation the final result
  is the identity operator when $\pi^1$ (respectively $u$) is linear with
  respect to the index $i$.
\end{example}

\section{Discrete elasticity}

\begin{discussion}[Discrete elasticity]
  Let $M$ be a mesh of dimension $3$, $K$ be the Forman decomposition of $M$.
  Let $L$ and $F$ denote length and force measures respectively.
  
  Discrete displacement is represented by
  \begin{equation}
    \eta^1 [L^2] \in C^1 K.
  \end{equation}
  Displacement gradient is represented by
  \begin{equation}
    \epsilon^0 [1] \in C^0 K,\ \epsilon^2 [L^2] \in C^2 K.
  \end{equation}
  Stress (force) is represented by
  \begin{equation}
    \tau^0 [F] \in C^0 K,\ \tau^2 [F L^2] \in C^2 K.
  \end{equation}
  Body force is represented by
  \begin{equation}
    \mathfrak{b}^1 [F] \in C^1 K.
  \end{equation}
  Let $\lambda, \mu [F] \in \R$ be the Lam{\'e} parameters.
  Our model is the following.
  \begin{subequations}
    \begin{align}
      & \epsilon^0 = \delta_1^\star \eta^1
      & (\text{volumetric displacement gradient}), \\
      & \epsilon^2 = \delta_1 \eta^1 
      & (\text{deviatoric displacement gradient}), \\
      & \tau^0 = \lambda \epsilon^0
      & (\text{hydrostatic force}), \\
      & \tau^2 = \mu \epsilon^2
      & (\text{deviatoric force}), \\
      & \delta_0 \tau^0 + \delta_2^\star \tau^2 + \mathfrak{b}^1 = 0
      & (\text{conservation of linear momentum}).
    \end{align}
  \end{subequations}
\end{discussion}

\begin{example}
  Consider the problem of a twist of a cylindrical bar described
  in Section 9.1 of
  (\href
    {https://www.sciencedirect.com/science/article/pii/S0020768311001727}
    {Hadjesfandiari 2011}).
  Let
    $\theta$ be the constant angle of twist per unit length,
    $\lambda$ and $\mu$ be the Lame parameters.
  Let
    ${\bf u}$ be the displacement vector,
    $\boldsymbol{\epsilon}$ be the strain tensor,
    $\boldsymbol{\omega}$ be the rotation tensor
    $\boldsymbol{\sigma}$ be the stress tensor.
  Then at any point $x = (x_1, x_2, x_3)$ we have:
  \begin{equation}
    {\bf u} =
    \begin{pmatrix}
      -\theta x_2 x_3 \\
      \theta x_1 x_3 \\
      0
    \end{pmatrix},\
    \boldsymbol{\epsilon} =
      \frac{\theta}{2}
      \begin{pmatrix}
        0 & 0 & - x_2  \\
        0 & 0 & x_1 \\
        -x_2 & x_1 & 0
      \end{pmatrix},\
    \boldsymbol{\omega} =
      \frac{\theta}{2}
      \begin{pmatrix}
        0 & 0 & - x_2  \\
        0 & 0 & x_1 \\
        x_2 & -x_1 & 0
      \end{pmatrix},\
    \boldsymbol{\sigma} =
      \mu \theta
      \begin{pmatrix}
        0 & 0 & - x_2  \\
        0 & 0 & x_1 \\
        -x_2 & x_1 & 0
      \end{pmatrix}.
  \end{equation}
  Let $h \in \R^+$ and consider a $3$D regular grid $K$ of size $h$.
  For integers $i, j, k$, nodes in $K$ have coordinates
  \begin{equation}
    {\bf x}_{(i, j, k)} := (i h, j h, k h).
  \end{equation}
  Nodes in $K$ will be denoted by $\mathcal{N}_{(i, j, k)}$.
  
  There are three type of edges in $K$ (parallel to the $3$ axes)
  constructed as follows.
  Let $p \in \{1, 2, 3\}$ and $e_p$ be the $p$-th unit vector.
  Denote by $\mathcal{E}^{(p)}_{(i, j, k)}$ the edge starting at
  $\mathcal{N}_{(i, j, k)}$ and ending at $\mathcal{N}_{(i, j, k) + e_p}$.
  In particular, the oriented boundary of $\mathcal{E}^{(1)}_{(i, j, k)}$ is
  \begin{equation}
    \partial_1 \mathcal{E}^{(1)}_{(i, j, k)} =
    - \mathcal{N}_{(i, j, k)} + \mathcal{N}_{(i + 1, j, k)}.
  \end{equation}
  Similar computation holds for $p = 2$ and $p = 3$.

  For $p, q \in \{1, 2, 3\},\ p < q$ faces in $K$ are denoted by
  $\mathcal{F}^{(p, q)}_{(i, j, k)}$ and represent squares starting at
  $\mathcal{N}_{(i, j, k)}$ with basis vectors going in directions
  $e_p$ and $e_q$.
  For instance, the oriented boundary of $\mathcal{F}^{(p, q)}_{(i, j, k)}$ is
  \begin{equation}
    \partial_2 \mathcal{F}^{(1, 2)}_{(i, j, k)} =
    - \mathcal{E}^{(1)}_{(i, j + 1, k)}
    + \mathcal{E}^{(1)}_{(i, j, k)}
    + \mathcal{E}^{(2)}_{(i + 1, j, k)}
    - \mathcal{E}^{(2)}_{(i, j, k)}.
  \end{equation}
  We will work with the approximation $\eta^1 := \underline{u}$.
  
  Let $p \in \{1, 2, 3\}$.
  Then for $\epsilon^0 := \delta_1^\star \eta^1$, using the fact that
  \begin{equation}
    \eta^1 \mathcal{E}^{(p)}_{(i, j, k) + e_p} =
    \eta^1 \mathcal{E}^{(p)}_{(i, j, k)},
  \end{equation}
  we calculate:
  \begin{equation}
    \epsilon^0 \mathcal{N}_{(i, j, k)}
    =
      \frac{1}{h^2}
      \sum_{p = 1}^3 (
          \eta^1 \mathcal{E}^{(p)}_{(i, j, k)}
        - \eta^1 \mathcal{E}^{(p)}_{(i, j, k) + e_p}
      )
    = 0.
  \end{equation}
  Hence,
  \begin{equation}
    \tau^0 = \lambda \epsilon^0 = 0.
  \end{equation} 
  Using the computation from
  \Cref{idec/vector_field_to_1_cochain/1d_example:exact_value}
  in each direction, we get
  \begin{subequations}
    \begin{alignat}{2}
      & \eta^1 \mathcal{E}^{(1)}_{(i, j, k)} && 
        = \frac{h}{2}\theta (-x_2 x_3 - x_2 x3)
        = - \theta h x_2 x_3
        = - \theta j k h^3, \\
      & \eta^1 \mathcal{E}^{(2)}_{(i, j, k)} && 
        = \frac{h}{2}\theta (x_1 x_3 + x_1 x3)
        = \theta h x_1 x_3
        = \theta i k h^3, \\
      & \eta^1 \mathcal{E}^{(3)}_{(i, j, k)} && 
        = \frac{h}{2}\theta (0 + 0)
        = 0.
    \end{alignat}
  \end{subequations}
  For $\epsilon^2 := \delta_1 \eta^1$, using that
  $\epsilon(c_2) = \eta^1(\partial c_2)$, we get:
  \begin{subequations}
    \begin{alignat}{2}
      & \epsilon^2 \mathcal{F}^{(1, 2)}_{(i, j, k)} &&
        = ( - \eta^1 \mathcal{E}^{(1)}_{(i, j + 1, k)}
            + \eta^1 \mathcal{E}^{(1)}_{(i, j, k)}
          )
        + ( \eta^1 \mathcal{E}^{(2)}_{(i + 1, j, k)}
            - \eta^1 \mathcal{E}^{(2)}_{(i, j, k)}
          )
        = \theta k h^3 + \theta k h^3
        = 2 \theta k h^3, \\
      & \epsilon^2 \mathcal{F}^{(1, 3)}_{(i, j, k)} && 
        = ( - \eta^1 \mathcal{E}^{(1)}_{(i, j, k + 1)}
            + \eta^1 \mathcal{E}^{(1)}_{(i, j, k)}
          )
        + ( \eta^1 \mathcal{E}^{(3)}_{(i + 1, j, k)}
            - \eta^1 \mathcal{E}^{(3)}_{(i, j, k)}
          )
        = \theta j h^3 + 0 = \theta j h^3, \\
      & \epsilon^2 \mathcal{F}^{(2, 3)}_{(i, j, k)} && 
        = ( - \eta^1 \mathcal{E}^{(2)}_{(i, j, k + 1)}
            + \eta^1 \mathcal{E}^{(2)}_{(i, j, k)}
          )
        + ( \eta^1 \mathcal{E}^{(3)}_{(i, j + 1, k)}
            - \eta^1 \mathcal{E}^{(3)}_{(i, j, k)}
          )
        = - \theta i h^3 + 0
        = - \theta i h^3.
    \end{alignat}
  \end{subequations}
  We have $\tau^2 = \mu \epsilon^2$ and hence
  $\delta_2^\star \tau^2 = \mu \delta_2^\star \epsilon^2$.
  Then 
  \begin{subequations}
    \begin{alignat}{2}
      & (\delta_2^\star \tau^2) \mathcal{E}^{(1)}_{(i, j, k)} &&
        =
          \frac{\mu}{h^2}
          (
            ( \epsilon^2 \mathcal{F}^{(1, 2)}_{(i, j, k)}
              - \epsilon^2 \mathcal{F}^{(1, 2)}_{(i, j - 1, k)}
            )
            -
            ( \epsilon^2 \mathcal{F}^{(1, 3)}_{(i, j, k)}
              - \epsilon^2 \mathcal{F}^{(1, 3)}_{(i, j, k - 1)}
            )
          )
        = \frac{\mu}{h^2} (0  - 0)
        = 0, \\
      & (\delta_2^\star \tau^2) \mathcal{E}^{(2)}_{(i, j, k)} &&
        =
          \frac{\mu}{h^2}
          (
            ( - \epsilon^2 \mathcal{F}^{(1, 2)}_{(i, j, k)}
              + \epsilon^2 \mathcal{F}^{(1, 2)}_{(i - 1, j, k)}
            )
            -
            ( \epsilon^2 \mathcal{F}^{(2, 3)}_{(i, j, k)}
              - \epsilon^2 \mathcal{F}^{(2, 3)}_{(i, j, k - 1)}
            )
          )
        = \frac{\mu}{h^2} (0  - 0)
        = 0, \\
      & (\delta_2^\star \tau^2) \mathcal{E}^{(3)}_{(i, j, k)} &&
        =
          \frac{\mu}{h^2}
          (
            ( - \epsilon^2 \mathcal{F}^{(1, 3)}_{(i, j, k)}
              + \epsilon^2 \mathcal{F}^{(1, 3)}_{(i - 1, j, k)}
            )
            -
            ( - \epsilon^2 \mathcal{F}^{(2, 3)}_{(i, j, k)}
              + \epsilon^2 \mathcal{F}^{(2, 3)}_{(i, j - 1, k)}
            )
          )
        = \frac{\mu}{h^2} (0  - 0)
        = 0.
    \end{alignat}
  \end{subequations}
  Hence, $\delta_2^\star \tau^2 = 0$.
  With zero body force $\mathfrak{b}^1$, we get:
  \begin{equation}
    \delta_0 \tau^0 + \delta_2^\star \tau^2 + \mathfrak{b}^1 = 0 + 0 + 0 = 0.
  \end{equation}
\end{example}

\end{document}
