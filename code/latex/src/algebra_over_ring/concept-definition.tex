\begin{definition}
  Let
    $R$ be a commutative ring with unity,
    $A$ be an $R$-module,
    $\mu \colon R \times R \to R$.
  We say that $(A, \mu)$ is an $R$-\textbf{algebra}
  (or \textbf{algebra over} $R$) if
  for all $a, b, c \in A$, $\lambda \in R$:
  \begin{subequations}
    \begin{alignat}{2}
      & \mu(a + b, c) = \mu(a, c) + \mu(b, c)
      && (\text{multiplication on $A$ is left-distributive}), \\
      %
      & \mu(a, b + c) = \mu(a, b) + \mu(a, c)
      && (\text{multiplication on $A$ is right-distributive}), \\
      %
      & \mu(\lambda * a, b) = \lambda * \mu(a, b)\qquad
      && (\text{$\mu$ respects multiplication of scalar on the left}), \\
      %
      & \mu(a, \lambda * b) = \lambda * \mu(a, b)\qquad
      && (\text{$\mu$ respects multiplication of scalar on the right}).
    \end{alignat}
  \end{subequations}
\end{definition}
\begin{definition}
  Let $R$ be a commutative ring with unity, $(A, \mu)$ be an $R$-algebra.
  \begin{enumerate}
    \item
      We say that $(A, \mu)$ is \textbf{associative algebra} if
      for any $a, b, c \in A$, $\mu(a, \mu(b, c)) = \mu(\mu(a, b), c)$.
      In other words, $(A, \mu)$ is a ring.
    \item
      We say that $(A, \mu)$ is \textbf{commutative algebra} if
      for any $a, b, c \in A$, $\mu(a, b) = \mu(b, a)$.
    \item
      Let $1 \in R$.
      We say that $(A, \mu, 1)$ is \textbf{unital algebra} if
      for any $a, \in A$, $\mu(a, 1) = \mu(1, a) = a$.
      ($1$ is the \textbf{unit element} or \textbf{unity}.)
    \item
      We say that $(A, \mu)$ is \textbf{alternating algebra} if
      for any $a \in A$, $\mu(a, a) = 0$.
    \item
      We say that $(A, \mu)$ is \textbf{anti-commutative algebra} if
      for any $a, b \in A$, $\mu(a, b) = - \mu(b, a)$.
  \end{enumerate}
\end{definition}
\begin{proposition}
  Let $R$ be a commutative ring with unity,
  Let $(A, \mu)$ be an algebra.
  \begin{itemize}
    \item
      If $A$ is alternating, then $A$ is anti-commutative.
    \item
      If $A$ is anti-commutative and the ring $R$ has the property
      \begin{equation}
        \label{equation:ring/characteristics_is_not_divisible_by_2}
        \forall x \in R,\ x + x = 0 \Rightarrow x = 0,
      \end{equation}
      then $A$ is alternating.
  \end{itemize}
\end{proposition}
\begin{proof}
  First, let $A$ be alternating.
  Then for any $a, b \in A$,
  \begin{equation}
    0 = \mu(a + b, a + b)
    = \mu(a, a) + \mu(a, b) + \mu(b, a) + \mu(b, b)
    = \mu(a, b) + \mu(b, a)
    \Rightarrow \mu(b, a) = - \mu(a, b).
  \end{equation}
  Conversely, let $A$ be anti-commutative. Taking $b = a$ in the definition
  leads to
  \begin{equation}
    \mu(a, a) = - \mu(a, a) \Rightarrow \mu(a, a) + \mu(a, a) = 0.
  \end{equation}
  Under the assumption of
  \Cref{equation:ring/characteristics_is_not_divisible_by_2},
  we conclude that $\mu(a, a) = 0$.
\end{proof}
