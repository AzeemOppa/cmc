\begin{definition}
  Let $R$ be a commutative ring with unity, $(A, \mu)$ be an $R$-algebra.
  \begin{enumerate}
    \item
      We say that $(A, \mu)$ is \textbf{associative algebra} if
      for any $a, b, c \in A$, $\mu(a, \mu(b, c)) = \mu(\mu(a, b), c)$.
      In other words, $(A, \mu)$ is a ring.
    \item
      We say that $(A, \mu)$ is \textbf{commutative algebra} if
      for any $a, b, c \in A$, $\mu(a, b) = \mu(b, a)$.
    \item
      Let $1 \in R$.
      We say that $(A, \mu, 1)$ is \textbf{unital algebra} if
      for any $a, \in A$, $\mu(a, 1) = \mu(1, a) = a$.
      ($1$ is the \textbf{unit element} or \textbf{unity}.)
    \item
      We say that $(A, \mu)$ is \textbf{alternating algebra} if
      for any $a \in A$, $\mu(a, a) = 0$.
    \item
      We say that $(A, \mu)$ is \textbf{anti-commutative algebra} if
      for any $a, b \in A$, $\mu(a, b) = - \mu(b, a)$.
  \end{enumerate}
\end{definition}
