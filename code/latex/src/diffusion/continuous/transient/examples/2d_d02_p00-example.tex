\begin{example}
  \label{cmc/diffusion/continuous/transient/examples/2d_d02_p00-example}
  Consider the transient continuous heat transport problem
  (\Cref{cmc/diffusion/continuous/transient/primal_strong-formulation},
   \Cref{cmc/diffusion/continuous/transient/primal_weak-formulation},
   \Cref{cmc/diffusion/continuous/transient/mixed_weak-formulation})
  with input data \verb|2d_d02_p00| in the nomenclature of the C codebase.

  Concretely,
    $X = [0, 20] \times [0, 15]$,
    $\tilde{\pi} \equiv 4$,
    $\tilde{\kappa} \equiv 6$,
    $u_0(x, y) = \begin{cases} 100, & x = 0 \\ 0, & x > 0 \end{cases}$,
    $f \equiv 0$,
    $G_D = \{0, 20\} \times [0, 15]$,
    $G_N = [0, 20] \times \{0, 15\}$,
    $g_D(x, y) = \begin{cases} 100, & x = 0 \\ 0, & x = 20 \end{cases}$,
    $g_N \equiv 0$.

  This problem has the following exact solution in steady-state:
  \begin{subequations}
    \begin{alignat}{3}
      & u(x, y) && = && 5 (20 - x), \\
      & q(x, y) && = && 30 \, d y.
    \end{alignat}
  \end{subequations}
  For this problem I use a mesh $M$ generated by
  \href{https://neper.info/}{Neper} with Forman subdivision $K$.
  Its potential and flow rate on $K$ consisting of the $3$ discussed cochain
  methods are shown on
  \Cref{figure:cmc/diffusion/transient/continuous_2d_d02_p00/2d_10_grains_forman_trapezoidal_0p05_1000_potential}
  and
  \Cref{figure:cmc/diffusion/transient/continuous_2d_d02_p00/2d_10_grains_forman_trapezoidal_0p05_1000_flow_rate}.
\end{example}
