\begin{example}
  Consider the transient continuous heat transport problem
  (\Cref{idec/diffusion/continuous/transient/primal_strong-formulation},
   \Cref{idec/diffusion/continuous/transient/primal_weak-formulation},
   \Cref{idec/diffusion/continuous/transient/mixed_weak-formulation})
  with input data \verb|2d_d03_p00| in the nomenclature of the C codebase.

  Concretely,
    $X = \set{(x, y) \in \R^2}{x^2 + y^2 \leq 1}$,
    $\pi_0 \equiv 4$,
    $\kappa_1 \equiv 1$,
    $u_0(x, y) = 2 - (x^2 + y^2)$,
    $f \equiv -4\, d x \wedge d y$,
    $G_D = \partial X$,
    $G_N = \emptyset$,
    $g_D \equiv 1$.

  This problem has the following exact solution in steady-state:
  \begin{subequations}
    \begin{alignat}{3}
      & u(x, y) && = && x^2 + y^2, \\
      & q(x, y) && = && -2 y\, d x + 2 x\, d y.
    \end{alignat}
  \end{subequations}
  Consider a mesh $M$ for $X$ consisting of $n_a$ rays and $n_d$ disks
  with Forman subdivision $K$.
  Its potential and flow on $K$ consisting of the exact solution and the $3$
  discussed cochain methods are shown on
  \Cref{figure:idec/diffusion/transient/continuous_2d_d03_p00/circular_4_3_forman_trapezoidal_0p05_1000}
  for $(n_a, n_d) = (4, 3)$.

  {\color{red} At the moment there are problems with the mixed method!!!}
\end{example}
