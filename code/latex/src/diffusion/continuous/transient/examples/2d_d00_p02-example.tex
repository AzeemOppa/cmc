\begin{example}
  \label{idec/diffusion/continuous/transient/examples/2d_d00_p02-example}
  Consider the transient continuous heat transport problem
  (\Cref{idec/diffusion/continuous/transient/primal_strong-formulation},
   \Cref{idec/diffusion/continuous/transient/primal_weak-formulation},
   \Cref{idec/diffusion/continuous/transient/mixed_weak-formulation})
  with input data \verb|2d_d00_p02| in the nomenclature of the C codebase.

  Concretely,
    $X = [0, 1]^2$,
    $\pi_0 \equiv 1$,
    $\kappa_1 \equiv 1$,
    $u_0(x, y) = x^2 + y^2$,
    $f \equiv - 4\, d x \wedge d y$,
    $G_D = \partial X$,
    $G_N = \emptyset$,
    $g_D(x, y) = x^2 + y^2$.

  This problem is in steady-state mode from the beginning
  with the following exact solution:
  \begin{subequations}
    \begin{alignat}{3}
      & u(x, y) && = && x^2 + y^2, \\
      & q(x, y) && = && -2 y\, d x + 2 x\, d y.
    \end{alignat}
  \end{subequations}
  Consider a mesh $M$ for $X$ consisting of $2 \times 2$ squares (each axis is
  divided into $2$ segments) with Forman subdivision $K$ ($4 \times 4$ squares).
  Its potential and flow rate on $K$ consisting of the $3$ discussed cochain
  methods are shown on
  \Cref{figure:idec/diffusion/transient/continuous_2d_d00_p02/brick_2d_5_forman_trapezoidal_0p001_1000_potential}
  and
  \Cref{figure:idec/diffusion/transient/continuous_2d_d00_p02/brick_2d_5_forman_trapezoidal_0p001_1000_flow_rate}.
\end{example}
