\begin{example}
  \label{cmc/diffusion/continuous/transient/examples/2d_d00_p03-example}
  Consider the transient continuous heat transport problem
  (\Cref{cmc/diffusion/continuous/transient/primal_strong-formulation},
   \Cref{cmc/diffusion/continuous/transient/primal_weak-formulation},
   \Cref{cmc/diffusion/continuous/transient/mixed_weak-formulation})
  with input data \verb|2d_d00_p03| in the nomenclature of the C codebase.

  Concretely,
    $X = [0, 1]^2$,
    $\pi_0 \equiv 4$,
    $\kappa_1 \equiv 1$,
    $u_0(x, y) = 0$,
    $f \equiv -2\, d x \wedge d y$,
    $G_D = \{0, 1\} \times [0, 1]$,
    $G_N = [0, 1] \times \{0, 1\}$,
    $g_D(x, y) = 0$,
    $g_N \equiv 0$.

  This problem has the following exact solution in steady-state:
  \begin{subequations}
    \begin{alignat}{3}
      & u(x, y) && = && x (x - 1), \\
      & q(x, y) && = && - (2 x - 1)\, d y.
    \end{alignat}
  \end{subequations}
  Consider a mesh $M$ for $X$ consisting of $5 \times 5$ squares (each axis is
  divided into $5$ segments) with Forman subdivision $K$
  ($10 \times 10$ squares).
  Its potential and flow rate on $K$ consisting of the $3$ discussed cochain
  methods are shown on
  \Cref{figure:cmc/diffusion/transient/continuous_2d_d00_p03/brick_2d_5_forman_trapezoidal_0p001_2500_potential}
  and
  \Cref{figure:cmc/diffusion/transient/continuous_2d_d00_p03/brick_2d_5_forman_trapezoidal_0p001_2500_flow_rate}.
\end{example}
