\begin{example}
  \label{idec/diffusion/continuous/steady_state/examples/2d_d04_p00-example}
  Consider the steady-state continuous heat transport problem
  (\Cref{idec/diffusion/continuous/steady_state/primal_strong-formulation},
   \Cref{idec/diffusion/continuous/steady_state/primal_weak-formulation},
   \Cref{idec/diffusion/continuous/steady_state/mixed_weak-formulation})
  with input data \verb|2d_d04_p00| in the nomenclature of the C codebase.

  Concretely,
    $X = \set{(x, y, z) \in \R^3}{x^2 + y^2 + z^2 = 1,\ z \geq 0}$ be a
      hemisphere with the induced metric,
    $\kappa_1 \equiv 2$,
    $f = 6 \kappa (x^2 - y^2) \vol$,
    $G_D = \partial X$,
    $G_N = \emptyset$,
    $g_D(x, y, z) = x^2 - y^2$.

  Use spherical coordinates
  \begin{equation}
    (x, y, z)
    = (\sin \theta \cos \varphi, \sin \theta \sin \varphi, \cos \theta),\
    0 \leq \theta \leq \pi / 2,\
    0 \leq \varphi \leq 2 \pi.
  \end{equation}
  This problem has the following exact solution:
  \begin{subequations}
    \begin{alignat}{3}
      & u(x, y, z) && = && x^2 - y^2, \\
      & \tilde{q}(\theta, \varphi) &&
      = && 2 \kappa (2 \sin \theta \sin(2 \varphi)\, d \theta
                     + \sin \theta \sin(2 \theta) \cos (2 \varphi)\, d \varphi).
    \end{alignat}
  \end{subequations}
  Consider a mesh $M$ for $X$ consisting of $n_a$ meridians and $n_d$ parallels
  with Forman subdivision $K$.
  Its potential and flow rate on the $xy$-projection of $K$ consisting of the
  exact solution and the $2$ of the discussed cochain methods (no primal strong)
  are shown on
  \Cref{figure:idec/diffusion/steady_state/continuous_2d_d04_p00/hemisphere_polar_4_3_forman_potential},
  \Cref{figure:idec/diffusion/steady_state/continuous_2d_d04_p00/hemisphere_polar_4_3_forman_flow_rate}
  ($(n_a, n_d) = (4, 3)$)
  and
  \Cref{figure:idec/diffusion/steady_state/continuous_2d_d04_p00/hemisphere_polar_6_6_forman_potential},
  \Cref{figure:idec/diffusion/steady_state/continuous_2d_d04_p00/hemisphere_polar_6_6_forman_flow_rate}
  ($(n_a, n_d) = (6, 6)$).
\end{example}
