\begin{discussion}
  \label{idec/diffusion/discrete/steady_state/mixed_weak_solve-discussion}
  We are going to derive a solution to
  \Cref{idec/diffusion/discrete/steady_state/mixed_weak-formulation}.
  For any $p \in \{0, ..., d\}$ denote
  \begin{equation}
    n_p := \abs{K_p} = \dim(C_p K) = \dim(C^p K).
  \end{equation}
  The cochains $(c^{p, 0}, ..., c^{p, n_0 - 1})$ form the standard basis of
  $C^p K$.
  Define the diagonal matrix $A \in M_{n_{d - 1} \times n_{d - 1}}(\R)$,
  and the sparse matrix $B \in M_{n_d \times n_{d - 1}}(\R)$ by
  \begin{subequations}
    \begin{alignat}{3}
      & A_{i, j}
      && := \inner{c^{d - 1, j}}{\kappa_{d - 1}^{-1} c^{d - 1, i}}, \enspace
      && i, j = 0, ..., n_{d - 1} - 1, \\
%
      & B_{k, i}
      && := \inner{\delta_{d - 1} c^{d - 1, i}}{c^{d, k}}, \enspace
      && k = 0, ..., n_d - 1,\ i = 0, ..., n_{d - 1} - 1.
    \end{alignat}
  \end{subequations}
  th
  and the vectors $F \in \R^{n_d}$, $G \in \R^{n_{d - 1}}$ by
  \begin{subequations}
    \begin{alignat}{3}
      & F_k
      && := \inner{f^d}{c^{d, k}}, \enspace
      && k = 0, ..., n_d - 1, \\
%
      & G_i
      && := (\tr_{\Gamma_D, d - 1} c^{d - 1, i} \smile g_D^0)[\Gamma_D],
        \enspace
      && i = 0, ..., n_{d - 1} - 1.
    \end{alignat}
  \end{subequations}
  Denote the unknown coefficients of $q^{d - 1}$ as
  $\{x_j\}_{j = 0}^{n_{d - 1} - 1}$, i.e.,
  \begin{equation}
    q^{d - 1} = \sum_{j = 0}^{n_{s - 1} - 1} x_j c^{d - 1, j},
  \end{equation}
  and the unknown coefficients of $\tilde{u}^d$ as
  $\{y_k\}_{k = 0}^{n_d - 1}$, i.e.,
  \begin{equation}
    \tilde{u}^d = \sum_{k = 0}^{n_d - 1} y_k c^{d, k}.
  \end{equation}
  Finally, let $J$ be the set of $(d - 1)$-cells on the Neumann boundary
  $\Gamma_N$, and $\overline{J} := \{0, ..., n_{d - 1} - 1\} \setminus I$.
  We get the system
  \begin{subequations}
    \begin{alignat}{3}
      & \sum_{j = 0}^{n_{d - 1} - 1} A_{i, j} x_j
        - \sum_{k = 0}^{n_d - 1} (B^T)_{i, k} y_k
      && = G_i, \enspace
      && i \in \overline{J}, \\
%
      & \sum_{i = 0}^{n_{d - 1} - 1} B_{k, i} x_i
      && = - F_k, \enspace
      && i \in \overline{J}, \\
%
      & x_i
      && = g_N^{d - 1}(c{d - 1, i}), \enspace
      && i \in J.
    \end{alignat}
  \end{subequations}
  This leads to the system of equations
  \begin{subequations}
    \begin{alignat}{3}
      & \sum_{j \in \overline{J}} A_{i, j} x_j
        - \sum_{k = 0}^{n_d - 1} (B^T)_{i, k} y_k
      && = G_i - \sum_{j \in J} A_{i, j} g_N^{d - 1}(c_{d - 1, j}), \enspace
      && i \in \overline{J}, \\
%
      & \sum_{i \in J} B_{k, i} x_i
      && = - F_k - \sum_{i \in J} B_{k, i} g_N^{d - 1}(c_{d - 1, i}), \enspace
      && i \in \overline{J}.
    \end{alignat}
  \end{subequations}
  (Note that since $A$ is diagonal, $A_{i, j} = 0$ when $i \in \overline{J}$ and
  $j \in J$.)
  Denote by $\overline{A},\ \overline{B},\ \overline{F},\ \overline{G},\
  \overline{x}$ the espective restricted matrices and vectors.
  We get the system of equations
  \begin{subequations}
    \begin{alignat}{2}
      & \overline{A} \overline{x} + \overline{B}^T y
      && = \overline{G}, \\
%
      & \overline{B} \overline{x}
      && = - \overline{F}.
    \end{alignat}
  \end{subequations}
  In general, when $A$ is sparse but not diagonal, it is not beneficial to use
  the inverse of $A$ in calculations since it will be a dense matrix.
  (This is the case in mixed finite element methods.)
  However, in our case $A$ is diagonal, so the following calculation makes sense
  computationally.
  We can solve for $\overline{x}$ by
  \begin{equation}
    \overline{x} = \overline{A}^{-1} (\overline{G} - \overline{B}^T y).
  \end{equation}
  Hence,
  \begin{equation}
    - \overline{F}
    = \overline{B} \overline{x}
    = \overline{B} \overline{A}^{-1} (\overline{G} - \overline{B}^T y),
  \end{equation}
  which translates to
  \begin{equation}
    \overline{B} \overline{A}^{-1} \overline{B}^T y
    = \overline{B} \overline{A}^{-1} \overline{G} + \overline{F}.
  \end{equation}
  Using the Cholesky decomposition, we can solve for $y$ and then substitute to
  find $\overline{x}$.
\end{discussion}
