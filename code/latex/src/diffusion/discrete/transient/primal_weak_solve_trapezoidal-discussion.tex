\begin{discussion}
  \label{cmc/diffusion/discrete/transient/primal_weak_solve_trapezoidal-discussion}
  We are going to derive a solution to
  \Cref{cmc/diffusion/discrete/transient/primal_weak-formulation}
  using the trapezoidal rule for time integration.
  We will assume that the heat capacity $\pi_0$ is time-independent which will
  allow us to rearrange the time derivative:
  \begin{equation}
    B(v^0, \frac{\partial u^0} {\partial t}) = \frac{d}{d t} B(v^0, u^0).
  \end{equation}
  For further simplicity we will also assume that all the rest input data (heat
  source, thermal conductivity, boundary conditions) are also time-independent.
  Denote $H := F - G$. We can then integrate the equation
  \begin{equation}
    \frac{d}{d t} B(v^0, u^0) + A(v^0, u^0) = H(v^0)
  \end{equation}
  with respect to $t$ in the interval $[\alpha, \beta] \subset I$ to get
  \begin{equation}
    B(v^0, u^0(\beta)) - B(v^0, u^0(\alpha))
    + A(v^0, \int_\alpha^\beta u^0\, d t)
    = (\beta - \alpha) H(v^0).
  \end{equation}
  For an interval $[\alpha, \beta]$ the trapezoidal rule gives the approximation
  \begin{equation}
    A(v^0, \int_\alpha^\beta u^0\, d t)
    \approx A(v^0, \frac{\beta - \alpha}{2} (u^0(\alpha) + u^0(\beta))).
  \end{equation}
  Hence, if we partition $I$ into segements with size $\tau$ with moments of
  time $\{t_i := t_0 + \tau i\}_{i \geq 0}$, and if we denote
  $\{y^i := u^0(t_i)\}_{i \geq 0}$, we get
  \begin{equation}
    B(v^0, y^{i + 1}) - B(v^0, y^i)
    + \frac{\tau}{2} (A(v^0, y^i) + A(v^0, y^{i + 1}))
    = \tau H(v^0).
  \end{equation}
  The above equation is restated as
  \begin{equation}
    (B - \frac{\tau}{2} A)(v^0, y^{i + 1})
    = (B + \frac{\tau}{2} A)(v^0, y^i) + \tau H(v^0).
  \end{equation}
  Define the left-hand side and right-hand side operators
  \begin{subequations}
    \begin{alignat}{1}
      & L_\tau := B - \frac{\tau}{2} A, \\
      & R_\tau := B + \frac{\tau}{2} A.
    \end{alignat}
  \end{subequations}
  We arrive at the following problem: find $\{y^i \in C^0 K\}_{i \geq 0}$
  such that
  \begin{subequations}
    \begin{alignat}{4}
      & \forall v^0 [\Theta] \in \Ker \tr_{\Gamma_D, 0}, \quad
      && L_\tau(v^0, y^{i + 1})
      && = R_\tau(v^0, y^i) + \tau H(v^0), \enspace
      && i \geq 0, \\
      %
      &
      && \tr_{\Gamma_D, 0} y^i
      && = g_D^0, \enspace
      && i > 0, \\
      %
      &
      && y^0
      && = u_0.
      &&
    \end{alignat}
  \end{subequations}
  As in the steady-state case
  \Cref{cmc/diffusion/discrete/steady_state/primal_weak_solve-discussion},
  let $I_D$ be the set of nodes on the Dirichlet
  boundary $\Gamma_D$,
  and $\overline{I_D} := \{0, ..., n_0 - 1\} \setminus I_D$.
  Denote the unknown coefficients of $y^i$ as $\{x^i_j\}_{j = 0}^{n_0 - 1}$,
  i.e.,
  \begin{equation}
    y^i = \sum_{j = 0}^{n_0 - 1} x^i_j N^j.
  \end{equation}
  In an analogous derivation to the one in
  \Cref{cmc/diffusion/discrete/steady_state/primal_weak_solve-discussion},
  let $R_\tau^{\overline{I_D}}$ be the matrix in the standard basis of the
  restriction of $R_\tau$ to the rows and colums in $\overline{I_D}$,
  $x^{\overline{I_D}, i}$ be the restriction of $x^i$ on $\overline{I_D}$,
  and $b_\tau^{\overline{I_D}, i}$ be a vector with indices in $\overline{I_D}$,
  defined by
  \begin{equation}
    {b_\tau^{\overline{I_D}, i}}_j := {(S_\tau y^i)}_j + \tau H_j -
    \sum_{k \in I_D} (R_\tau)_{j, k} g_D^0(N_k).
  \end{equation}
  This vector can also be written as
  \begin{equation}
    \overline{b}_\tau^i := \overline{(S_\tau y^i)} + \overline{c_\tau},
  \end{equation}
  where $\overline{(S_\tau y^i)}$ is the restriction of $S_\tau y^i$ on
  $\overline{I}$ and
  \begin{equation}
    \overline{c_\tau}
    := \tau H_j - \sum_{k \in I_D} (R_\tau)_{j, k} g_D^0(N_k),\
    j \in \overline{I}
  \end{equation}
  Hence, at each time step we get the equation
  \begin{equation}
    R_\tau^{\overline{I_D}} x^{\overline{I_D}, i} = b_\tau^{\overline{I_D}, i},
  \end{equation}
  This leads to the the following iterative process.
\end{discussion}
