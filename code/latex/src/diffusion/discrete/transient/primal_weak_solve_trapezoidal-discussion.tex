\begin{discussion}
  \label{cmc/diffusion/discrete/transient/primal_weak_solve_trapezoidal-discussion}
  We are going to derive a solution to
  \Cref{cmc/diffusion/discrete/transient/primal_weak-formulation}
  using the trapezoidal rule for time integration.
  We will assume that the heat capacity $\tilde{\pi}$ is time-independent which will
  allow us to rearrange the time derivative:
  \begin{equation}
    B(v, \frac{\partial u} {\partial t}) = \frac{d}{d t} B(v, u).
  \end{equation}
  For further simplicity we will also assume that all the rest input data
  internal production rate, thermal conductivity, boundary conditions)
  are also time-independent.
  Denote $H := F - G$.
  We can then integrate the equation
  \begin{equation}
    \frac{d}{d t} B(v, u) + A(v, u) = H(v)
  \end{equation}
  with respect to $t$ in the interval $[\alpha, \beta] \subset I$ to get
  \begin{equation}
    B(v, u(\beta)) - B(v, u(\alpha))
    + A(v, \int_\alpha^\beta u\, d t)
    = (\beta - \alpha) H(v).
  \end{equation}
  For an interval $[\alpha, \beta]$ the trapezoidal rule gives the approximation
  \begin{equation}
    A(v, \int_\alpha^\beta u\, d t)
    \approx A(v, \frac{\beta - \alpha}{2} (u(\alpha) + u(\beta))).
  \end{equation}
  Hence, if we partition $I$ into segements with size $\tau$ with moments of
  time $\{t_s := t_0 + \tau s\}_{s \geq 0}$, and if we denote
  $\{U^s := u(t_s)\}_{s \geq 0}$, we get
  \begin{equation}
    B(v, U^s) - B(v, U^{s - 1})
    + \frac{\tau}{2} (A(v, U^{s - 1}) + A(v, U^s))
    = \tau H(v).
  \end{equation}
  The above equation is restated as
  \begin{equation}
    (B - \frac{\tau}{2} A)(v, U^s)
    = (B + \frac{\tau}{2} A)(v, U^{s - 1}) + \tau H(v).
  \end{equation}
  Define the left-hand side and right-hand side operators
  \begin{subequations}
    \begin{alignat}{1}
      & L_\tau := B - \frac{\tau}{2} A, \\
      & R_\tau := B + \frac{\tau}{2} A.
    \end{alignat}
  \end{subequations}
  The initial condition corresponds to $U^0 = u_0(t_0, \cdot)$.
  At any moment $s > 0$ we get the following problem for $U^s \in C^0 K$:
  \begin{subequations}
    \begin{alignat}{3}
      & \forall v [\Theta] \in \Ker \tr_{\Gamma_D, 0}, \quad
      && L_\tau(v, U^s)
      && = R_\tau(v, U^{s - 1}) + \tau H(v), \\
      %
      &
      && \tr_{\Gamma_D, 0} U^s
      && = g_D.
    \end{alignat}
  \end{subequations}
  As in the steady-state case
  \Cref{cmc/diffusion/discrete/steady_state/primal_weak_solve-discussion},
  let $J$ be the set of nodes on the Dirichlet boundary $\Gamma_D$,
  and $\overline{J} := \{0, ..., n_0 - 1\} \setminus J$.
  Denote the unknown coefficients of $U^s$ as
  $\{{\bf U}^s_j\}_{j = 0}^{n_0 - 1}$,
  i.e.,
  \begin{equation}
    U^s = \sum_{j = 0}^{n_0 - 1} {\bf U}^s_j N^j.
  \end{equation}
  In an analogous derivation to the one in
  \Cref{cmc/diffusion/discrete/steady_state/primal_weak_solve-discussion},
  let $\overline{{\bf L}_\tau}$ be the matrix in the standard basis of the
  restriction of ${\bf L}_\tau$ to the rows and colums in $\overline{J}$,
  $\overline{\bf U}^s$ be the restriction of ${\bf U}^s$ on $\overline{J}$,
  and $\overline{{\bf H}_\tau} \in \R^{\abs{\overline{J}}}$
  be the vector defined by
  \begin{equation}
    \overline{{\bf H}_\tau}
    := \tau {\bf H}_i - \sum_{j \in J} ({\bf L}_\tau)_{i, j} g_D(N_j),\
    i \in \overline{J}.
  \end{equation}
  % where ${\bf S}_\tau$ realises the Cholesky decomposition of ${\bf R}_\tau$.
  This leads to the system
  \begin{equation}
    \overline{{\bf L}_\tau}\, \overline{{\bf U}}^s
    = \overline{{\bf R}_\tau {\bf U}^{s - 1}}
    + \overline{{\bf H}_\tau},
  \end{equation}
  where $\overline{{\bf R}_\tau {\bf U}^{s - 1}}$ is the restriction of
  ${\bf R}_\tau {\bf U}^{s - 1}$ to $\overline{J}$. 
  % This vector can also be written as
  % \begin{equation}
  %   \overline{{\bf H}_\tau}^s
  %   := \overline{(S_\tau {\bf U}^s)} + \overline{{\bf C}_\tau},
  % \end{equation}
  % where $\overline{{\bf S}_\tau {\bf U}^s}$ is the restriction of
  % ${\bf S}_\tau {\bf U}^s$ on $\overline{J}$ and
  % \begin{equation}
  %   \overline{{\bf C}_\tau}
  %   := \tau {\bf H}_i - \sum_{j \in J} ({\bf R}_\tau)_{i, j} g_D(N_j),\
  %   i \in \overline{J}
  % \end{equation}
  % Hence, at each time step we get the equation
  % \begin{equation}
  %   \overline{{\bf R}_\tau} \overline{\bf U}^s = \overline{{\bf H}_\tau}^s,
  % \end{equation}
  This leads to the the following iterative process.
\end{discussion}
