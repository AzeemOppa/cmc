\begin{algorithm}[Algorithm for solving the transient mixed weak formulation
  for the discrete heat transfer phenomenon using trapezoidal rule for time
  integration, assuming time-independent input data]
  \label{cmc/diffusion/discrete/transient/mixed_weak_solve_trapezoidal-algorithm}
  Let:
  \begin{itemize}
    \item
      Let $D$ be a positive integer (space dimension);
    \item
      $K$ be an oriented quasi-cubical \hyperref[cmc:mesh:definition]{mesh} of
      dimension $D$ representing the material body;
    \item
      $[K]$ be the fundamental class of $K$;
    \item
      $t_0 [T] \in \R$ be the initial time;
    \item
      $\tau [T] \in \R^+$ be the time step;
    \item
      $f [E T^{-1}] \in C^D K$ be the internal production rate;
    \item
      $u_0 [\Theta] \in C^0 K$ be the initial temperature;
    \item
      $\pi [E L^{-D} \Theta^{-1}] \colon I \times C^D K \to C^D K$
      be the heat capacity of the material;
    \item
      $\kappa [E L^{2 - D} T^{-1} \Theta^{-1}]
        \colon C^{D - 1} K \to C^{D - 1} K$
      be the thermal conductivity of the material;
    \item
      $\partial K = \Gamma_D \cup \Gamma_N$ be the partition of the boundary of
      $K$ into Dirichlet ($\Gamma_D$) and Neumann ($\Gamma_N$) regions;
    \item
      $[\Gamma_D]$ be the fundamental class of $\Gamma_D$, where $\Gamma_D$
      has the boundary orientation induced from $K$;
    \item
      $g_D [\Theta] \in C^0 \Gamma_D$
      be the prescribed temperature on the Dirichlet boundary;
    \item
      $g_N [E T^{-1}] \in C^{D - 1} \Gamma_N$
      be the prescribed flow rate on the Neumann boundary.
  \end{itemize}
  Our algorithm has $3$ phases.
  \begin{enumerate}
    \item
      \textbf{Time-independent phase.}
      Calculate:
      \begin{itemize}
        \item
          $n_p := \abs{K_p}$, $p = D - 1$ and $p = D$;
        \item
          the diagonal matrix ${\bf A} \in M_{n_{D - 1} \times n_{D - 1}}(\R)$,
          \begin{equation}
            {\bf A}_{i, j}
            := \inner{c^{D - 1, j}}
              {\kappa^{-1} c^{D - 1, i}}_{K, D - 1},\
            i, j = 0, ..., n_{D - 1} - 1;
          \end{equation}
        \item
          the sparse matrix ${\bf B} \in M_{n_d \times n_{D - 1}}(\R)$,
          \begin{equation}
            {\bf B}_{k, i}
            := \inner{\delta_{D - 1} c^{D - 1, i}}{c^{D, k}}_{K, D},\
            i = 0, ..., n_{D - 1} - 1, k = 0, ..., n_d - 1;
          \end{equation}
        \item
          the diagonal matrix ${\bf C} \in M_{n_d \times n_d}(\R)$,
          \begin{equation}
            {\bf C}_{k, l} := \inner{\pi c^{D, l}}{c^{D, k}}_{K, D},\
            k, l = 0, ..., n_d - 1;
          \end{equation}
        \item
          the (sparse) vector ${\bf G} \in \R^{n_{D - 1}}$,
          \begin{equation}
            {\bf G}_i
            :=(\tr_{\Gamma_D, D - 1} c^{D - 1, i} \smile g_D)[\Gamma_D],\
            i = 0, ..., n_{D - 1} - 1;
          \end{equation}
        \item
          the vector ${\bf F} \in \R^{n_d}$,
          \begin{equation}
            {\bf F}_k := \inner{f}{c^{D, k}}_{K, D},\ k = 0, ..., n_d - 1;
          \end{equation}
        \item
          the sets
          \begin{subequations}
            \begin{alignat}{2}
              & J
              && := \set{i \in \{0, ..., n_{D - 1}\}}
                {c_{D - 1, i} \in (\Gamma_N)_{D - 1}}, \\
              %
              & \overline{J}
              && := \{0, ..., n_{D - 1}\} \setminus J;
            \end{alignat}
          \end{subequations}
        \item
          the restricted diagonal matrix
          $\overline{\bf A} \in
            M_{\abs{\overline{J}} \times \abs{\overline{J}}}(\R)$
          constructed out of ${\bf A}$ with rows and columns in $J$ removed;
        \item
          the restricted sparse matrix
          $\overline{\bf B} \in M_{n_d \times \abs{\overline{J}}}(\R)$
          constructed out of ${\bf B}$ with columns in $J$ removed;
        \item
          the modified and restricted vector
          $\overline{\bf G} \in \R^{\abs{\overline{J}}}$,
          \begin{equation}
            \overline{\bf G}_i :=
            {\bf G}_i
            + \sum_{j \in J} {\bf A}_{i, j} g_N(c_{D - 1, j}),\
            i \in \overline{J}
          \end{equation}
          (in our case  $A$ is diagonal and so for all $i \in \overline{J}$
          we get $\overline{\bf G}_i = {\bf G}_i$, i.e., no modification);
        \item
          the modified vector $\widetilde{\bf F} \in \R^{n_d}$,
          \begin{equation}
            \widetilde{\bf F}_k
            := 2 {\bf F}_k
              - \sum_{i \in J} {\bf B}_{k, i} g_N(c_{D - 1, i}),\
            k = 0, ..., n_d - 1;
          \end{equation}
        \item
          the left-hand side matrix ${\bf N}_\tau \in M_{n_d \times n_d}(\R)$,
          \begin{equation}
            {\bf N}_\tau
            := \overline{\bf B}\, \overline{\bf A}^{-1} \overline{\bf B}^T
              + \frac{2}{\tau} {\bf C};
          \end{equation}
        \item
          the lower-triangular positive definite sparse matrix
          ${\bf L}_\tau \in M_{n_d \times n_d}(\R)$
          realising the Cholesky decomposition
          \begin{equation}
            {\bf N}_\tau = {\bf L}_\tau {\bf L}^T_\tau;
          \end{equation}
        \item
          the time-independent part of the heat flow rate
          $\overline{\bf P} \in \R^{\abs{\overline{J}}}$,
          \begin{equation}
            \overline{\bf P} := \overline{\bf A}^{-1} \overline{\bf G};
          \end{equation}
        \item
          the time-independent matrix multiplier
          $\overline{\bf R} \in M_{\abs{\overline{J}} \times n_d}(\R)$,
          \begin{equation}
            \overline{\bf R} := \overline{\bf A}^{-1} \overline{\bf B}^T;
          \end{equation}
        \item
          the constant right-hand side vector ${\bf Z} \in \R^{n_d}$,
          \begin{equation}
            {\bf Z} := \overline{\bf B} \overline{\bf P} + \widetilde{\bf F};
          \end{equation}
        \item
          the time-independent part of the dual temperature
          ${\bf V}_\tau \in \R^{n_d}$,
          \begin{equation}
            {\bf V}_\tau
            := {\bf N}_\tau^{-1} {\bf Z}
            = {\bf L}_\tau^{-T} {\bf L}_\tau^{-1} {\bf Z}
          \end{equation}
          (calculated with forward and back substitution);
        \item
          the initial cooefficients
          ${\bf U}^0 \in \R^{n_d}$ of the dual temperature,
          and ${\bf Q}^0 \in \R^{n_{D - 1}}$ of the heat flow rate:
          \begin{subequations}
            \begin{alignat}{2}
              & {\bf Q}^0_i := ({\bf {\rm flow\_rate}}({\bf u}_0))_i,\
              && i = 0, ..., n_{D - 1} - 1, \\
              %
              & {\bf U}^0_k := ({\bf \star}_0 {\bf u}_0)_k,\
              && k = 0, ..., n_d - 1.
            \end{alignat}
          \end{subequations}
      \end{itemize}
      Allocate memory for the following vectors, to be modified at each step in
      the looping phase (superscript index $s$ only shows their time relevance):
      \begin{itemize}
        \item
          time-dependent part of the right-hand side
          ${\bf Y}_\tau^s \in \R^{n_d}$;
        \item
          time-dependent part of the solution
          ${\bf W}_\tau^s \in \R^{n_d}$;
        \item
          the restricted coordinates
          $\overline{\bf Q}^{s + 1} \in \R^{\abs{\overline{J}}}$.
      \end{itemize}
    \item
      \textbf{Time-dependent (loop) phase.}
      The constant input consists of
      $g_N,\ {\bf B},\ \overline{\bf B},\ {\bf C},\ \tau,\
      {\bf L}_\tau,\ {\bf V}_\tau,\ \overline{\bf P}$.
      Temporary mutable variables include the vectors
      ${\bf Y}_\tau^s,\ {\bf W}_\tau^s,\ \overline{\bf Q}^{s + 1}$.
      The output consists of the coordinates ${\bf Q}$ of the heat flow rate,
      and the coordinates ${\bf U}$ of dual temperature.
      ${\bf Q}$ and ${\bf U}$ are either pre-allocated as arrays of size
      $(\verb|number_of_time_steps| + 1) \times n_p$
      (for $p = D - 1$ and $p = D$ respectively) and initialized,
      or are only initialized, and memory is allocated at each step until some
      error condition is satisfied
      (e.g., the relative error between two consecutive steps becomes below some
      $\varepsilon > 0$, in which case the system converges to steady-state).

      For any $s > 0$ (until some predefined final moment is reached or some
      condition for small error is fulfilled), calculate:
      \begin{itemize}
        \item
          RHS term ${\bf Y}_\tau^s$:
          \begin{equation}
            {\bf Y}_\tau^s
            := - {\bf B} {\bf Q}^{s - 1}
            + \frac{2}{\tau} {\bf C} {\bf U}^{s - 1};
          \end{equation}
        \item
          solution term ${\bf W}_\tau^s$:
          \begin{equation}
          {\bf W}_\tau^s
            := {\bf N}_\tau^{-1} {\bf Y}_\tau^s
            = {\bf L}_\tau^{-T} {\bf L}_\tau^{-1} {\bf Y}_\tau^s;
          \end{equation}
          (calculated with forward and back substitution);
        \item
          the coefficients of the dual temperature ${\bf U}^s$
          (stored for all $s$),
          \begin{equation}
            {\bf U}^s := {\bf V}_\tau + {\bf W}_\tau^s;
          \end{equation}
        \item
          the non-Neumann coefficients of the heat flow rate ${\bf Q}^s$,
          \begin{equation}
            \overline{\bf Q}^s
            := - \overline{\bf P} + \overline{\bf R} {\bf U}^s;
         \end{equation}
        \item
          the heat flow rate ${\bf Q}^s \in \R^{n_{D - 1}}$
          (stored for all $s$),
          \begin{equation}
            {\bf Q}^s_j :=
            \begin{cases}
              \overline{\bf Q}^s_j, & j \in \overline{J} \\
              g_N(c_{D - 1, j}), & j \in J
            \end{cases}.
          \end{equation}
      \end{itemize}
    \item
      \textbf{Post-processing.}
      Define the sets
      \begin{subequations}
        \begin{alignat}{2}
          & J
          && := \set{i \in \{0, ..., n_0 - 1\}}{c_{0, i} \in (\Gamma_D)_0}, \\
          %
          & \overline{J}
          && := \{0, ..., n_0 - 1\} \setminus J.
        \end{alignat}
      \end{subequations}
      For each time moment $t_s$ the coordinates of the temperature
      ${\bf u}^s \in \R^{n_0}$
      in the standard basis are calculated by the formula
      \begin{equation}
        {\bf u}^s_i :=
        \begin{cases}
          ({\bf \star}_D {\bf U}^{D, s})_i, & i \in \overline{J} \\
          g_D(c_{0, i}), & i \in J
        \end{cases}.
      \end{equation}
  \end{enumerate}
\end{algorithm}
