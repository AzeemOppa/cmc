\begin{discussion}
  \label{cmc/diffusion/discrete/transient/mixed_weak_solve_trapezoidal-discussion}
  We are going to derive a solution to
  \Cref{cmc/diffusion/discrete/transient/mixed_weak-formulation}
  using the trapezoidal rule for time integration.
  We will assume that the heat capacity $\pi_0$ is time-independent which will
  allow us to rearrange the time derivative:
  \begin{equation}
    C(v^d, \frac{\partial \tilde{u}^d} {\partial t})
    = \frac{d}{d t} C(v^d, \tilde{u}^d).
  \end{equation}
  For further simplicity we will also assume that all the rest input data (heat
  source, thermal conductivity, boundary conditions) are also time-independent.
  We can then integrate the equation
  \begin{equation}
    - B(v^d, q^{d - 1}) + \frac{d}{d t} C(v^d, \tilde{u}^d) = F(v^d)
  \end{equation}
  with respect to $t$ in the interval $[\alpha, \beta] \subset I$ to get
  \begin{equation}
    - B(v^d, \int_\alpha^\beta q^{d - 1}\, d t)
    + (C(v^d, \tilde{u}^d(\beta)) - C(v^d, \tilde{u}^d(\alpha)))
    = (\beta - \alpha) F(v^d).
  \end{equation}
  If we partition $I$ into segements with size $\tau$ with moments of
  time $\{t_s := t_0 + \tau s\}_{s \geq 0}$, and if we denote
  \begin{subequations}
    \begin{alignat}{3}
      & Q^s
      && := q^{d - 1}(t_s), \enspace
      && s \geq 0, \\
      %
      & U^s
      && := \tilde{u}^d(t_s), \enspace
      && s \geq 0,
    \end{alignat}
  \end{subequations}
  we get
  \begin{equation}
    - \frac{\tau}{2} (B(v^d, Q^s) + B(v^d, Q^{s + 1}))
    + (C(v^d, U^{s + 1}) - C(v^d, U^s))
    = \tau F(v^d).
  \end{equation}
  By multiplying the above equation with $2 / \tau$ and rearranging we get:
  \begin{equation}
    - B(v^d, Q^{s + 1}) + \frac{2}{\tau} C(v^d, U^{s + 1})
    = B(v^d, Q^s) + \frac{2}{\tau} C(v^d, U^s) + 2 F(v^d).
  \end{equation}
  Define $H^s(v^d) := B(v^d, Q^s) + \frac{2}{\tau} C(v^d, U^s)$.
  We arrive at the following problem: find
  $\{(Q^s, U^s) \in C^{d - 1} K \times C^d K\}_{s \geq 0}$
  such that
  \begin{subequations}
    \begin{alignat}{4}
      & \forall r^{d - 1} [E T^{-1}] \in \Ker \tr_{\Gamma_N, D - 1}, \:
      && A(r^{d - 1}, Q^{s + 1}) - B^T(r^{d - 1}, U^{s + 1})
      && = - G(r^{d - 1}), \:
      && s \geq 0, \\
      %
      & \forall v^d [\Theta L^d] \in C^d K, \:
      && - B(v^d, Q^{s + 1}) + \frac{2}{\tau} C(v^d, U^{s + 1})
      && = H^s(v^d) + 2 F(v^d), \:
      && s \geq 0, \\
      %
      &
      && \tr_{\Gamma_N, d - 1} Q^s
      && = g_N^{d - 1}, \:
      && s > 0, \\
      %
      &
      && Q^0
      && = {\rm flow\_rate}(u_0),
      && \\
      %
      &
      && U^0
      && = \star_0 u_0,
      &&
    \end{alignat}
  \end{subequations}
  where
  \begin{equation}
    {\rm flow\_rate}(y^0)
    := ((-1)^{d - 1} \kappa_{d - 1} \star_1 \delta_0)(y^0),\
    y^0 \in C^0 K.
  \end{equation}
  Let
  \begin{subequations}
    \begin{alignat}{2}
      & J
      && := \set{i \in \{0, ..., n_{d - 1}\}}
        {c_{d - 1, i} \in (\Gamma_N)_{d - 1}}, \\
      %
      & \overline{J}
      && := \{0, ..., n_{d - 1}\} \setminus J.
    \end{alignat}
  \end{subequations}
  Initial conditions give us $Q^0$ and $U^0$.
  Denoting the matrices ${\bf A}$, ${\bf B}$, ${\bf C}$ and vectors
  ${\bf Q}^s$, ${\bf U}^s$, ${\bf F}$, ${\bf G}$
  of the corresponding operators in standard bases, we get the system
  for ${\bf Q}^s$ and ${\bf U}^s$, $s > 0$.
  \begin{subequations}
    \begin{alignat}{3}
      & \sum_{j = 0}^{n_{d - 1} - 1} {\bf A}_{i, j} {\bf Q}^{s + 1}_j
        - \sum_{k = 0}^{n_d - 1} ({\bf B}^T)_{i, k} {\bf U}^{s + 1}_k
      && = - {\bf G}_i, \:
      && i \in \overline{J}, \\
      %
      & - \sum_{i = 0}^{n_{d - 1} - 1} {\bf B}_{k, i} {\bf Q}^{s +1}
        + \frac{2}{\tau} {\bf C} {\bf U}^{s + 1}
      && = ({\bf B} {\bf Q}^s + \frac{2}{\tau} {\bf C} {\bf U}^s)_k
        + 2 {\bf F}_k, \:
      && k \in \{0, ..., n_d - 1\}, \\
      %
      & x_i
      && = g_N^{d - 1}(c_{d - 1, i}), \:
      && i \in J.
    \end{alignat}
  \end{subequations}
  This leads to the follwoing system of equations for any $s \in \N$:
  \begin{subequations}
    \begin{alignat}{3}
      & \sum_{j \in \overline{J}} {\bf A}_{i, j} {\bf Q}^{s + 1}_j
        - \sum_{k = 0}^{n_d - 1} ({\bf B}^T)_{i, k} {\bf U}^{s + 1}_k
      && = - {\bf G}_i -
        \sum_{j \in J} {\bf A}_{i, j} g_N^{d - 1}(c_{d - 1, j}), \:
      && i \in \overline{J}, \\
      %
      & - \sum_{i \in J} {\bf B}_{k, i} {\bf Q}^{s + 1}_i
        + \sum_{l = 0}^{n_d - 1}
          \frac{2}{\tau}{\bf C}_{k, l} {\bf U}^{s + 1}_l
      && = ({\bf B} {\bf Q}^s + \frac{2}{\tau} {\bf C} {\bf U}^s)_k
        + 2 {\bf F}_k + \sum_{i \in J} {\bf B}_{k, i} g_N^{d - 1}(c_{d - 1, i}),
        \:
      && i \in \overline{J}.
    \end{alignat}
  \end{subequations}
  Let $\overline{\bf A}$ be the restriction of ${\bf A}$ to the rows and colums
  in $\overline{J}$,
  $\overline{\bf B}$ be the restristion of ${\bf B}$ to the colums in
  $\overline{J}$,
  $\overline{\bf Q}^{s + 1}$ be the restriction of ${\bf Q}^{s}$ to the indices
  in $\overline{J}$,
  $\widetilde{\bf F} \in \R^{n_d}$ be defined as,
  \begin{equation}
    \widetilde{\bf F}_k
    := 2 {\bf F}_k + \sum_{i \in J} {\bf B}_{k, i} g_N^{d - 1}(c_{d - 1, i}),\
    k = 0, ..., n_d - 1,
  \end{equation}
  $\overline{\bf G}$ be the restriction of ${\bf G}$ to the indices in
  $\overline{J}$ (since $A$ is diagonal, ${\bf G}$ is not modified before
  restriction).
  For any $s \in \N$ we get the mixed system
  \begin{subequations}
    \begin{alignat}{3}
      & \overline{\bf A}\, \overline{\bf Q}^{s + 1}
      &&  - \overline{\bf B}^T {\bf U}^{s + 1}
      && = - \overline{\bf G}, \\
      %
      & - \overline{\bf B}\, \overline{\bf Q}^{s + 1}
      && + \frac{2}{\tau} {\bf C} {\bf U}^{s + 1}
      && = {\bf B} {\bf Q}^s + \frac{2}{\tau} {\bf C} {\bf U}^s
        + \widetilde{\bf F}.
    \end{alignat}
  \end{subequations}
  We can solve for $\overline{\bf Q}^{s + 1}$, $s \in \N$ as follows:
  \begin{equation}
    \overline{\bf Q}^{s + 1}
    = \overline{\bf A}^{-1}
      (- \overline{\bf G} + \overline{\bf B}^T {\bf U}^{s + 1})
    = - \overline{\bf P}
    + \overline{\bf R} {\bf U}^{s + 1},
  \end{equation}
  where we have denoted
  \begin{align}
    \overline{\bf P} & := \overline{\bf A}^{-1} \overline{\bf G}, \\
    \overline{\bf R} & := \overline{\bf A}^{-1} \overline{\bf B}^T.
  \end{align}
  Hence,
  \begin{equation}
    {\bf B} {\bf Q}^s + \frac{2}{\tau} {\bf C} {\bf U}^s + \widetilde{\bf F}
    = - \overline{\bf B}\, \overline{\bf Q}^{s + 1}
      + \frac{2}{\tau} {\bf C} {\bf U}^{s + 1}
    = - \overline{\bf B}\, \overline{\bf A}^{-1}
      (- \overline{\bf G} + \overline{\bf B}^T {\bf U}^{s + 1})
      + \frac{2}{\tau} {\bf C} {\bf U}^{s + 1}.
  \end{equation}
  This translates to
  \begin{equation}
    (\overline{\bf B}\, \overline{\bf A}^{-1} \overline{\bf B}^T
    - \frac{2}{\tau} {\bf C}) {\bf U}^{s + 1}
    = \overline{\bf B} \overline{\bf A}^{-1} \overline{\bf G}
    - \widetilde{\bf F} - {\bf B} {\bf Q}^s - \frac{2}{\tau} {\bf C} {\bf U}^s.
  \end{equation}
  Define the left-hand side matrix ${\bf N}_\tau \in M_{n_d \times n_d}(\R)$,
  \begin{equation}
    {\bf N}_\tau
    := \overline{\bf B}\, \overline{\bf A}^{-1} \overline{\bf B}^T
      - \frac{2}{\tau} {\bf C},
  \end{equation}
  and the constant right-hand side vector ${\bf Z} \in \R^{n_d}$,
  \begin{equation}
    {\bf Z}
    := \overline{\bf B} \overline{\bf A}^{-1} \overline{\bf G}
      - \widetilde{\bf F}
    = \overline{\bf B} \overline{\bf P} - \widetilde{\bf F}.
  \end{equation}
  This leads to the following linear $n_d \times n_d$ system:
  \begin{equation}
    {\bf N}_\tau {\bf U}^{s + 1}
    = {\bf Z} - {\bf B} {\bf Q}^s - \frac{2}{\tau} {\bf C} {\bf U}^s.
  \end{equation}
  Define 
  \begin{align}
    {\bf V}_\tau   & := {\bf N}_\tau^{-1} {\bf Z}, \\
    {\bf Y}_\tau^s & := {\bf B} {\bf Q}^s + \frac{2}{\tau} {\bf C} {\bf U}^s, \\
    {\bf W}_\tau^s & := {\bf N}_\tau^{-1} {\bf Y}_\tau^s.
  \end{align}
  To find ${\bf V}_\tau$ and ${\bf W}_\tau^s$ we first find the Cholesky
  decomposition of ${\bf N}_\tau$:
  \begin{equation}
    {\bf N}_\tau = {\bf L}_\tau {\bf L}_\tau^T.
  \end{equation}
  Hence,
  \begin{equation}
    {\bf U}^{s + 1}
    = {\bf N}_\tau^{-1}
      ({\bf Z} - {\bf B} {\bf Q}^s - \frac{2}{\tau} {\bf C} {\bf U}^s)
    = {\bf V}_\tau - {\bf W}_\tau^s.
  \end{equation}
  Summarasing, we get the following algorithmic procedure.
\end{discussion}
