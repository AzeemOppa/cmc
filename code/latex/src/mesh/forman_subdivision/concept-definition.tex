\begin{definition}
  Let
    $d \in \N$,
    $M$ be a \hyperref[idec:mesh:definition]{mesh} of dimension $d$.
  Consider a mesh $K$ constructed as follows.
  The nodes of $K$ are the centroids of the cells of $M$.
  (In general, the topology of $K$ can always be constructed while the geometry
  is tricky.
  For simplicity we may assume that all the cells of $M$ are convex,
  although for non-simplicial or non-brick meshes in dimensions $3$ and above
  the resulting mesh may contain non-flat polytopes.)

  For $p_f \in [0, d]$, a $p_f$-cell of $K$ is constructed as follows.
  Let $p \in [p_f, d]$, $s = p - p_f$ consider two cells
  \begin{equation}
    c(p, i) \succeq c(s, l)
  \end{equation}
  Then a $p_f$-cell is such a pair $(c(p, i), c(s, l))$.
  If $q_f \in [0, p_f]$, $q \in [q_f, d]$, $r = q - q_f$,
  \begin{equation}
    c(q, j) \succeq c(r, k),
  \end{equation}
  then $(c(q, j), c(r, k))$ is a subface of $(c(p, i), c(s, l))$ if
  \begin{equation}
    c(p, i) \succeq c(q, j) \succeq c(r, k) \succeq c(s, l).
  \end{equation}
  The constructed space is a mesh which we call the \textbf{Forman subdivision}.
\end{definition}
