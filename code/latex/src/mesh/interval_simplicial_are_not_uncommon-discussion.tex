\begin{discussion}
  As we saw, the Forman subdivision of an interval-simplicial mesh leads to a
  quasi-cubical mesh.
  Interval-simplicial meshes are not that uncommon:
  \begin{enumerate}
    \item
      all meshes of dimension at most $2$ are interval-simplicial;
    \item
      all $3$D meshes of simple polytopes are interval-simplicial;
    \item
      all simplicial and quasi-cubical meshes are interval-simplicial;
    \item
      the product of interval-simplicial meshes is an interval-simplicial mesh.
  \end{enumerate}
  For that reason we will build our calculus on quasi-cubical meshes thought as
  the Forman subdivision of an interval-simplicial mesh.
\end{discussion}
