% \begin{example}
%   Let
%     $d \in \N$,
%     $K$ be a cubical grid of dimension $d$ with the standard grid orientation,
%     $V = C_1 K$.
%   Define $\nabla \colon C^\bullet(K, C_1 K) \to C^\bullet(K, C_1 K)$ as follows.
%   Let $p \in \{0, ..., d - 1\}$ and $e_p$ be the sum of all basis $1$-chains
%   whose cells are parallel to the $p$-th basis vector in the standard basis of
%   the embedding of $K$.
%   Then we require that for each $\sigma^\bullet \in C^\bullet K$,
%   \begin{equation}
%     \nabla_p(e_p \usmile \sigma^\bullet) = e_p \usmile \delta \sigma^\bullet.
%   \end{equation}
%   Denote by
%   \begin{equation}
%     \Omega^\bullet(\abs{K}, \R^d)
%   \end{equation}
%   the space of $\R^d$-valued smooth differential forms on $\abs{K}$.
%   Let
%   \begin{equation}
%     S \colon \chi \abs{K} \to C^1 K
%   \end{equation}
%   be the discretization of vector fields into $1$-forms.
%   We define the discretization of vector-valued smooth $0$-forms,
%   \begin{equation}
%     {\bf S} \colon \Omega^0(\abs{K}, \R^d) \to C^0(K, C_1 K),
%   \end{equation}
%   as follows: if for $p = 0, ..., d - 1$,
%   $u_p \colon \abs{K} \to \R$ is smooth with discretization
%   $\eta_{(p)}^0 \in C^1 K = \widetilde{u_p}$, then
%   \begin{equation}
%     {\bf S}\left(\sum_{p = 0}^{d - 1} u_p \frac{\partial}{\partial x_p}\right)
%     := \sum_{p = 0}^{d - 1} e_p \smile \eta_{(p)}^0.
%   \end{equation}
%   Then if we take the vector gradient ${\boldsymbol \nabla}$
%   (which comes from the Levi-Civitta connection on the flat metric on
%   $\abs{K}$), then for any smooth function $f \colon \abs{K} \to \R$,
%   \begin{equation}
%     {\bf S}({\boldsymbol \nabla} f)
%     = \sum_{p = 0}^{d - 1}
%       e_p \usmile S\left(\frac{\partial}{\partial x_p} f\right).
%   \end{equation}
% \end{example}

% We will summarize the operations relating cochains and embedding.
% \begin{discussion}
%   Let
%     $d \in \N$,
%     $K$ be a quasi-cubical flat mesh of dimension $d$,
%     $X$ be the manifold it encompasses.
%   We define the following data:
%   \begin{itemize}
    % \item
    %   $R \colon \Omega^\bullet X \to C^\bullet K$ denotes the de Rham map,
    %   \begin{equation}
    %     (R_p \omega^p) c_p := \int_{\abs{c_p}} \omega^p;
    %   \end{equation}
    % \item
    %   $S \colon \Omega^\bullet(X, \chi X) \to C^\bullet(K, C^1 K)$
    %   denotes the discretization of vector-valued smooth forms into
    %   $1$-cochain valued cochains,
    %   \begin{equation}
    %     S_p(u \otimes \omega^p) := \underline{u} \otimes R_p \omega^p;
    %   \end{equation}
    % \item
    %   $\nabla \colon \chi X \to \Omega^1(X, \chi X)$ is the vector gradient
    %   (connection on vector fields);
    % \item
    %   $D \colon \Omega^\bullet(X, \chi X) \to \Omega^\bullet(X, \chi X)$
    %   is the respective covariant exterior derivative on vector-valued forms,
    %   \begin{equation}
    %     D_p(u \otimes \omega^p) :=
    %     \nabla u \uwedge \omega^p + u \otimes d_p \omega^p;
    %   \end{equation}
    % \item
    %   $\nabla^{(K)} \colon C^1 K \to C^1(K, C^1 K)$
    %   is the discrete vector gradient (discrete connection on $1$-cochains);
    % \item
    %   $D^{(K)} \colon C^\bullet(K, C^1 K) \to C^\bullet(K, C^1 K)$
    %   is the respective discrete covariant exterior derivative on
    %   $1$-cochain valued cohains,
    %   \begin{equation}
    %     D^{(K)}_p(\pi^1 \otimes \rho^p) :=
    %     \nabla^{(K)} \pi^1 \usmile \rho^p + \pi^1 \otimes \delta_p \rho^p.
    %   \end{equation}
%   \end{itemize}
% \end{discussion}
