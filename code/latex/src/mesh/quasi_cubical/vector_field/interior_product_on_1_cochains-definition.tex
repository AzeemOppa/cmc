\begin{definition}
  Let $\mathcal{K}$ be a quasi-cubical mesh.
  The \textbf{discrete interior product on $1$-cochains} is defined as
  \begin{equation}
    i \colon \mathfrak{X} K \to \Hom(C^1 \mathcal{K}, C^0 \mathcal{K}),
    i_{\mathcal{X}} \sigma := \sigma \circ \mathcal{X} \in C^0 \mathcal{K},\
    \mathcal{X} \in \mathfrak{X} K,\ \sigma \in C^1 \mathcal{K}.
  \end{equation}
  In other words, for a node $\mathcal{N}$,
  \begin{equation}
    (i_{\mathcal{X}} \sigma)(\mathcal{N})
    = (\sigma \circ \mathcal{X})(\mathcal{N})
    = \sum_{\mathcal{E} \succ \mathcal{N}}
      \mathcal{X}^{\mathcal{E}}_{\mathcal{N}} \sigma_{\mathcal{E}}.
  \end{equation}
\end{definition}
\begin{definition}
  Let
    $D \in \N^+$,
    $K$ be an orthogonal parallelotope in $\R^D$,
    whose unit directions are the vectors $e_1, ..., e_D$.
  Define a regular mesh for $K$ as follows.
  Let
    $h_1, ..., h_D \in \R^+$,
    $\mathcal{K}$ be a grid of (orthogonal) parallelotopes with sides
      $h_1, ..., h_D$.
  Let $X \in \mathfrak{X} K$,
  $X = \sum_{p = 1}^D f^p \frac{\partial}{\partial x^p}$.
  Define the approximation
  \begin{equation}
    J \colon \mathfrak{X} K \to \mathfrak{X} \mathcal{K}
  \end{equation}
  as follows.
  Let $p \in \{1, ..., D\}$,
  $\mathcal{E}$ be an edge in the direction of the basis vector $e_p$,
  $\mathcal{N}$ be a node of $\mathcal{E}$ with coordinates
  $x = (x_1, ..., x_D)$.
  Then,
  \begin{equation}
    (J X)^{\mathcal{E}}_{\mathcal{N}} :=
    \begin{cases}
      \frac{f^p(x)}{2 h_p}, & \text{$\mathcal{N}$ is an interior node} \\
      \frac{f^p(x)}{h_p}, & \text{$\mathcal{N}$ is a boundary node} \\
    \end{cases}.
  \end{equation}
\end{definition}
\begin{remark}
  Consider the setup of the previous definition.
  We will show that the discrete interior product is a good approximation of the
  continuous one.
  Precisely, let $X \in \mathfrak{X} K$, $\omega \in \Omega^1 K$.
  We will compare $i_{J X}{R_1 \omega}$ with $R i_X \omega$.
  By definition, $i_X \omega = \omega(X)$.
  In coordinates,
  \begin{align}
    X & = \sum_{p = 1}^D {f^p} \frac{\partial}{\partial x^p},\
      f^1, ..., f^D \in \mathcal{F} K, \\
    \omega & = \sum_{p = 1}^D {g_p}\, d x^p,\
      g_1, ..., g_D \in \mathcal{F} K, \\
    i_X \omega & = \sum_{p = 1}^D f^p\; g_p.
  \end{align}
  Consider a node $\mathcal{N}$ with coordinates $x = (x_1, ..., x_D)$.
  Then
  \begin{equation}
    (R i_X \omega) \mathcal{N} = \sum_{p = 1}^D f^p(x_p)\; g_p(x_p).
  \end{equation}
  On the other hand,
  \begin{equation}
    (i_{J X}{R_1 \omega}) \mathcal{N}
    = \sum_{\mathcal{E} \succ \mathcal{N}}
      (J X)^{\mathcal{E}}_{\mathcal{N}} \int_{\mathcal{E}} \omega
    = \sum_{p = 1}^D
      \sum_{\mathcal{E} \succ \mathcal{N},\ \mathcal{E} \parallel e_p}
        (J X)^{\mathcal{E}}_{\mathcal{N}} \int_{\mathcal{E}} \omega.
  \end{equation}
  In the above equation, for any $p \in \{1, ..., D\}$
  the internal sum consists of $1$ or $2$ terms:
  $1$ when $\mathcal{N}$ is on the boundary of the direction of $e_p$,
  and $2$ elsewhere.
  Define the this internal sum as $A_p$.
  We need to show it is close to $f^p(x_p)\; g_p(x_p)$.
  
  First, assume that $\mathcal{N}$ is in the interior.
  Then it is the boundary of two edges parallel to $e_p$:
  $\mathcal{E}_1$ and $\mathcal{E}_2$.
  Combined, they give the segment $\mathcal{E}$ connecting
  $x - h_p e_p$ with $x + h_p e_p$.
  Then
  \begin{equation}
    A_p = \frac{f^p(x)}{2 h_p} \int_{\mathcal{E}} \omega
    = f^p(x_p) \frac{1}{2 h_p}
      \int_{x_p - h_p}^{x_p + h_p}
      g_p(x_1, ..., x_{p - 1}, t, x_{p + 1}, ..., x_D)\, d t.
  \end{equation}
  However, from $1$D numerical analysis it follows that the approximation
  (the lowest order Gauss quadrature)
  \begin{equation}
    \frac{1}{b - a} \int_a^b g(t)\, d t \approx g\left(\frac{a + b}{2}\right)
  \end{equation}
  is exact for polynomials of degree $\leq 1$.
  In our case, substituting $a = x_p - h_p,\ b = x_p + h_p$
  implies that $f^p(x_p)\; g_p(x_p)$ is an $O(h_p^2)$ approximation of $A_p$.

  Second, assume that $\mathcal{E}$ is a boundary node.
  Fix it to be on the left.
  Then, using an analogous argument, we need to compare
  $g(a)$ with $\frac{1}{b - a} \int_a^b g(t)\, dt$
  which is the formula of left rectangles, this time exact only for constants.
  This gives total error of $O(h_p)$ in the direction of $e_p$ with $O(h^2)$ in
  the interior.
  Hence, the approximation formula is of order $O(h_1 + ... + h_D)$ with order
  $O(h_1^2 + ... + h_D^2)$ in the interior.
\end{remark}
\begin{remark}
  \begin{equation}
    i_{f \mathcal{X}} \sigma = f \smile i_\mathcal{X} \sigma.
  \end{equation}
  Indeed, for any $\mathcal{N} \in \mathcal{K}_0$,
  \begin{equation}
    (i_{f \mathcal{X}} \sigma) \mathcal{N}_\bullet
    = (\sigma \circ {f \mathcal{X}}) \mathcal{N}_\bullet
    = \sigma (f \mathcal{N}_\bullet\, \mathcal{X} \mathcal{N}_\bullet)
    = f \mathcal{N}_\bullet\, \sigma(\mathcal{X} \mathcal{N}_\bullet)
    = (f \smile i_\mathcal{X} \sigma) \mathcal{N}_\bullet.
  \end{equation}
\end{remark}
