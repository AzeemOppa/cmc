\begin{definition}
  Let
    $\mathcal{K}$ be a quasi-cubical mesh,
    $D := \dim \mathcal{K}$,
    $p \in \{0, ..., D\}$,
    $\mathcal{N} \in \mathcal{K}$.
  A $p$-\textbf{chain around} $\mathcal{N}$ is a $p$-chain such whose
  coefficients are zero on all $p$-cells not containing $\mathcal{N}$.
  By $\restrict{\Lambda_p \mathcal{K}}{\mathcal{N}}$ we will denote the space of
  all $p$-chains around $\mathcal{N}$.
  The corresponding bundle of \textbf{$p$-chain spaces} on $\mathcal{K}$
  is defined by
  \begin{equation}
    \Lambda_p \mathcal{K}
    := \coprod_{\mathcal{N} \in \mathcal{K}}
      \restrict{\Lambda_p \mathcal{K}}{\mathcal{N}}.
  \end{equation}
  (In particular, $\Lambda_1 \mathcal{K} = \mathfrak{X} K$.)
  We define the bundle of all \textbf{chain spaces} as
  \begin{equation}
    \Lambda_\bullet \mathcal{K} := \bigoplus_{p = 0}^D \Lambda_p \mathcal{K}.
  \end{equation}
\end{definition}
\begin{definition}
  Let
    $\mathcal{K}$ be a quasi-cubical mesh,
    $D := \dim \mathcal{K}$,
    $p \in \{0, ..., D\}$,
    $\mathcal{N} \in \mathcal{K}$.
  A $p$-\textbf{cochain around} $\mathcal{N}$ is a $p$-cochain whose
  coefficients are zero on all $p$-cells not containing $\mathcal{N}$.
  By $\restrict{\Lambda^p \mathcal{K}}{\mathcal{N}}$ we will denote the space of
  all $p$-cochains around $\mathcal{N}$.
  The bundle of \textbf{$p$-cochain spaces} is given by
  \begin{equation}
    \Lambda^p \mathcal{K}
    := (\Lambda_p \mathcal{K})^*
    = \coprod_{\mathcal{N} \in \mathcal{K}}
      \restrict{\Lambda^p \mathcal{K}}{\mathcal{N}}.
  \end{equation}
  The bundle of \textbf{cochain spaces} is given by
  \begin{equation}
    \Lambda^\bullet \mathcal{K}
     := \bigoplus_{p = 0}^D \Lambda^p \mathcal{K}
     \simeq (\Lambda_\bullet \mathcal{K})^*.
  \end{equation}
\end{definition}
\begin{definition}
  Let
    $\mathcal{K}$ be a quasi-cubical mesh,
    $D = \dim \mathcal{K}$,
    $p \in \{0, ..., D\}$.
  The space $\Omega^p \mathcal{K}$ of \textbf{$p$-forms} is defined by
  \begin{equation}
    \Omega^p \mathcal{K} := \Gamma(\Lambda^p \mathcal{K}).
  \end{equation}
  Correspondingly, the space of all forms $\Omega^\bullet \mathcal{K}$
  is defined by
  \begin{equation}
    \Omega^\bullet \mathcal{K}
     := \bigoplus_{p = 0}^D \Omega^p \mathcal{K}.
  \end{equation}
\end{definition}
\begin{notation}
  Let
    $\mathcal{K}$ be a quasi-cubical mesh,
    $D = \dim \mathcal{K}$,
    $p \in \{0, ..., D\}$,
    $\mathcal{N} \in \mathcal{K}$.
  We will denote
  \begin{equation}
    \Omega^p_\mathcal{N} \mathcal{K}
    := \restrict{\Lambda^p \mathcal{K}}{\mathcal{N}}.
  \end{equation}
\end{notation}
\begin{proposition}
  Let
    $\mathcal{K}$ be a quasi-cubical mesh,
    $\mathcal{N} \in \mathcal{K}$,
    $p, q \in N$,
    $\omega \in \Omega^p_\mathcal{N} \mathcal{K}$,
    $\eta \in \Omega^q_\mathcal{N} \mathcal{K}$.
  Then
  $\omega \smile \eta \in \Omega^{p + q}_\mathcal{N} \mathcal{K}$.
\end{proposition}
\begin{proof}
  Let $a \in \mathcal{K}_{p + q}$.
  Then
  \begin{equation}
    (\omega \smile \eta) a
    = \frac{1}{2^{p + q}}
      \sum_{(b, c) \in \perp_{p, q} a}
        \varepsilon^\perp(a, b, c)\, \omega(b)\, \eta(c).
  \end{equation}
  Assume that $a$ does not contain $\mathcal{N}$.
  Then it cannot contain any cells that contain $\mathcal{N}$.
  But the cells containing $\mathcal{N}$ are the only cells leading to nonzero
  contributions in $\omega$ and $\eta$.
  Therefore, in that case $(\omega \smile \eta) a = 0$.
  Hence,
  $\omega \smile \eta \in \Omega^{p + q}_\mathcal{N} \mathcal{K}$.
\end{proof}
\begin{definition}
  Let
    $\mathcal{K}$ be a quasi-cubical mesh,
    $p, q \in \N$,
    $\omega \in \Omega^p \mathcal{K}$,
    $\eta \in \Omega^q \mathcal{K}$.
  Define $\omega \smile \eta \in \Omega^{p + q} \mathcal{K}$ as follows:
  for any $\mathcal{N} \in \mathcal{K}_0$,
  \begin{equation}
    \restrict{(\omega \smile \eta)}{\mathcal{N}}
    := \restrict{\omega}{\mathcal{N}} \smile \restrict{\eta}{\mathcal{N}},
  \end{equation}
  where the product on the right is the standard cup product.
  (The definition is correct because of the previous proposition.)
\end{definition}
\begin{proposition}
  Let
    $\mathcal{K}$ be a quasi-cubical mesh,
    $\mathcal{N} \in \mathcal{K}$,
    $p \in N$,
    $\omega \in \Omega^p_\mathcal{N} \mathcal{K}$.
  Then $\delta_p \omega \in \Omega^{p + 1}_\mathcal{N} \mathcal{K}$.
\end{proposition}
\begin{proof}
  Let $a \in \mathcal{K}_{p + 1}$.
  Then
  \begin{equation}
    (\delta_p \omega) a = \omega(\partial_{p + 1} a).
  \end{equation}
  Assume that $a$ does not contain $\mathcal{N}$.
  Then none of its boundary cells contain $\mathcal{N}$.
  But the cells containing $\mathcal{N}$ are the only cells leading to nonzero
  contributions in $\omega$.
  Therefore, in that case $(\delta_p \omega) a = 0$.
  Hence,
  $\delta_p \omega \in \Omega^{p + 1}_\mathcal{N} \mathcal{K}$.
\end{proof}
\begin{definition}
  Let
    $\mathcal{K}$ be a quasi-cubical mesh,
    $p \in \N$,
    $\omega \in \Omega^p \mathcal{K}$.
  Define $\delta_p \omega \in \Omega^{p + 1} \mathcal{K}$ as follows:
  for any $\mathcal{N} \in \mathcal{K}_0$,
  \begin{equation}
    \restrict{(\delta_p \omega)}{\mathcal{N}}
    := \delta_p\left(\restrict{\omega}{\mathcal{N}}\right),
  \end{equation}
  where the $\delta_p$ on the right is the standard coboundary operator.
  (The definition is correct because of the previous proposition.)
\end{definition}
\begin{proposition}[graded Leibniz rule for discrete bundle-type cochains]
  Let
    $\mathcal{K}$ be a quasi-cubical mesh,
    $p, q \in \N$,
    $\omega \in \Omega^p \mathcal{K}$,
    $\eta \in \Omega^q \mathcal{K}$.
  Then
  \begin{equation}
    \delta_{p + q} \omega \smile \eta
    = \delta_p \omega \smile \eta + (-1)^p \omega \smile \delta_q \eta.
  \end{equation}
\end{proposition}
\begin{proof}
  Choose a node $\mathcal{N} \in \mathcal{K}_0$.
  Then, restricted to $\mathcal{N}$, both sides represend the left and right
  hand side of the standard graded Leibniz rule for cochains.
\end{proof}
