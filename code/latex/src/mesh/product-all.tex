\begin{definition}
  Let $\mathcal{K}$ and $\mathcal{L}$ be meshes.
  The \textbf{product mesh} $\mathcal{K} \times \mathcal{L}$ is defined as the
  poset product of $\mathcal{K}$ and $\mathcal{L}$.
\end{definition}
\begin{proposition}
  Let $\mathcal{K}$ and $\mathcal{L}$ be meshes.
  Then $\mathcal{K} \times \mathcal{L}$ is also a mesh.
\end{proposition}
\begin{proposition}
  Let $\mathcal{K}$ and $\mathcal{L}$ be quasi-cubical meshes.
  Then $\mathcal{K} \times \mathcal{L}$ is also a quasi-cubical mesh.
\end{proposition}
\begin{proposition}
  Let $\mathcal{M}$ and $\mathcal{N}$ be combinatorial meshes.
  Then
  \begin{equation}
    {\rm Forman}(\mathcal{M} \times \mathcal{N})
    = {\rm Forman}(\mathcal{M}) \times {\rm Forman}(\mathcal{N}).
  \end{equation}
\end{proposition}
\begin{definition}
  Let $\varphi \colon \mathcal{K} \to \mathcal{P} K$ and
    $\psi \colon \mathcal{L} \to \mathcal{P} L$ be
  mesh embeddings.
  Define the \textbf{product embedding}
  \begin{equation}
    \varphi \times \psi \colon
    \mathcal{K} \times \mathcal{L} \to \mathcal{P} (K \times L)
  \end{equation}
  as follows: for any $(a, \alpha) \in \mathcal{K} \times \mathcal{L}$,
  \begin{equation}
    (\varphi \times \psi)(a, \alpha) = (\varphi a) \times (\psi \alpha).
  \end{equation}
\end{definition}
\begin{proposition}
  Let
    $(\mathcal{K}, \varepsilon^{\mathcal{K}})$ and
      $(\mathcal{K}, \varepsilon^{\mathcal{K}})$
      be meshes with relative orientations,
    $C_\bullet(\mathcal{K}, \partial^{\mathcal{K}})$ and
      $C_\bullet(\mathcal{L}, \partial^{\mathcal{L}})$
    be the corresponding chain complexes (of boundary operators).
  Then the chain complex
  \begin{equation}
    C_\bullet(\mathcal{K}, \partial^{\mathcal{K}}) \otimes
    C_\bullet(\mathcal{L}, \partial^{\mathcal{L}})
  \end{equation}
  induces a relative orientations on $\mathcal{K} \times \mathcal{L}$.
  Precisely, if $0, \leq p \leq \dim \mathcal{K}$,
  $0, \leq q \leq \dim \mathcal{L}$,
  $a \in \mathcal{K}_p, \alpha \in \mathcal{L}_{q}$,
  $(b, \beta) \in \partial_{p + q}(a, \alpha)$,
  \begin{equation}
    \varepsilon^{\mathcal{K} \times \mathcal{L}}_{p + q}(
      (a, \alpha), (b, \beta)) =
    \begin{cases}
      \varepsilon_p^{\mathcal{K}}(a, b),\ & \alpha = \beta \\
      (-1)^p \varepsilon_q^{\mathcal{L}}(\alpha, \beta),\ & a = b
    \end{cases}.
  \end{equation}
\end{proposition}
\begin{definition}
  Let $\mathcal{K}, \mu^{\mathcal{K}}$ and $\mathcal{L}, \mu^{\mathcal{L}}$ be
  Riemannian meshes.
  Define the \textbf{product measures}
  $\mu^{\mathcal{K}} \times \mu^{\mathcal{L}}$
  on $\mathcal{K} \times \mathcal{L}$ as follows:
  for any $a \in \mathcal{K},\ \alpha \in \mathcal{L}$,
  \begin{equation}
    (\mu^{\mathcal{K}} \times \mu^{\mathcal{L}})(a, \alpha)
    := \mu^{\mathcal{K}}(a)\, \mu^{\mathcal{L}}(\alpha).
  \end{equation}
\end{definition}
\begin{proposition}
  Let
    $\mathcal{K}$ and $\mathcal{L}$ be combinatorial meshes,
    $(K, g^K)$ and $(L, g^L)$ be Riemannian manifolds,
    $\varphi^{\mathcal{K}} \colon \mathcal{K} \to K$ and
      $\varphi^{\mathcal{L}} \colon \mathcal{L} \to L$
      be mesh embeddings,
    $\mu^{\mathcal{K}}$ and $\mu^{\mathcal{L}}$ be the induced measures.
  Then $\mu^{\mathcal{K}} \times \mu^{\mathcal{L}}$ is the measure induced by
  the embedding $\varphi^{\mathcal{K}} \times \varphi^{\mathcal{L}}$ of the mesh
  $\mathcal{K} \times \mathcal{L}$
  in the Riemannian manifold $(K \times L, g^K \times g^L)$.
\end{proposition}
\begin{proposition}
  Let
    $(\mathcal{K}, \mu^{\mathcal{K}})$ and $(\mathcal{L}, \mu^{\mathcal{L}})$
      be Riemannian meshes,
    $\inner{\cdot}{\cdot}_{\mathcal{K}}$ and
      $\inner{\cdot}{\cdot}_{\mathcal{L}}$
      be the respective induced inner products,
    $\inner{\cdot}{\cdot}_{\mathcal{K} \times \mathcal{L}}$
      be the inner product induced by
      $\mu^{\mathcal{L}} \times \mu^{\mathcal{L}}$,
    $(a, \alpha) \in \mathcal{K} \times \mathcal{L}$.
  Then
  \begin{equation}
    \inner{(a, \alpha)^\bullet}{(a, \alpha)^\bullet}
    _{\mathcal{K} \times \mathcal{L}}
    = \inner{a^\bullet}{a^\bullet}_\mathcal{K}\, 
      \inner{\alpha^\bullet}{\alpha^\bullet}_\mathcal{L}.
  \end{equation}
\end{proposition}
\begin{proof}
  Let $D = \dim \mathcal{K}$, $d = \dim \mathcal{L}$.
  Then $\dim (\mathcal{K} \times \mathcal{L}) =  D + d$.
  Hence,
  \begin{equation}
    \begin{split}
      \inner{(a, \alpha)^\bullet}{(a, \alpha)^\bullet}
      _{\mathcal{K} \times \mathcal{L}}
      & = \frac{1}{2^{D + d} \mu^{\mathcal{K} \times \mathcal{L}}(a, \alpha)}
          \sum_{(b, \beta) \perp (a, \alpha)}
            \mu^{\mathcal{K} \times \mathcal{L}}(b, \beta) \\
      & = \frac{1}{2^D \mu^{\mathcal{K}}(a)}
          \frac{1}{2^d \mu^{\mathcal{L}}(\alpha)}
          \sum_{b \perp a,\ \beta \perp \alpha}
            \mu^{\mathcal{K}}(b)\, \mu^{\mathcal{L}}(\beta) \\
      & = \left(
            \frac{1}{2^D \mu^{\mathcal{K}}(a)}
            \sum_{b \perp a} \mu^{\mathcal{K}}(b)
          \right)\,
          \left(
            \frac{1}{2^d \mu^{\mathcal{L}}(\alpha)}
            \sum_{\beta \perp \alpha} \mu^{\mathcal{L}}(\beta)
          \right) \\
      & = \inner{a^\bullet}{a^\bullet}_{\mathcal{K}}\,
          \inner{\alpha^\bullet}{\alpha^\bullet}_{\mathcal{L}}.
    \end{split}
  \end{equation}
\end{proof}
\begin{proposition}
  Let
    $\mathcal{K}$ and $\mathcal{L}$ be combinatorial meshes,
    $(K, g^K)$ and $(L, g^L)$ be smooth Riemannian manifolds that realise them,
    $\inner{\cdot}{\cdot}_K$ and $\inner{\cdot}{\cdot}_L$ be the induced
      inner products of differential forms,
    $W^\mathcal{K} \colon C^\bullet \mathcal{K} \to H \Omega^\bullet K$ and
      $W^\mathcal{L} \colon C^\bullet \mathcal{L} \to H \Omega^\bullet L$
      be Whitney maps,
    $\inner{\cdot}{\cdot}_{\mathcal{K}}$, $\inner{\cdot}{\cdot}_{\mathcal{L}}$,
      and $\inner{\cdot}{\cdot}_{\mathcal{K} \times \mathcal{L}}$
      be the induced Whitney inner products
      (the last one is induced by $W^\mathcal{K} \otimes W^\mathcal{L}$),
    $\sigma, \tau \in C^\bullet \mathcal{K}$,
    $\omega, \eta \in C^\bullet \mathcal{L}$.
  Then
  \begin{equation}
    \inner{\sigma \otimes \omega}{\tau \otimes \eta}
    _{\mathcal{K} \times \mathcal{L}}
    = \inner{\sigma}{\tau}_{\mathcal{K}}\, \inner{\omega}{\eta}_{\mathcal{L}}.
  \end{equation}
\end{proposition}
\begin{proof}
  A direct computation:
  \begin{equation}
    \begin{split}
      \inner{\sigma \otimes \omega}{\tau \otimes \eta}
      _{\mathcal{K} \times \mathcal{L}}
      & = \inner{W^{\mathcal{K} \times \mathcal{L}}(\sigma \otimes \omega)}
          {W^{\mathcal{K} \times \mathcal{L}}(\tau \otimes \eta)}
          _{K \times L} \\
      & = \inner{W^{\mathcal{K}} \sigma \boxtimes W^{\mathcal{L}} \omega}
          {W^{\mathcal{K}} \tau \boxtimes W^{\mathcal{L}} \eta}
          _{K \times L} \\
      & = \int_{K \times L}
          g^{K \times L}(
            W^{\mathcal{K}} \sigma \boxtimes W^{\mathcal{L}} \omega,
            W^{\mathcal{K}} \tau \boxtimes W^{\mathcal{L}} \eta)\,
          \vol_{K \times L} \\
      & = \left(
            \int_K g^K(W^{\mathcal{K}} \sigma, W^{\mathcal{K}} \tau) \vol_K
          \right)\,
          \left(
            \int_L g^L(W^{\mathcal{L}} \omega, W^{\mathcal{L}} \eta) \vol_L
          \right) \\
      & = \inner{W^{\mathcal{K}} \sigma}{W^{\mathcal{K}} \tau}_K\,
          \inner{W^{\mathcal{L}} \omega}{W^{\mathcal{L}} \eta}_L \\
      & = \inner{\sigma}{\tau}_{\mathcal{K}}\,
          \inner{\omega}{\eta}_{\mathcal{L}}.
    \end{split}
  \end{equation}
\end{proof}
