\begin{definition}
  Let
    $R$ be a commutative ring with unity,
    $V$ be a module over $R$.
  The \textbf{exterior algebra} of $V$, $\Lambda V$ is the smallest associative
  algebra with unity containing $V$ as a subspace and satisfying the
  \textit{alternating rule}:
  for every $v \in V$, if $\wedge$ is the multiplication on $\Lambda V$, then
  \begin{equation}
    v \wedge v = 0.
  \end{equation}
\end{definition}
\begin{corollary}
  Let
    $R$ be a commutative ring with unity,
    $V$ be a module over $R$,
    $v, w \in V$.
  Then on $\Lambda V$,
  \begin{equation}
    w \wedge v = - v \wedge w.
  \end{equation}
\end{corollary}
\begin{proof}
  By the alternating rule,
  \begin{equation}
    0 = (v + w) \wedge (v + w)
    = v \wedge v + v \wedge w + w \wedge v + w \wedge w
    = 0 + v \wedge w + w \wedge v + 0
    \Rightarrow w \wedge v = - v \wedge w.
  \end{equation}
\end{proof}
\begin{proposition}
  Let
    $R$ be a commutative ring with unity,
    $d \in \N$
    $V$ be a free module over $R$ of dimension $d$,
    $e_0, ..., e_{d - 1}$ be a basis of $V$.
  Then the $2^d$-element set $S(e)$,
  \begin{equation}
    S(e) :=
    \set{e_{i_0} \wedge ... \wedge e_{i_{p - 1}}}
    {p \in \{0, ..., d\},\ 0 \leq i_0 < ... < i_{p - 1} \leq d - 1}
  \end{equation}
  (for $p = 0$ the empty wedge product is defined to be $1$)
  forms a basis of $\Lambda V$.
\end{proposition}
\begin{remark}
   Let
    $R$ be a commutative ring with unity,
    $d \in \N$
    $V$ be a free module over $R$ of dimension $d$,
    $e_0, ..., e_{d - 1}$ be a basis of $V$,
  The elements of $S(e)$ which are wedge products of $p$ vectors are called
  \textbf{$p$-vectors}.
  Obviously, there are $\binom{n}{p}$ of them.
  Denote by $\Lambda^p V$ the space spanned by those $p$ vectors
  (it does not depend on the basis of $e$ -- in fact $\Lambda^p V$ is the space
  spanned by linear combinations of wedge products of arbitrary vectors).
  Then we have the module decomposition
  \begin{equation}
    \Lambda V := \sum_{p = 0}^d \Lambda^p V.
  \end{equation}
  This decompositions turns $\Lambda V$ into a graded algebra, that is for each
  $p, q \in \N$,
  \begin{equation}
    \omega_p \in \Lambda^p V,\ \eta_q \in \Lambda^q V
    \Rightarrow \omega_p \wedge \eta_q \in \Lambda^{p + q} V
  \end{equation}
  (here we define $\Lambda^r = 0$ for $r > d$).
\end{remark}
\begin{proposition}
  Let
    $R$ be a commutative ring with unity,
    $d \in \N$
    $V$ be a free module over $R$ of dimension $d$,
    $v_0, ..., v_{d - 1} \in V$.
  Then $(v_0, ..., v_{d - 1})$ form a basis of $V$
  if and only if
  \begin{equation}
    v_0 \wedge ... \wedge v_{d - 1}
  \end{equation}
  forms a basis of the $1$-dimensional module $\Lambda^d V$.
\end{proposition}
\begin{proposition}
  Let
    $R$ be a commutative ring with unity,
    $d \in \N$
    $V$ be a free module over $R$ of dimension $d$,
    $e_0, ..., e_{d - 1}$ be a basis of $V$,
    $v_0, ..., v_{d - 1} \in V$ such that for $j = 0, ..., d - 1$,
  \begin{equation}
    v_j = \sum_{i = 0}^{d - 1} e_i A_{i, j}
  \end{equation}
  (in matrix form, $(v_0, ..., v_{d - 1}) = (e_0, ..., e_{d - 1}) A$).
  Then
  \begin{equation}
    v_0 \wedge ... \wedge v_{d - 1} = (\det A) e_0 \wedge ... \wedge e_{d - 1}.
  \end{equation}
  As a consequence, the matrix $A$ is a change of basis matrix if and only if
  $\det A$ is an invertible element of $R$
  (a nonzero element when $R$ is a field).
\end{proposition}
