\begin{discussion}
  We are going to state the governing laws for the discrete heat transport
  phenomenon.

  Let:
  \begin{itemize}
    \item $M$ be a manifold-like flat mesh of dimension $3$;
    \item $K$ be the Forman subdivision of $M$;
    \item $t_0 \in \R$ be the initial time, $I = [t_0, \infty)$.
  \end{itemize}
  Physical quantities in our model are:
  \begin{itemize}
    \item
      temperature $u \colon I \to C^0 K$ of physical dimension $[\Theta]$;
    \item
      heat energy density $Q \colon I \to C^0 K$ of physical dimension
      $[E L^{-3}]$;
    \item
      heat flux $q \colon I \to C^1 K$ of physical dimension
      $[E L^{-1} T^{-1}]$.
  \end{itemize}
  The governing laws are the following.
  \begin{itemize}
    \item
      Let
        $K' := K \setminus \partial K$ be the interior of $K$,
        $f \in C^0 K'$ be the external heat source of physical dimension
          $[E L^{-3} T^{-1}]$.
      \textbf{Conservation of heat energy} is modeled by the equation
      \begin{equation}
        \restrict{\frac{\partial Q}{\partial t}}{K'_0} =
        \restrict{(\delta_1^\star q)}{K'_0} + f.
      \end{equation}
    \item
      Let
        $u_0 \in C^0 K$ be the initial temperature, $u_0\ [\Theta]$.
      The \textbf{initial condition} is prescribed initial temperature:
      \begin{equation}
        u(t_0, \cdot) = u_0.
      \end{equation}
    \item
      Let $\pi_0 \colon C^0 K \to C^0 K$
      of physical dimension $[E L^{-3} \Theta^{-1}]$
      be a material property of the nodes of $K$,
      playing the role of volumetric heat capacity
      (its matrix in the standard basis is diagonal).
      The \textbf{relation between temperature and heat energy} is given by
      \begin{equation}
        Q = \pi_0 u.
      \end{equation}
    \item
      Let $\kappa_1 \colon C^1 K \to C^1 K$
      of physical dimension $[E L^{-1} T^{-1} \Theta^{-1}]$
      be a material property of the edges of $K$,
      playing the role of thermal conductivity
      (its matrix in the standard basis is diagonal).
      The \textbf{Fourier's constitutive relation} is given by
      \begin{equation}
        q = \kappa_1 (\delta_0 u).
      \end{equation}
  \end{itemize}
  We complete our model with boundary conditions.
  Let $\Gamma_D, \Gamma_N$ form a partition of $(\partial K)_0$
  into Dirichlet and Neumann boundary.
  \begin{itemize}
    \item
      Let $g_D \colon [t_0, \infty) \to ({\rm Free}_\R(\Gamma_D))^*$
      of physical dimension $[\Theta]$
      be the prescribed temperature on the Dirichlet boundary $\Gamma_D$.
      The \textbf{Dirichlet boundary condition} is given by
      \begin{equation}
        \restrict{u}{\Gamma_D} = g_D.
      \end{equation}
    \item
      Let
        ${\bf n} \colon \Gamma_N \to \R^d$
          be the dimensionless generalized exterior unit normal,
        $g_N \colon [t_0, \infty) \to ({\rm Free}_\R(\Gamma_N))^*$
          of physical dimension $[E L^{-1} T^{-1}]$
          be the prescribed flux on the Neumann boundary $\Gamma_N$.
      The \textbf{Neumann boundary condition} is given by
      \begin{equation}
        \restrict{\overline{q}}{\Gamma_N} \cdot {\bf n} = g_N.
      \end{equation}
  \end{itemize}
\end{discussion}
