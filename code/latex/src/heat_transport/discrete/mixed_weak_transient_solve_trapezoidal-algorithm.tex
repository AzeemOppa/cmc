\begin{algorithm}[Algorithm for solving the transient mixed weak formulation
  for the discrete heat transfer phenomenon using trapezoidal rule for time
  integration, assuming time-independent input data]
  \label{idec/heat_transport/discrete/mixed_weak_transient_solve_trapezoidal-algorithm}
  Let:
  \begin{itemize}
    \item
      Let $d$ be a positive integer (space dimension);
    \item
      $K$ be an oriented quasi-cubical \hyperref[idec:mesh:definition]{mesh} of
      dimension $d$ representing the material body;
    \item
      $[K]$ be the fundamental class of $K$;
    \item
      $t_0 [T] \in \R$ be the initial time;
    \item
      $\tau [T] \in \R^+$ be the time step;
    \item
      $f^d [E T^{-1}] \in C^d K$ be the heat source;
    \item
      $u_0 [\Theta] \in C^0 K$ be the initial temperature;
    \item
      $\tilde{\pi}_d [E L^{-d} \Theta^{-1}] \colon I \times C^d K \to C^d K$
      be the heat capacity of the material;
    \item
      $\kappa_{d - 1} [E L^{2 - d} T^{-1} \Theta^{-1}]
        \colon C^{d - 1} K \to C^{d - 1} K$
      be the thermal conductivity of the material;
    \item
      $\partial K = \Gamma_D \cup \Gamma_N$ be the partition of the boundary of
      $K$ into Dirichlet ($\Gamma_D$) and Neumann ($\Gamma_N$) regions;
    \item
      $[\Gamma_D]$ be the fundamental class of $\Gamma_D$, where $\Gamma_D$
      has the boundary orientation induced from $K$;
    \item
      $g_D^0 [\Theta] \in C^0 \Gamma_D$
      be the prescribed temperature on the Dirichlet boundary;
    \item
      $g_N^{d - 1} [E T^{-1}] \in C^{d - 1} \Gamma_N$
      be the prescribed flow on the Neumann boundary.
  \end{itemize}
  Our algorithm has $3$ phases.
  \begin{enumerate}
    \item
      \textbf{Time-independent phase.}
      Calculate:
      \begin{itemize}
        \item
          $n_p := \abs{K_p}$, $p = d - 1$ and $p = d$;
        \item
          the diagonal matrix ${\bf A} \in M_{n_{d - 1} \times n_{d - 1}}(\R)$,
          \begin{equation}
            {\bf A}_{i, j}
            := \inner{c^{d - 1, j}}
              {\kappa_{d - 1}^{-1} c^{d - 1, i}}_{K, d - 1},\
            i, j = 0, ..., n_{d - 1} - 1;
          \end{equation}
        \item
          the sparse matrix ${\bf B} \in M_{n_d \times n_{d - 1}}(\R)$,
          \begin{equation}
            {\bf B}_{k, i} := \inner{\delta_{d - 1} c^{d - 1, i}}{c^{d, k}}_{K, d},\
            i = 0, ..., n_{d - 1} - 1, k = 0, ..., n_d - 1;
          \end{equation}
        \item
          the diagonal matrix ${\bf C} \in M_{n_d \times n_d}(\R)$,
          \begin{equation}
            {\bf C}_{k, l} := \inner{\tilde{\pi}_d c^{d, l}}{c^{d, k}}_{K, d},\
            k, l = 0, ..., n_d - 1;
          \end{equation}
        \item
          the (sparse) vector ${\bf G} \in \R^{n_{d - 1}}$,
          \begin{equation}
            {\bf G}_i
            :=(\tr_{\Gamma_D, d - 1} c^{d - 1, i} \smile g_D^0)[\Gamma_D],\
            i = 0, ..., n_{d - 1} - 1;
          \end{equation}
        \item
          the vector ${\bf F} \in \R^{n_d}$,
          \begin{equation}
            {\bf F}_k := \inner{f^d}{c^{d, k}}_{K, d},\ k = 0, ..., n_d - 1;
          \end{equation}
        \item
          the sets
          \begin{subequations}
            \begin{alignat}{2}
              & J
              && := \set{i \in \{0, ..., n_{d - 1}\}}
                {c_{d - 1, i} \in (\Gamma_N)_{d - 1}}, \\
              %
              & \overline{J}
              && := \{0, ..., n_{d - 1}\} \setminus J;
            \end{alignat}
          \end{subequations}
        \item
          the restricted diagonal matrix
          $\overline{\bf A} \in
            M_{\abs{\overline{J}} \times \abs{\overline{J}}}(\R)$
          constructed out of ${\bf A}$ with rows and columns in $J$ removed;
        \item
          the restricted sparse matrix
          $\overline{\bf B} \in M_{n_d \times \abs{\overline{J}}}(\R)$
          constructed out of ${\bf B}$ with columns in $J$ removed;
        \item
          the modified and restricted vector
          $\overline{\bf G} \in \R^{\abs{\overline{J}}}$,
          \begin{equation}
            \overline{\bf G}_i :=
            {\bf G}_i
            - \sum_{j \in J} {\bf A}_{i, j} g_N^{d - 1}(c_{d - 1, j}),\
            i \in \overline{J}
          \end{equation}
          (in our case  $A$ is diagonal and so for all $i \in \overline{J}$
          we get $\overline{\bf G}_i = {\bf G}_i$, i.e., no modification);
        \item
          the modified vector $\widetilde{\bf F} \in \R^{n_d}$,
          \begin{equation}
            \widetilde{\bf F}_k
            := 2 {\bf F}_k
              + \sum_{i \in J} {\bf B}_{k, i} g_N^{d - 1}(c_{d - 1, i}),\
            k = 0, ..., n_d - 1;
          \end{equation}
        \item
          the left-hand side matrix ${\bf N}_\tau \in M_{n_d \times n_d}(\R)$,
          \begin{equation}
            {\bf N}_\tau
            := \overline{\bf B}\, \overline{\bf A}^{-1} \overline{\bf B}^T
              + \frac{2}{\tau} {\bf C};
          \end{equation}
        \item
          the lower-triangular positive-diagonal sparse matrix
          ${\bf L}_\tau \in M_{n_d \times n_d}(\R)$
          realising the Cholesky-decomposition
          \begin{equation}
            {\bf N}_\tau = {\bf L}_\tau {\bf L}^T_\tau;
          \end{equation}
        \item
          the constant right-hand side vector ${\bf Z} \in \R^{n_d}$,
          \begin{equation}
            {\bf Z}
            := \overline{\bf B}\, \overline{\bf A}^{-1} \overline{\bf G}
              + \widetilde{\bf F};
          \end{equation}
        \item
          the time-independent part of the heat flow
          $\overline{\bf P} \in \R^{\abs{\overline{J}}}$,
          \begin{equation}
            \overline{\bf P} := \overline{\bf A}^{-1} \overline{\bf G};
          \end{equation}
        \item
          the time-independent part of the dual temperature
          ${\bf V}_\tau \in \R^{n_d}$,
          \begin{equation}
            {\bf V}_\tau
            := {\bf N}_\tau^{-1} {\bf Z}
            = {\bf L}_\tau^{-T} {\bf L}_\tau^{-1} {\bf Z}
          \end{equation}
          (calculated with forward and back substitution);
        \item
          the initial coordinates
          ${\bf U}^0 \in \R^{n_d}$ of the dual temperature,
          and ${\bf Q}^0 \in \R^{n_{d - 1}}$ of the heat flow:
          \begin{subequations}
            \begin{alignat}{2}
              & {\bf Q}^0_i := ({\bf {\rm flow}}({\bf u}_0))_i,\
              && i = 0, ..., n_{d - 1} - 1, \\
              %
              & {\bf U}^0_k := ({\bf \star}_0 {\bf u}_0)_k,\
              && k = 0, ..., n_d - 1.
            \end{alignat}
          \end{subequations}
      \end{itemize}
      Allocate memory for the following vectors, to be modified at each step in
      the looping phase (superscript index $s$ only shows their time relevance):
      \begin{itemize}
        \item
          time-dependent part of the right-hand side
          ${\bf Y}_\tau^s \in \R^{n_d}$;
        \item
          time-dependent part of the solution
          ${\bf W}_\tau^s \in \R^{n_d}$;
        \item
          the restricted coordinates
          $\overline{\bf Q}^{s + 1} \in \R^{\abs{\overline{J}}}$.
      \end{itemize}
    \item
      \textbf{Time-dependent (loop) phase.}
      The constant input consists of
      $g_N^{d - 1},\ {\bf B},\ \overline{\bf B},\ {\bf C},\ \tau,\
      {\bf L}_\tau,\ {\bf V}_\tau,\ \overline{\bf P}$.
      Temporary mutable variables include the vectors
      ${\bf Y}_\tau^s,\ {\bf W}_\tau^s,\ \overline{\bf Q}^{s + 1}$.
      The output consists of the coordinates ${\bf Q}$ of the heat flow,
      and the coordinates ${\bf U}$ of dual temperature.
      ${\bf Q}$ and ${\bf U}$ are either pre-allocated as arrays of size
      $(\verb|number_of_time_steps| + 1) \times n_p$
      (for $p = d - 1$ and $p = d$ respectively) and initialized,
      or are only initialized, and memory is allocated at each step until some
      error condition is satisfied
      (e.g., the relative error between two consecutive steps becomes below some
      $\varepsilon > 0$, in which case the system converges to steady-state).

      For any $s \geq 0$ (until some predefined final moment is reached or some
      condition for small error is fulfilled), calculate:
      \begin{itemize}
        \item
          \qquad\;
          $ {\bf Y}_\tau^s
            := {\bf B} {\bf Q}^s + \frac{2}{\tau} {\bf C} {\bf U}^s$;
          \hfill\refstepcounter{equation}\textup{(\theequation)}
        \item
          \qquad\;
          $ {\bf W}_\tau^s
            := {\bf N}_\tau^{-1} {\bf Y}_\tau^s
            = {\bf L}_\tau^{-T} {\bf L}_\tau^{-1} {\bf Y}_\tau^s$
          \hfill\refstepcounter{equation}\textup{(\theequation)}
          \newline
          (calculated with forward and back substitution);
        \item
          the coordinates of the dual temperature ${\bf U}^{s + 1}$
          (stored for all $s$),
          \begin{equation}
            {\bf U}^{s + 1} := {\bf V}_\tau + {\bf W}_\tau^s;
          \end{equation}
        \item
          \qquad\;
          $ \overline{\bf Q}^{s + 1}
            := \overline{\bf P} - \overline{\bf B}^T {\bf U}^{s + 1}$;
          \hfill\refstepcounter{equation}\textup{(\theequation)}
        \item
          the heat flow ${\bf Q}^{s + 1} \in \R^{n_{d - 1}}$
          (stored for all $s$),
          \begin{equation}
            {\bf Q}^{s + 1}_j :=
            \begin{cases}
              \overline{\bf Q}^{s + 1}_j, & j \in \overline{J} \\
              g_N^{d - 1}(c_{d - 1, j}), & j \in J
            \end{cases}.
          \end{equation}
      \end{itemize}
    \item
      \textbf{Post-processing.}
      Define the sets
      \begin{subequations}
        \begin{alignat}{2}
          & D_0
          && := \set{i \in \{0, ..., n_0 - 1\}}{c_{0, i} \in (\Gamma_D)_0}, \\
          %
          & \overline{D_0}
          && := \{0, ..., n_0 - 1\} \setminus D_0.
        \end{alignat}
      \end{subequations}
      For each time moment $t_s$ the coordinates of the temperature
      ${\bf u}^{0, s} \in \R^{n_0}$
      in the standard basis are calculated by the formula
      \begin{equation}
        {\bf u}^{0, s}_i :=
        \begin{cases}
          ({\bf \star}_d {\bf U}^{d, s})_i, & i \in \overline{D_0} \\
          g_D^0(c_{0, i}), & i \in D_0
        \end{cases}.
      \end{equation}
  \end{enumerate}
\end{algorithm}
