\begin{formulation}
  \label{idec/discrete_heat_transport/phenomenon-formulation}
  By substituting $q$ with $\kappa_1 \delta_0 u$ and $Q$ with $\pi_0 u$,
  we arrive at the following formulation with only one unknown
  (the temperature $u$).
  Let:
  \begin{itemize}
    \item
      $d \in \N$;
    \item
      $M$ be a manifold-like flat mesh of dimension $d$;
    \item
      $K$ be the Forman subdivision of $M$;
    \item
      $K' := K \setminus \partial K$ be the interior of $K$;
    \item
      $t_0 \in \R$ be the initial time, $t_0\ [T]$;
    \item
      $u_0 \in C^0 K$ be the initial temperature, $u_0\ [\Theta]$;
    \item
      $f \in C^0 K'$ be the external heat source, $f\ [E L^{-3} T^{-1}]$;
    \item
      $\Gamma_D, \Gamma_N$ form a partition of $(\partial K)_0$;
    \item
      ${\bf n} \colon \Gamma_N \to \R^d$
      be the generalized exterior unit normal, ${\bf n}\ [1]$;
    \item
      $g_D \colon [t_0, \infty) \to ({\rm Free}_\R(\Gamma_D))^*$
      be the Dirichlet boundary condition, $g_D\ [\Theta]$;
    \item
      $g_N \colon [t_0, \infty) \to ({\rm Free}_\R(\Gamma_N))^*$
      be the Neumann boundary condition, $g_N\ [E L^{-1} T^{-1}]$;
    \item
      $\pi_0 \colon C^0 K \to C^0 K$ be a material property of the nodes of $K$
      (its matrix in the standard basis is diagonal),
      $\pi_0\ [E L^{-3} \Theta^{-1}]$;
    \item
      $\kappa_1 \colon C^1 K \to C^1 K$ be a material property of the edges of
      $K$ (its matrix in the standard basis is diagonal),
      $\kappa_1\ [E L^{-1} T^{-1} \Theta^{-1}]$.
  \end{itemize}
  We are solving the following problem.
  \begin{equation}
    \begin{split}
      & \text{Find $u \in [t_0, \infty) \to C^0 K$, $u\ [\Theta]$, such that} \\
      &
      \begin{cases}
        \restrict{\frac{\partial (\pi_0 u)}{\partial t}}{K'_0} =
        \restrict{((\delta_1^\star \circ \kappa_1 \circ \delta_0) u)}{K'_0} + f
        & (\text{conservation of heat energy},\ [E L^{-3} T^{-1}]), \\
%
        u(t_0, \cdot) = u_0
        & (\text{initial condition},\ [\Theta]), \\
%
        \restrict{u}{\Gamma_D} = g_D
        & (\text{Dirichlet boundary condition},\ [\Theta]), \\
%
        \restrict{\overline{(\kappa_1 \circ \delta_0) u}}{\Gamma_N}
        \cdot {\bf n} = g_N
        & (\text{Neumann boundary condition},\ [E L^{-1} T^{-1}]).
      \end{cases}
    \end{split}
  \end{equation}
\end{formulation}
