\begin{discussion}
  Consider \Cref{idec/discrete_heat_transport/phenomenon-formulation}.
  This formulation is discrete in space but continuous in time.
  In order to numerically solve it we need to discretize the time variable.
  We will use the trapezoidal (Crank-Nicolson) method.

  Let
    $\tau \in \R^+$ of physical dimension $[T]$ be the time step,
    $i \in \N$,
    $t_i := t_0 + i \tau$,
    $y_i := u(t_i, \cdot) \in C^0 K$ of physical dimension $[\Theta]$,
    $\rho_0 := \delta_1^\star \circ \pi_1 \circ \delta_0$
      of physical dimension $[E L^{-3} T^{-1} \Theta^{-1}]$.
  Integrating the conservation of heat energy in $[t_i, t_{i + 1}]$, we get
  \begin{equation}
    \restrict{(\pi_0 y_{i + 1} - \pi_0 y_i)}{K'_0}
    =   \int_{t_i}^{t_{i + 1}} \restrict{(\rho_0 u(t, \cdot))}{K'_0}\, d t
      + \int_{t_i}^{t_{i + 1}} f\, d t
    \approx
    \frac{\tau}{2} \restrict{(\rho_0 y_i + \rho_0 y_{i + 1})}{K'_0} + \tau f.
  \end{equation}
  Rearranging, we get the discretized equation
  \begin{equation}
    \restrict{((\pi_0 - \frac{\tau}{2} \rho_0) y_{i + 1})}{K'_0}
    = \restrict{((\pi_0 + \frac{\tau}{2} \rho_0) y_i)}{K'_0} + \tau f.
  \end{equation}
  Define the $\dim C^0 K \times \dim C^0 K$ matrices of physical dimension
  $[E L^{-3} \Theta^{-1}]$
  \begin{equation}
    A := \pi_0 - \frac{\tau}{2} \rho_0,\ B := \pi_0 + \frac{\tau}{2} \rho_0.
  \end{equation}
  The discretized in time (space is already discrete) formulation reads as
  follows.
  \begin{equation}
    \begin{split}
      & \text{Find $y \colon \N \to C^0 K$ such that} \\
      &
      \begin{cases}
        y_0 = u_0 \\
        \text{for any $i \in \N^+$, $y_i$ solves the system} \\
        \begin{cases}
          \restrict{(A y_i)}{K'_0} = \restrict{(B y_{i - 1})}{K'_0} + \tau f, \\
          \restrict{y_i}{\Gamma_D} = g_D, \\
          \restrict{\overline{(\pi_1 \circ \delta_0) y_i}}{\Gamma_N}
            \cdot {\bf n} = g_N.
        \end{cases}
      \end{cases}
    \end{split}
  \end{equation}
  Of course, in practice we solve it for a finite number of time steps.
  Usually, we compare tho adjacent solutions $y_i$ and $y_{i + 1}$ and stop when
  the relative error is sufficiently small (we have reached a steady state).
  % Moreover, if $\pi_0$ and $\pi_1$ are constants, and $K$ has mesh size $h$,
  % then $h^2 / \tau$ should be close to the thermal diffusivity $\pi_1 / \pi_0$
  % (both of physical dimension $[L^2 T^{-1}]$).
\end{discussion}
