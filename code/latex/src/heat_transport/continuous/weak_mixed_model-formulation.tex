\begin{discussion}
  We are going to formulate the \textbf{mixed weak formulation for continuous
  heat transport with differential forms}.
  Consider the model
  \Cref{idec/heat_transport/continuous/model_with_differential_forms-discussion}
  with the same domains and variable names.
  Let $v \in \Omega^0 X$ and $r \in \Ker \tr_{\Gamma_N, 2}$ be test functions.
  Define
  \begin{equation}
    \pi_2 :=
    \star_2^{-1} \circ \pi_1 \circ \star_2 \colon \Omega^2 X \to \Omega^2 X.
  \end{equation}
  Then
  \begin{equation}
    \pi_2^{-1} q = \star_2^{-1} (d_0 u),
  \end{equation}
  and therefore
  \begin{equation}
    \begin{split}
      \inner{r}{\pi_2^{-1} q}
      & = \int_X (r \wedge \star_2 (\pi_2^{-1} q)) \\
      & = \int_X (r \wedge d_0 u) \\
      & = \int_{\partial X} (\tr_{X, 2} r \wedge \tr_{X, 0} u)
        - \int_{X} (d_2 r \wedge u) \\
      & = \int_{\Gamma_D} (\tr_{\Gamma_D, 2} r \wedge g_D)
        - \int_{X} (d_2 r \wedge u).
    \end{split}
  \end{equation}
  Multiplying the balance equation with $v$ and integrating over $X$ gives
  \begin{equation}
    \frac{d}{d t} \int_X (v \wedge \star_0 (\pi_0 u))
    = \int_X v \wedge d_2 q + \int_X (v \wedge f).
  \end{equation}
  We get the following formulation.
\end{discussion}
% \begin{formulation}
%   [Mixed weak formulation for the continuous transient heat equation]
%   Let:
%   \begin{itemize}
%     \item
%       $X$ be an open region in $3$D, $X$ represents a material body;
%     \item
%       $t_0 \in \R$ be the initial time;
%     \item
%       $I = [t_0, \infty)$ be the time-interval where the process occurs;
%     \item
%       $f [E T^{-1}] \colon I \to \Omega^3 X$ be the heat source;
%     \item
%       $u_0 [\Theta] \in \Omega^0 X$ be the initial temperature;
%     \item
%       $\pi_2 [E L^{-1} T^{-1} \Theta^{-1}] \colon \Omega^2 X \to \Omega^2 X$
%       be the thermal conductivity of the material;
%     \item
%       $\pi_3 [E \Theta^{-1}] \colon \Omega^3 X \to \Omega^3 X$
%       be the volumetric heat capacity of the material;
%     \item
%       $\partial K = \Gamma_D \cup \Gamma_N$ be the partition of the boundary of
%       $K$ into Dirichlet ($\Gamma_D$) and Neumann ($\Gamma_N$) regions;
%     \item
%       $g_D [\Theta] \colon I \to \Omega^0 \Gamma_D$
%       be the prescribed temperature on the Dirichlet boundary;
%     \item
%       $g_N [E T^{-1}] \colon I \to \Omega^2 \Gamma_N$
%       be the prescribed flow on the Neumann boundary.
%   \end{itemize}
%   Our unknowns are:
%   \begin{itemize}
%     \item $q [E T^{-1}] \colon I \to \Omega^2 X$ (heat flow);
%     \item $u [\Theta L^3] \colon I \to \Omega^3 X$ (temperature-volume).
%   \end{itemize}
%   We are solving the following problem for $q$ and $u$:
%   \begin{subequations}
%     \begin{alignat}{4}
%       & \forall r [E T^{-1}] \in \Ker \tr_{\Gamma_N, 2}, \quad
%       && \inner{r}{\pi_2^{-1} q}_{X, 2} + \inner{u}{d r}_{X, 3}
%       && = \int_X(\tr_{\Gamma_D, 2} r \wedge g_D) \qquad
%       && [E T^{-1} \Theta], \\
% %
%       & \forall v [E L^3] \in \Omega^3 K, \quad
%       && \inner{v}{d_2 q}_{X, 3}
%         - \inner{v}{\pi_3 \frac{\partial u}{\partial t}}_{X, 3}
%       && = \inner{v}{f}_{X, 3} \qquad
%       && [E T^{-1} \Theta], \\
% %
%       &
%       && \tr_{\Gamma_N, 2} q
%       && = g_N \qquad
%       && [E T^{-1}], \\
% %
%       &
%       && u(t_0)
%       && = \star_{X, 0} u_0 \qquad
%       && [\Theta L^{-3}].
%     \end{alignat}
%   \end{subequations}
% \end{formulation}
% \begin{formulation}
%   [Mixed weak formulation for the continuous stead-state heat equation]
%   Let:
%   \begin{itemize}
%     \item
%       $X$ be an open region in $3$D, $X$ represents a material body;
%     \item
%       $f [E T^{-1}] \colon I \to \Omega^3 X$ be the heat source;
%     \item
%       $\pi_2 [E L^{-1} T^{-1} \Theta^{-1}] \colon \Omega^2 X \to \Omega^2 X$
%       be the thermal conductivity of the material;
%     \item
%       $\partial K = \Gamma_D \cup \Gamma_N$ be the partition of the boundary of
%       $K$ into Dirichlet ($\Gamma_D$) and Neumann ($\Gamma_N$) regions;
%     \item
%       $g_D [\Theta] \in \Omega^0 \Gamma_D$
%       be the prescribed temperature on the Dirichlet boundary;
%     \item
%       $g_N [E T^{-1}] \in \Omega^2 \Gamma_N$
%       be the prescribed flow on the Neumann boundary.
%   \end{itemize}
%   Our unknowns are:
%   \begin{itemize}
%     \item $q [E T^{-1}] \in \Omega^2 X$ (heat flow);
%     \item $u [\Theta L^3] \in \Omega^3 X$ (temperature-volume).
%   \end{itemize}
%   We are solving the following problem for $q$ and $u$:
%   \begin{subequations}
%     \begin{alignat}{4}
%       & \forall r [E T^{-1}] \in \Ker \tr_{\Gamma_N, 2}, \quad
%       && \inner{r}{\pi_2^{-1} q}_{X, 2} + \inner{u}{d r}_{X, 3}
%       && = \int_X(\tr_{\Gamma_D, 2} r \wedge g_D) \qquad
%       && [E T^{-1} \Theta], \\
% %
%       & \forall v [E L^3] \in \Omega^3 K, \quad
%       && \inner{v}{d_2 q}_{X, 3}
%       && = \inner{v}{f}_{X, 3} \qquad
%       && [E T^{-1} \Theta], \\
% %
%       &
%       && \tr_{\Gamma_N, 2} q
%       && = g_N \qquad
%       && [E T^{-1}].
%     \end{alignat}
%   \end{subequations}
% \end{formulation}
