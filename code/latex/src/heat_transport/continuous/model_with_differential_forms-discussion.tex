\begin{discussion}
  \label{idec/heat_transport/continuous/model_with_differential_forms-discussion}
  Let:
  \begin{itemize}
    \item
      $X$ be an open region in $3$D (the space region);
    \item
      $t_0 [T] \in \R$ be the initial time;
    \item
      $I = [t_0, \infty)$ be the time interval this phenomenon happens.
  \end{itemize}
  The main physical quantities in our model are:
  \begin{itemize}
    \item
      temperature $u [\Theta] \colon I \to \Omega^0 X$:
      for any moment $t \in I$ and any point $x \in X$,
      \begin{equation}
        \text{``temperature $[\Theta]$ on $x$ at time $t$''}
        = u(t)(x) := u(t, x);
      \end{equation}
    \item
      heat energy $Q [E]\colon I \to \Omega^3 X$:
      for any moment $t \in I$ and any volume $V \subseteq X$,
      \begin{equation}
        \text{``total heat energy $[E]$ of the system on $V$ at time $t$''}
        = \int_V Q(t);
      \end{equation}
    \item
      heat flow $q [E T^{-1}] \colon I \to \Omega^2 X$:
      for any time interval $[t_1, t_2] \subset I$
      and any ($2$D) surface $S \subset X$,
      \begin{equation}
        \text{``total inflow $[E]$ through $S$ between $t_1$ and $t_2$''}
        = \int_{t_1}^{t_2} \left(\int_S q(\cdot) \right)\, d t.
      \end{equation}
  \end{itemize}
  The governing laws are formulated as follows.
  \begin{itemize}
    \item
      Let $f [E T^{-1}] \colon I \to \Omega^3 X$ be an external heat source:
      for any time interval $[t_1, t_2] \subset I$
      and any volume $V \subseteq X$,
      \begin{equation}
        \text{``total net heat production $[E]$
          in $V$ between $t_1$ and $t_2$''}
        = \int_{t_1}^{t_2} \left(\int_V f(\cdot) \right)\, d t.
      \end{equation}
      \textbf{Conservation of heat energy} is given by the following relation:
      for any time interval $[t_1, t_2] \subset I$
      and any volume $V \subseteq X$,
      \begin{equation}
        \begin{split}
        \text{``heat difference on $V$ between moments $t_2$ and $t_1$''}
        & =
          \text{``heat inflow through the boundary of $V$ in $[t_1, t_2]$''} \\
        & +
          \text{``heat production inside $V$ in $[t_1, t_2]$''}.
        \end{split}
      \end{equation}
      In symbolic terms, the last equation is written as
      \begin{equation}
        \int_V (Q(t_2) - Q(t_1)) =
          \int_{t_1}^{t_2} \left(\int_{\partial V} q(\cdot) \right)\, d t
        + \int_{t_1}^{t_2} \left(\int_V f(\cdot) \right)\, d t.
      \end{equation}
      Using Stokes' theorem twice, we get the equation
      \begin{equation}
        \int_{t_1}^{t_2}
          \left(\int_V \frac{\partial Q}{\partial t}\right)\, d t =
          \int_{t_1}^{t_2} \left(\int_V d_X q \right)\, d t
        + \int_{t_1}^{t_2} \left(\int_V f \right)\, d t.
      \end{equation}
      Since the time interval $[t_1, t_2]$ and the volume $V$ are arbitrary,
      we can drop integrals and arrive at the differential form
      \begin{equation}
        \frac{\partial Q}{\partial t} = d_X q + f.
      \end{equation}
    \item
      Let
        $u_0 [\Theta] \in \Omega^0 X$ be the initial temperature.
      The \textbf{initial condition} is the prescribed initial temperature:
      \begin{equation}
        u(t_0) = u_0.
      \end{equation}
    \item
      Let $\pi_0 [E L^{-3} \Theta^{-1}] \colon \Omega^0 X \to \Omega^0 X$
      be the role of volumetric heat capacity,
      $\tilde{Q}^0 := \star_3 Q$.
      The \textbf{relation between temperature and heat energy} is given by
      \begin{equation}
        \frac{\partial Q}{\partial t}
        = \star_0 \left(\pi_0 \frac{\partial u}{\partial t}\right).
      \end{equation}
    \item
      Consider two adjacent volumes $V_0$ and $V_1$ with a common surface $S$,
      such that $\varepsilon(V_0, S) = -1,\ \varepsilon(V_1, S) = 1$.
      According to the second law of thermodynamics, heat flows from regions of
      higher temperature to regions of lower temperatures.
      Therefore, the net flow through $S$ is in the negative direction of the
      temperature difference between $V_1$ and $V_0$.
      
      Let $\pi_1 [E L^{-1} T^{-1} \Theta^{-1}] \colon \Omega^1 X \to \Omega^1 X$
      be the thermal conductivity.
      The \textbf{Fourier's constitutive relation}
      quantifies the above relation by a proportionality factor:
      \begin{equation}
        \tilde{q} := \star_2 q = - \pi_1 (d_0 u).
      \end{equation}
  \end{itemize}
  We complete our model with boundary conditions.
  Let $\Gamma_D, \Gamma_N$ form a partition of $\partial X$
  into Dirichlet and Neumann boundary.
  \begin{itemize}
    \item
      Let $g_D [\Theta] \colon I \to \Omega^0 \Gamma_D$
      be the prescribed temperature on the Dirichlet boundary $\Gamma_D$.
      The \textbf{Dirichlet boundary condition} is given by
      \begin{equation}
        \tr_{\Gamma_D, 0} u := \restrict{u}{\Gamma_D} = g_D.
      \end{equation}
    \item
      Let $g_N [E T^{-1}] \colon I \to \Omega^2 \Gamma_N$
      be the prescribed flow on the Neumann boundary $\Gamma_N$.
      The \textbf{Neumann boundary condition} is given by
      \begin{equation}
        \tr_{\Gamma_N, 2} q = g_N.
      \end{equation}
      The prescribed flow is usually calculated using its density form
      \begin{equation}
        \widetilde{g_N} := \star^{(\Gamma_N)}_2 g_N.
      \end{equation}
  \end{itemize}
\end{discussion}
