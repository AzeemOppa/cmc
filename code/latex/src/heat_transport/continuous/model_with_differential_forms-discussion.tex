\begin{discussion}
  \label{idec/heat_transport/continuous/model_with_differential_forms-discussion}
  Let:
  \begin{itemize}
    \item
      $D$ be a positive integer (space dimension);
    \item
      $X$ be an open region in $\R^D$ (the space region);
    \item
      $t_0 [T] \in \R$ be the initial time;
    \item
      $I := [t_0, \infty)$;
  \end{itemize}
  The main physical quantities in our model are:
  \begin{itemize}
    \item
      temperature $u^0 [\Theta] \colon I \to \Omega^0 X$:
      for any moment $t \in I$ and any point $x \in X$,
      \begin{equation}
        \text{``temperature $[\Theta]$ on $x$ at time $t$''}
        = u^0(t)(x) := u^0(t, x);
      \end{equation}
    \item
      heat energy $Q^D [E] \colon I \to \Omega^D X$:
      for any moment $t \in I$ and any volume $V_D \subseteq X$,
      \begin{equation}
        \text{``total heat energy $[E]$ of the system on $V$ at time $t$''}
        = \int_{V_D} Q^D(t);
      \end{equation}
    \item
      heat flow $q^{D - 1} [E T^{-1}] \colon I \to \Omega^{D - 1} X$:
      for any time interval $[t_1, t_2] \subset I$
      and any hypersurface $S_{D - 1} \subset X$,
      \begin{equation}
        \text{``total flow $[E]$ through $S_{D - 1}$ in $[t_1, t_2]$''}
        = \int_{t_1}^{t_2}\left(\int_{S_{D - 1}} q^{D - 1}(t)\right)\, d t.
      \end{equation}
      (Here we assume that $S_{D - 1}$ is oriented.
       Let $U_D$ and $V_D$ be adjacent regions having $S_{D - 1}$ as a common
       boundary, such that $\varepsilon(U_D, S_{D - 1}) = -1$, 
       $\varepsilon(V_D, S_{D - 1}) = 1$.
       Then the above equation measures the total flow from $U_D$ to $V_D$.)
  \end{itemize}
  We will also need the dual variables of heat energy, temperature and flow.
  \begin{itemize}
    \item
      temperature-volume $\tilde{u}^D [\Theta L^D] \colon I \to \Omega^D X$
      defined by
      \begin{equation}
        \tilde{u}^D := \star_0 u^0
      \end{equation}
      (althiugh using non-zero based temperature scale might make $\star_0$ not
      well defined, this will not cause problems as we will always take
      temperature differences when substituting in equations);
    \item
      heat energy density $\tilde{Q}^0 [E L^{-D}] \colon I \to \Omega^0 X$
      defined by
      \begin{equation}
        \tilde{Q}^0 := \star_D Q^D;
      \end{equation}
    \item
      orthogonal (edge) flux
      $\tilde{q}^1 [E T L^{2 - D}] \colon I \to \Omega^1 X$
      defined by
      \begin{equation}
        \tilde{q} := \star_{D - 1} q;
      \end{equation}
  \end{itemize}
  The governing laws are formulated as follows.
  \begin{itemize}
    \item
      Let $f^D [E T^{-1}] \colon I \to \Omega^D X$ be an external heat source:
      for any time interval $[t_1, t_2] \subset I$
      and any volume $V_D \subseteq X$,
      \begin{equation}
        \text{``total net heat production $[E]$ in $V_D$ in $[t_1, t_2]$''}
        = \int_{t_1}^{t_2} \left(\int_{V_D} f^D(t) \right)\, d t.
      \end{equation}
      \textbf{Conservation of heat energy} is given by the following relation:
      for any time interval $[t_1, t_2] \subset I$
      and any volume $V_D \subseteq X$,
      \begin{equation}
        \begin{split}
        \text{``heat difference on $V_D$ between moments $t_2$ and $t_1$''}
        & =
          \text{``heat inflow through the boundary of $V_D$
          in $[t_1, t_2]$''} \\
        & +
          \text{``heat production inside $V_D$ in $[t_1, t_2]$''}.
        \end{split}
      \end{equation}
      In symbolic terms, the last equation is written as
      \begin{equation}
        \int_{V_D} (Q^D(t_2) - Q^d(t_1))
        = \int_{t_1}^{t_2}
          \left(\int_{\partial V_D} q^{D - 1}(t) \right)\, d t
        + \int_{t_1}^{t_2} \left(\int_{V_D} f^D(t) \right)\, d t.
      \end{equation}
      Using Stokes' theorem twice, we get the equation
      \begin{equation}
        \int_{t_1}^{t_2}
          \left(\int_{V_D} \frac{\partial Q^d}{\partial t}\right)\, d t =
          \int_{t_1}^{t_2} \left(\int_{V_D} d_{D - 1} q^{D - 1} \right)\, d t
        + \int_{t_1}^{t_2} \left(\int_{V_D} f^D \right)\, d t.
      \end{equation}
      Since the time interval $[t_1, t_2]$ and the volume $V_D$ are arbitrary,
      we can drop integrals and arrive at the differential form
      \begin{equation}
        \frac{\partial Q^D}{\partial t} = d_{D - 1} q^{D - 1} + f^D.
      \end{equation}
    \item
      Let
        $u_0 [\Theta] \in \Omega^0 X$ be the initial temperature.
      The \textbf{initial condition} is the prescribed initial temperature:
      \begin{equation}
        u^0(t_0) = u_0.
      \end{equation}
    \item
      Let $\pi_0 [E L^{-D} \Theta^{-1}] \colon \Omega^0 X \to \Omega^0 X$
      be the volumetric heat capacity.
      The \textbf{relation between temperature change and heat energy change}
      is given by
      \begin{equation}
        \frac{\partial Q^D}{\partial t}
        = \star_0 \left(\frac{\partial \tilde{Q}^0}{\partial t}\right).
        = \star_0 \left(\pi_0 \frac{\partial u^0}{\partial t}\right).
      \end{equation}
    \item
      Consider two adjacent volumes $U_D$ and $V_D$
      with a common surface $S_{D - 1}$, such that
      $\varepsilon(U_D, S_{D - 1}) = -1$ and
      $\varepsilon(V_D, S_{D - 1}) = 1$.
      According to the second law of thermodynamics, heat flows from regions of
      higher temperature to regions of lower temperatures.
      Therefore, the net flow through $S_{D - 1}$ is in the negative direction
      of the temperature difference between $U_D$ and $V_D$.
      
      Let
      $\kappa_{D - 1} [E L^{2 - D} T^{-1} \Theta^{-1}]
      \colon \Omega^{D - 1} X \to \Omega^{D - 1} X$
      be the thermal conductivity.
      The \textbf{Fourier's constitutive relation}
      quantifies the above relation by using $\kappa_{D - 1}$
      as a proportionality factor:
      \begin{equation}
        q^{D - 1}
        = - \kappa_{D - 1} \delta_D^\star \tilde{u}^D
        = - \kappa_{D - 1} \delta_D^\star \star_0 u^0
        = (-1)^{D - 1} \kappa_{D - 1} \star_1 \delta_0 u^0
        = (-1)^{D - 1} \star_1 \tilde{\kappa}_1 \delta_0 u,
      \end{equation}
      where we have denoted
      \begin{equation}
        \tilde{\kappa}_1
        :=\star_1^{-1} \kappa_2 \star_1 [E L^{2 - D} T^{-1} \Theta^{-1}]
        \colon \Omega^1 X \to \Omega^1 X.
      \end{equation}
  \end{itemize}
  We complete our model with boundary conditions.
  Let $\Gamma_D, \Gamma_N$ form a partition of $\partial X$
  into Dirichlet and Neumann boundary.
  \begin{itemize}
    \item
      Let $g_D^0 [\Theta] \colon I \to \Omega^0 \Gamma_D$
      be the prescribed temperature on the Dirichlet boundary $\Gamma_D$.
      The \textbf{Dirichlet boundary condition} is given by
      \begin{equation}
        \tr_{\Gamma_D, 0} u^0 := \restrict{u}{\Gamma_D} = g_D^0.
      \end{equation}
    \item
      Let $g_N^{D - 1} [E T^{-1}] \colon I \to \Omega^2 \Gamma_N$
      be the prescribed flow on the Neumann boundary $\Gamma_N$.
      The \textbf{Neumann boundary condition} is given by
      \begin{equation}
        \tr_{\Gamma_N, D - 1} q^{D - 1} = g_N^{D - 1}.
      \end{equation}
  \end{itemize}
\end{discussion}
