\begin{example}
  The following are examples and counterexamples of commutative rings with
  unity.
  \begin{enumerate}
    \item
      The number sets $\Z$, $\Q$, $\R$, $\C$ of respectively integers,
      rationals, reals, and complex numbers are commutative rings with unity
      with respect to the standard arithmetic operations.
      The set $\N$ of natural numbers is not a ring since it lacks negation.
    \item
      For any positive integer $n$, the set $\Z_n$ of integers modulo $n$ is a
      commutative ring with unity with respect to addition and multiplication
      modulo $n$.
      (This is the original inspiration behind the word ``ring'' as its elements
      can be arranged in a loop, like the hours on a clock.)
    \item
      Let $X$ be a set, $R$ be a commmutative ring with unity.
      Then the set of functions $R^X := \set{f}{f \colon X \to R}$ is a
      commutative ring with unity ring with respect to pointwise operations:
      if $f, g \colon X \to R$, $x \in X$, we define
      \begin{subequations}
        \begin{alignat}{3}
          & 0_{R^X}(x) && := && 0_R, \\
          & 1_{R^X}(x) && := && 1_R, \\
          & (-_{R^X} f)(x) && := && -_R(f(x)), \\
          & (f +_{R^X} g)(x) && := && f(x) +_R g(x), \\
          & (f *_{R^X} g)(x) && := && f(x) *_R g(x).
        \end{alignat}
      \end{subequations}
      We will drop subscripts and overload ring operations on $R^X$.
  \end{enumerate}
\end{example}
