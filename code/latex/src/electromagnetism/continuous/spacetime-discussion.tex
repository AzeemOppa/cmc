\begin{discussion}
  Let $M$ be a space domain manifold, $I$ be an interval.
  In the split space and time approach quantities in electromagnetism are
  represented via bundle-vlued functions
  $f \in \mathcal{F}(I, \Omega^\bullet M)$.
  We will instead formulate the laws of electromagnetism in terms of
  $\Omega^\bullet (I \times M)$ (or more generally, a $4$-dimensional spacetime
  that is not the Cartesian product of space and time).
  For any $p \in \N$ we will make use of the following isomorphism:
  \begin{equation}
    C^\infty(I, \Omega^p M) \oplus C^\infty(I, \Omega^{p - 1} M)
    \simeq \Omega^p(I \times M)
  \end{equation}
  realised by the following map:
  \begin{equation}
    \omega \mapsto
    \begin{cases}
      f \pi_M^* \eta, & \omega = f \eta \in C^\infty(I, \Omega^p M) \\
      f \pi_M^* \eta \wedge d t,
      & \omega = f \eta \in C^\infty(I, \Omega^{p - 1} M).
    \end{cases}
  \end{equation}
  We form the following pairs and their spacetime versions on 
  \Cref{table:electromagnetism/continuous/spacetime_quantities}.
  Consequently, the laws are given on
  \Cref{table:electromagnetism/continuous/spacetime_laws}.
  Note that the spacetime formulation is manifestly covariant in the category of
  Lorentzian $4$-manifolds (pseudo-Riemannian manifolds with $(1, 3)$ signature
  of the metric).
\end{discussion}
