\documentclass[fleqn]{article}
\usepackage[margin = 0.6in, paper = a4paper]{geometry}
\usepackage{xcolor}
\usepackage[fleqn]{amsmath}
\usepackage{amssymb}
\usepackage{amsthm}
\usepackage{bookmark}
\usepackage{hyperref}
\hypersetup{
  pdfauthor = {Kiprian Berbatov},
  pdftitle = {IDEC Documentation},
  pdfsubject = {Discrete calculus},
  pdfkeywords = {discrete, microstructure, calculus, geometry},
  pdfcreator = {pdflatex},
  pdfproducer = {Latex2e with hyperref},
  colorlinks = true,
  linkcolor = blue,
  citecolor = green,
  urlcolor = cyan,
  bookmarksnumbered = true
}
\usepackage{cleveref}
\usepackage[most]{tcolorbox}
\newcommand\coloredcomponent[2]
{
  \tcolorboxenvironment{#1}
  {
    breakable,
    enhanced,
    colback = white,
    colframe = #2,
    boxrule = 1pt,
    left = 2pt,
    right = 2pt,
    top = 2pt,
    bottom = 2pt,
    sharp corners,
    before skip = \topsep,
    after skip = \topsep,
  }
}

\counterwithin{equation}{section}
\theoremstyle{definition}

\newtheorem{theorem}{Theorem}[section]
\coloredcomponent{theorem}{blue}

\newtheorem{proposition}[theorem]{Proposition}
\coloredcomponent{proposition}{blue}

\newtheorem{lemma}[theorem]{Lemma}
\coloredcomponent{lemma}{blue}

\newtheorem{corollary}[theorem]{Corollary}
\coloredcomponent{corollary}{blue}

\newtheorem{hypothesis}[theorem]{Hypothesis}
\coloredcomponent{hypothesis}{red}

\newtheorem{notation}[theorem]{Notation}
\coloredcomponent{notation}{green}

\newtheorem{definition}[theorem]{Definition}
\coloredcomponent{definition}{green}

\newtheorem{formulation}[theorem]{Formulation}
\coloredcomponent{formulation}{brown}

\newtheorem{discussion}[theorem]{Discussion}
\coloredcomponent{discussion}{yellow}

\newtheorem{example}[theorem]{Example}
\coloredcomponent{example}{purple}

\newtheorem{remark}[theorem]{Remark}
\coloredcomponent{remark}{orange}

\coloredcomponent{proof}{cyan}

\renewcommand{\thefootnote}{\arabic{footnote}}

\newcommand{\N}{\mathbb{N}}
\newcommand{\Z}{\mathbb{Z}}
\newcommand{\Q}{\mathbb{Q}}
\newcommand{\R}{\mathbb{R}}
\renewcommand{\S}{\mathbb{S}}

\newcommand{\set}[2]{\left\{ #1 \mid #2 \right\}}
\newcommand{\restrict}[2]{\left. #1 \right|_{#2}}

\newcommand{\norm}[1]{\left\lVert#1\right\rVert}
\newcommand{\abs}[1]{\left\lvert#1\right\rvert}
\newcommand{\inner}[2]{\langle#1,#2\rangle}

\newcommand{\linearspan}{\mathop{\rm span}\nolimits}
\newcommand{\Ker}{\mathop{\rm Ker}\nolimits}
\renewcommand{\Im}{\mathop{\rm Im}\nolimits}
\newcommand{\Hom}{\mathop{\rm Hom}\nolimits}
\newcommand{\id}{\mathop{\rm id}\nolimits}
\newcommand{\tr}{\mathop{\rm tr}\nolimits}
\newcommand{\sym}{\mathop{\rm sym}\nolimits}

\newcommand{\grad}{\mathop{\rm grad}\nolimits}
\renewcommand{\div}{\mathop{\rm div}\nolimits}

\newcommand{\Aff}{\mathop{\rm Aff}\nolimits}
\newcommand{\Con}{\mathop{\rm Con}\nolimits}

\newcommand{\Cl}{\mathop{\rm Cl}\nolimits}
\newcommand{\Link}{\mathop{\rm Link}\nolimits}

\newcommand{\sgn}{\mathop{\rm sgn}\nolimits}
\newcommand{\OR}{\mathop{\rm or}\nolimits}
\newcommand{\vol}{\mathop{\rm vol}\nolimits}
\newcommand{\usmile}{\underline{\smile}}
\newcommand{\uwedge}{\underline{\wedge}}

\title{IDEC documentation}
\author{Kiprian Berbatov}
\date{11 September 2024}

\begin{document}

\maketitle

\tableofcontents

\section{Exterior algebra}
\label{section:exterior_algebra}
\begin{definition}
  Let
    $\mathcal{K}$ be a quasi-cubical mesh,
    $D = \dim \mathcal{K}$,
    $E$ be a vector bundle on $\mathcal{K}$,
    $p \in \{0, ..., D\}$.
  The space of \textbf{$E$-valued differential $p$-forms} is the
  $(C^0 \mathcal{K})$-module
  \begin{equation}
    \Omega^{p}(\mathcal{K}, E)
    := \Gamma(E \otimes L^p \mathcal{K})
    \simeq \Gamma E \otimes \Omega^p \mathcal{K}.
  \end{equation}
  The \textbf{algebra of $E$-valued differential forms} is the
  $(C^0 \mathcal{K})$-algebra
  \begin{equation}
    \Omega^\bullet(\mathcal{K}, E)
    := \bigoplus_{p = 0}^D \Omega^p(\mathcal{K}, E).
  \end{equation}
\end{definition}
\begin{remark}
  Clearly, $\Omega^{0}(\mathcal{K}, E) \simeq \Gamma E$.
\end{remark}
\begin{definition}
  Let
    $\mathcal{K}$ be a quasi-cubical mesh,
    $D = \dim \mathcal{K}$,
    $E$ be a vector bundle on $\mathcal{K}$,
    $p, q \in \N$.
  The \textbf{cup product of an $E$-valued differential forms with a form}
  is the map
  \begin{equation}
    \usmile
    \colon \Omega^p(\mathcal{K}, E) \times \Omega^q \mathcal{K}
    \to \Omega^{p + q}(\mathcal{K}, E),
  \end{equation}
  defined as follows: it is the unique bilinear map such that for any
  $\sigma \in \Gamma E$,
  $\omega \in \Omega^p \mathcal{K}$,
  $\eta \in \Omega^q \mathcal{K}$,
  \begin{equation}
    (\sigma \otimes \omega) \usmile \eta :=
    \sigma \otimes (\omega \smile \eta).
  \end{equation}
\end{definition}
\begin{remark}
  The extension of the cup product define above is a straightforward
  generalisation of the cup product when one of the objects has two ``legs'':
  a bundle leg and a form leg.
  (Only the form legs multiply; the bundle leg stays the same.)
  We can further generalise if we have an operation on the bundle sections
  (e.g., a Lie bracket),
  or even an external operation between the sections of different bundles
  (e.g., a tensor product). 
\end{remark}
\begin{definition}
  Let
    $\mathcal{K}$ be a quasi-cubical mesh,
    $D = \dim \mathcal{K}$,
    $E, F, G$ be vector bundles on $\mathcal{K}$,
    $\mu \colon \Gamma E \times \Gamma F \to \Gamma G$ be a bilinear map,
    $p, q \in \N$.
  The \textbf{$\mu$-cup product of bundle-valued differential forms} is the map
  \begin{equation}
    \smile_\mu
    \colon \Omega^p(\mathcal{K}, E) \times \Omega^q(\mathcal{K}, F)
    \to \Omega^{p + q}(\mathcal{K}, G),
  \end{equation}
  defined as follows: it is the unique bilinear map such that for any
  $\sigma \in \Gamma E$,
  $\tau \in \Gamma F$,
  $\omega \in \Omega^p \mathcal{K}$,
  $\eta \in \Omega^q \mathcal{K}$,
  \begin{equation}
    (\sigma \otimes \omega) \smile_\mu (\tau \otimes \eta) :=
    \mu(\sigma, \tau) \otimes (\omega \smile \eta).
  \end{equation}
\end{definition}
\begin{example}
  Let
    $\mathcal{K}$ be a quasi-cubical mesh,
    $D = \dim \mathcal{K}$,
    $p, q \in \N$.
  The \textbf{$[\cdot, \cdot]$-cup product of vector-valued differential forms}
  is the map
  \begin{equation}
    \smile_{[\cdot, \cdot]}
    \colon \Omega^p(\mathcal{K}, \mathfrak{X} \mathcal{K})
    \times \Omega^q(\mathcal{K}, \mathfrak{X} \mathcal{K})
    \to \Omega^{p + q}(\mathcal{K}, \mathfrak{X} \mathcal{K}),
  \end{equation}
  defined as follows: it is the unique bilinear map such that for any
  $X, Y \in \mathfrak{X} \mathcal{K}$,
  $\omega \in \Omega^p \mathcal{K}$,
  $\eta \in \Omega^q \mathcal{K}$,
  \begin{equation}
    (X \otimes \omega) \smile_{[\cdot, \cdot]} (Y \otimes \eta) :=
    [X, Y] \otimes (\omega \smile \eta).
  \end{equation}
\end{example}


\section{Inner products and Hodge star}
\label{section:inner_products_and_hodge_star}
\begin{definition}
  Let
    $\mathcal{K}$ be a quasi-cubical mesh,
    $D = \dim \mathcal{K}$,
    $E$ be a vector bundle on $\mathcal{K}$,
    $p \in \{0, ..., D\}$.
  The space of \textbf{$E$-valued differential $p$-forms} is the
  $(C^0 \mathcal{K})$-module
  \begin{equation}
    \Omega^{p}(\mathcal{K}, E)
    := \Gamma(E \otimes L^p \mathcal{K})
    \simeq \Gamma E \otimes \Omega^p \mathcal{K}.
  \end{equation}
  The \textbf{algebra of $E$-valued differential forms} is the
  $(C^0 \mathcal{K})$-algebra
  \begin{equation}
    \Omega^\bullet(\mathcal{K}, E)
    := \bigoplus_{p = 0}^D \Omega^p(\mathcal{K}, E).
  \end{equation}
\end{definition}
\begin{remark}
  Clearly, $\Omega^{0}(\mathcal{K}, E) \simeq \Gamma E$.
\end{remark}
\begin{definition}
  Let
    $\mathcal{K}$ be a quasi-cubical mesh,
    $D = \dim \mathcal{K}$,
    $E$ be a vector bundle on $\mathcal{K}$,
    $p, q \in \N$.
  The \textbf{cup product of an $E$-valued differential forms with a form}
  is the map
  \begin{equation}
    \usmile
    \colon \Omega^p(\mathcal{K}, E) \times \Omega^q \mathcal{K}
    \to \Omega^{p + q}(\mathcal{K}, E),
  \end{equation}
  defined as follows: it is the unique bilinear map such that for any
  $\sigma \in \Gamma E$,
  $\omega \in \Omega^p \mathcal{K}$,
  $\eta \in \Omega^q \mathcal{K}$,
  \begin{equation}
    (\sigma \otimes \omega) \usmile \eta :=
    \sigma \otimes (\omega \smile \eta).
  \end{equation}
\end{definition}
\begin{remark}
  The extension of the cup product define above is a straightforward
  generalisation of the cup product when one of the objects has two ``legs'':
  a bundle leg and a form leg.
  (Only the form legs multiply; the bundle leg stays the same.)
  We can further generalise if we have an operation on the bundle sections
  (e.g., a Lie bracket),
  or even an external operation between the sections of different bundles
  (e.g., a tensor product). 
\end{remark}
\begin{definition}
  Let
    $\mathcal{K}$ be a quasi-cubical mesh,
    $D = \dim \mathcal{K}$,
    $E, F, G$ be vector bundles on $\mathcal{K}$,
    $\mu \colon \Gamma E \times \Gamma F \to \Gamma G$ be a bilinear map,
    $p, q \in \N$.
  The \textbf{$\mu$-cup product of bundle-valued differential forms} is the map
  \begin{equation}
    \smile_\mu
    \colon \Omega^p(\mathcal{K}, E) \times \Omega^q(\mathcal{K}, F)
    \to \Omega^{p + q}(\mathcal{K}, G),
  \end{equation}
  defined as follows: it is the unique bilinear map such that for any
  $\sigma \in \Gamma E$,
  $\tau \in \Gamma F$,
  $\omega \in \Omega^p \mathcal{K}$,
  $\eta \in \Omega^q \mathcal{K}$,
  \begin{equation}
    (\sigma \otimes \omega) \smile_\mu (\tau \otimes \eta) :=
    \mu(\sigma, \tau) \otimes (\omega \smile \eta).
  \end{equation}
\end{definition}
\begin{example}
  Let
    $\mathcal{K}$ be a quasi-cubical mesh,
    $D = \dim \mathcal{K}$,
    $p, q \in \N$.
  The \textbf{$[\cdot, \cdot]$-cup product of vector-valued differential forms}
  is the map
  \begin{equation}
    \smile_{[\cdot, \cdot]}
    \colon \Omega^p(\mathcal{K}, \mathfrak{X} \mathcal{K})
    \times \Omega^q(\mathcal{K}, \mathfrak{X} \mathcal{K})
    \to \Omega^{p + q}(\mathcal{K}, \mathfrak{X} \mathcal{K}),
  \end{equation}
  defined as follows: it is the unique bilinear map such that for any
  $X, Y \in \mathfrak{X} \mathcal{K}$,
  $\omega \in \Omega^p \mathcal{K}$,
  $\eta \in \Omega^q \mathcal{K}$,
  \begin{equation}
    (X \otimes \omega) \smile_{[\cdot, \cdot]} (Y \otimes \eta) :=
    [X, Y] \otimes (\omega \smile \eta).
  \end{equation}
\end{example}


\section{Jagged arrays}
\label{section:jagged_arrays}
\begin{definition}
  \label{idec/mesh/quasi_cubical/hodge_star/concept-definition}
  Let
    $d \in \N$,
    $K$ be a compatibly oriented quasi-cubical
    \hyperref[idec:mesh:definition]{mesh} of dimension $d$,
    $[K] := \sum_{c_d \in K_d} c^d$ be the fundamental class of $K$
    $\inner{\cdot}{\cdot}$ be an inner product on $K$,
    $p \in \{0, ..., d\}$.
  The \textbf{Hodge star operator on $p$-cochains}
  $\star_p \colon C^p K \to C^{d - p} K$
  is defined as the unique map satisfying the following equation:
  for any $\pi^p \in C^p K$ and $\rho^{d - p} \in C^{d - p} K$,
  \begin{equation}
    \inner{\rho^{d - p}}{\star_p \pi^p}_{d - p}
    = (\rho^{d - p} \smile \pi^p)[K].
  \end{equation}
  The operator $\star_p$ has physical dimension $[L^{d - 2 p}]$.
\end{definition}

\begin{example}
  Let $R$ be a commutative ring with unity.
  The following are examples of modules over $R$.
  \begin{enumerate}
    \item
      For any $n \in \N$, the space $R^n$ is a module over $R$ under
      component-wise addition and multiplication with a scalar.
    \item
      For any $m, n \in \N$, the space $M_{m \times n}(R)$ of $m \times n$
      matrices with elements in $R$ is a module over $R$ under under
      component-wise addition and multiplication with a scalar.
    \item
      For any set $X$, the ring $R^X$ can also be considered as a module over
      $R$ with pointwise addition and multiplication with a scalar.
      It generalises the previous two cases when $X = \{1, ..., n\}$ and
      $X = \{1, ..., m\} \times \{1, ..., n\}$ respectively.
  \end{enumerate}
\end{example}

\begin{definition}
  A \hyperref[idec:mesh:definition]{mesh}
  is represented in memory by its topology and geometry.
  The topology is represented using:
  \begin{itemize}
    \item
      integer \emph{dim} ($d$) storing mesh's dimension;
    \item
      array \emph{cn} storing the number of $p$-cells for $p = 0, ..., d$;
    \item
      a jagged array \emph{cf} of order $4$ storing its topology.
      More precisely, if $0 < p \leq d$, $0 \leq q < p$,
      $0 \leq i < {\rm cn}[p]$, then ${\rm cf}[p][q][i]$ stores the indices of
      all $q$-dimensional subfaces of the $i$-th $p$-cell.
  \end{itemize}
  For a mesh of flat polytopes, the topology is represented by:
  \begin{itemize}
    \item
      integer \emph{dim\_embedded} ($d$) storing mesh's embedding dimension;
    \item
      array of floating point numbers \emph{coord} of length
      ${\rm cn}[0] * {\rm dim\_embedded}$ storing coordinates of the nodes
      (${\rm cn}[0]$ nodes of dimension $\R^{{\rm cn}[0]}$).
  \end{itemize}
\end{definition}

\begin{example}
  Consider two triangles dividing a square.
  Representation of the topology: ${\rm dim} = 2$, ${\rm cn} = \{4, 5, 2\}$,
  \begin{equation}
    {\rm cf} =
    (
      (
        ((0, 1), (1, 2), (2, 3), (3, 0), (0, 2)),
      ),
      (
        ((0, 1, 2), (0, 1, 3)),
        ((0, 1, 4), (2, 3, 4))
      )
    ).
  \end{equation}
  Representation of the geometry: ${\rm dim\_embedded} = 2$,
  ${\rm coord} = (-1, -1, 1, -1, 1, 1, 1, -1)$
  (the latter represents the nodes with coordinates
  $(-1, -1), (1, -1), (1, 1), (1, -1)$ respectively).
\end{example}


\section{Meshes}
\label{section:meshes}
\begin{definition}
  \label{idec/mesh/quasi_cubical/hodge_star/concept-definition}
  Let
    $d \in \N$,
    $K$ be a compatibly oriented quasi-cubical
    \hyperref[idec:mesh:definition]{mesh} of dimension $d$,
    $[K] := \sum_{c_d \in K_d} c^d$ be the fundamental class of $K$
    $\inner{\cdot}{\cdot}$ be an inner product on $K$,
    $p \in \{0, ..., d\}$.
  The \textbf{Hodge star operator on $p$-cochains}
  $\star_p \colon C^p K \to C^{d - p} K$
  is defined as the unique map satisfying the following equation:
  for any $\pi^p \in C^p K$ and $\rho^{d - p} \in C^{d - p} K$,
  \begin{equation}
    \inner{\rho^{d - p}}{\star_p \pi^p}_{d - p}
    = (\rho^{d - p} \smile \pi^p)[K].
  \end{equation}
  The operator $\star_p$ has physical dimension $[L^{d - 2 p}]$.
\end{definition}

\begin{definition}
  A \hyperref[idec:mesh:definition]{mesh}
  is represented in memory by its topology and geometry.
  The topology is represented using:
  \begin{itemize}
    \item
      integer \emph{dim} ($d$) storing mesh's dimension;
    \item
      array \emph{cn} storing the number of $p$-cells for $p = 0, ..., d$;
    \item
      a jagged array \emph{cf} of order $4$ storing its topology.
      More precisely, if $0 < p \leq d$, $0 \leq q < p$,
      $0 \leq i < {\rm cn}[p]$, then ${\rm cf}[p][q][i]$ stores the indices of
      all $q$-dimensional subfaces of the $i$-th $p$-cell.
  \end{itemize}
  For a mesh of flat polytopes, the topology is represented by:
  \begin{itemize}
    \item
      integer \emph{dim\_embedded} ($d$) storing mesh's embedding dimension;
    \item
      array of floating point numbers \emph{coord} of length
      ${\rm cn}[0] * {\rm dim\_embedded}$ storing coordinates of the nodes
      (${\rm cn}[0]$ nodes of dimension $\R^{{\rm cn}[0]}$).
  \end{itemize}
\end{definition}

\begin{example}
  Consider two triangles dividing a square.
  Representation of the topology: ${\rm dim} = 2$, ${\rm cn} = \{4, 5, 2\}$,
  \begin{equation}
    {\rm cf} =
    (
      (
        ((0, 1), (1, 2), (2, 3), (3, 0), (0, 2)),
      ),
      (
        ((0, 1, 2), (0, 1, 3)),
        ((0, 1, 4), (2, 3, 4))
      )
    ).
  \end{equation}
  Representation of the geometry: ${\rm dim\_embedded} = 2$,
  ${\rm coord} = (-1, -1, 1, -1, 1, 1, 1, -1)$
  (the latter represents the nodes with coordinates
  $(-1, -1), (1, -1), (1, 1), (1, -1)$ respectively).
\end{example}


\section{Relative orientation on meshes}
\label{section:relative_orientation_on_meshes}
\begin{theorem}
  Let
    $d \in \N$,
    $M$ be a \hyperref[cmc:mesh:definition]{mesh} of dimension $d$,
    $p \in \{2, ..., d\}$,
    $a_p \in M_p$,
    $c_{p - 2} \in M_{p - 2}$,
    $a_p \succ c_{p - 2}$.
  Then there exist exactly two $(p - 1)$-cells
  $b_{p - 1}, b_{p - 1}' \in M_{p - 1}$
  that are between $a_p$ $c_{p - 2}$, i.e.,
  \begin{equation}
    a_p \succ b_{p - 1} \succ c_{p - 2}\ \text{and}\
    a_p \succ b_{p - 1}'' \succ c_{p - 2}.
  \end{equation}
\end{theorem}

\begin{definition}
  \label{idec/mesh/quasi_cubical/hodge_star/concept-definition}
  Let
    $d \in \N$,
    $K$ be a compatibly oriented quasi-cubical
    \hyperref[idec:mesh:definition]{mesh} of dimension $d$,
    $[K] := \sum_{c_d \in K_d} c^d$ be the fundamental class of $K$
    $\inner{\cdot}{\cdot}$ be an inner product on $K$,
    $p \in \{0, ..., d\}$.
  The \textbf{Hodge star operator on $p$-cochains}
  $\star_p \colon C^p K \to C^{d - p} K$
  is defined as the unique map satisfying the following equation:
  for any $\pi^p \in C^p K$ and $\rho^{d - p} \in C^{d - p} K$,
  \begin{equation}
    \inner{\rho^{d - p}}{\star_p \pi^p}_{d - p}
    = (\rho^{d - p} \smile \pi^p)[K].
  \end{equation}
  The operator $\star_p$ has physical dimension $[L^{d - 2 p}]$.
\end{definition}

\begin{remark}
  Note that the last condition in the above definition can be written as
  \begin{equation}
    \sum_{b_{p - 1} \in (c_{p - 2}, a_p)}
    \varepsilon(a_p, b_{p - 1}) \varepsilon(b_{p - 1}, c_{p - 2}) = 0.
  \end{equation}
\end{remark}

\begin{theorem}
  Let
    $d \in \N$,
    $M$ be a \hyperref[idec:mesh:definition]{mesh} of dimension $d$,
    $R$ be a commutative ring with unity.
  Then there exists a relative orientation on $M$.
\end{theorem}


\section{Chains and boundary operator on meshes}
\label{section:chains_and_boundary_operator_on_meshes}
\begin{definition}
  \label{cmc:mesh_chain:definition}
  Let
    $d \in \N$,
    $M$ be a \hyperref[cmc:mesh:definition]{mesh} of dimension $d$,
    $p \in \N,\ p \in [0, d]$,
    $R$ be a commutative ring with unity (for instance, $R = \R$).
  The space $C_p(M; R)$ of \textbf{$p$-chains}
  on $M$ with coefficients in $R$ is the free $R$-module
  (vector space over $R$ when $R$ is a field, e.g., when $R = \R$)
  generated by $M_p$:
  \begin{equation}
    C_p(M; R) := {\rm Free}_R(M_p).
  \end{equation}
  In other words, the elements of $C_p(M; R)$ are the formal linear combinations
  of cells in $M_p$ in coefficients in $R$.
  An element $c_p$ of $C_p(M; R)$ has the form
  \begin{equation}
    c_p := \lambda_0 c(p_, h_0) + ... + \lambda_{n - 1} c(p, h_{n - 1}),
  \end{equation}
  where for $i = 0, ..., n - 1$, $\lambda_i \in R$ and $c(p, h_i) \in M_p$.
\end{definition}

\begin{definition}
  \label{idec/mesh/quasi_cubical/hodge_star/concept-definition}
  Let
    $d \in \N$,
    $K$ be a compatibly oriented quasi-cubical
    \hyperref[idec:mesh:definition]{mesh} of dimension $d$,
    $[K] := \sum_{c_d \in K_d} c^d$ be the fundamental class of $K$
    $\inner{\cdot}{\cdot}$ be an inner product on $K$,
    $p \in \{0, ..., d\}$.
  The \textbf{Hodge star operator on $p$-cochains}
  $\star_p \colon C^p K \to C^{d - p} K$
  is defined as the unique map satisfying the following equation:
  for any $\pi^p \in C^p K$ and $\rho^{d - p} \in C^{d - p} K$,
  \begin{equation}
    \inner{\rho^{d - p}}{\star_p \pi^p}_{d - p}
    = (\rho^{d - p} \smile \pi^p)[K].
  \end{equation}
  The operator $\star_p$ has physical dimension $[L^{d - 2 p}]$.
\end{definition}

\begin{definition}
  \label{idec/mesh/quasi_cubical/hodge_star/concept-definition}
  Let
    $d \in \N$,
    $K$ be a compatibly oriented quasi-cubical
    \hyperref[idec:mesh:definition]{mesh} of dimension $d$,
    $[K] := \sum_{c_d \in K_d} c^d$ be the fundamental class of $K$
    $\inner{\cdot}{\cdot}$ be an inner product on $K$,
    $p \in \{0, ..., d\}$.
  The \textbf{Hodge star operator on $p$-cochains}
  $\star_p \colon C^p K \to C^{d - p} K$
  is defined as the unique map satisfying the following equation:
  for any $\pi^p \in C^p K$ and $\rho^{d - p} \in C^{d - p} K$,
  \begin{equation}
    \inner{\rho^{d - p}}{\star_p \pi^p}_{d - p}
    = (\rho^{d - p} \smile \pi^p)[K].
  \end{equation}
  The operator $\star_p$ has physical dimension $[L^{d - 2 p}]$.
\end{definition}

\begin{proposition}
  Let
    $d \in \N$,
    $M$ be a \hyperref[cmc:mesh:definition]{mesh} of dimension $d$,
    $R$ be a commutative ring with unity,
    $\varepsilon$ be a relative orientation on $M$.
  Then the algebra $(C_\bullet M, \partial)$ is a chain complex, i.e.,
  \begin{equation}
    \partial^2 = 0.
  \end{equation}
\end{proposition}

\begin{proof}
  It is enough to prove that for any $p \in \{0, ..., d\}$, $c_p \in M_d$,
  \begin{equation}
    \partial^2 c_p = 0.
  \end{equation}
  The proposition is trivially true for $p = 0$ and $p = 1$
  because $\partial_0 = 0$.
  Assume that $p \geq 2$.
  Then
  \begin{equation}
    \begin{split}
      \partial^2 a_p
      & = \partial_{p - 1} (\partial_p a_p) \\
      & = \partial_{p - 1}
      \left(
        \sum_{b_{p - 1} \prec a_p} \varepsilon(a_p, b_{p - 1}) b_{p - 1}
      \right) \\
      & =
      \sum_{b_{p - 1} \prec a_p}
        \sum_{c_{p - 2} \prec b_{p - 1}}
            \varepsilon(a_p, b_{p - 1})
            \varepsilon(b_{p - 1}, c_{p - 2})
            c_{p - 2} \\
      & =
      \sum_{c_{p - 2} \prec a_p}
        \left(
          \sum _{b_{p - 1} \in (c_{p - 2}, a_p)}
            \varepsilon(a_p, b_{p - 1}) \varepsilon(b_{p - 1}, c_{p - 2})
        \right)
        c_{p - 2} \\
      & = 0
    \end{split}
  \end{equation}
  (the last equation follows from the last condition in the definition of
  \hyperref[idec:relative_orientation:definition]{relative orientation}).
\end{proof}

\begin{proposition}
  Let
    $d \in \N$,
    $M$ be a \hyperref[idec:mesh:definition]{mesh} of dimension $d$,
    $R$ be a commutative ring with unity,
    $\varepsilon$ and $\varepsilon'$ be a relative orientations on $M$
      with corresponding boundary operators
      $\partial$ and $\partial'$ respectively.
  Then
  \begin{equation}
    (C_\bullet(M; R), \partial) \cong (C_\bullet(M; R), \partial')
  \end{equation}
  ($\cong$ is understood as isomorphism of chain complexes).
\end{proposition}

\begin{remark}
  The above proposition says that the boundary operator is essentially unique,
  i.e., up to isomorphism it does not depend on the chosen relative orientation.
  This motivates the notion of ``\emph{the} boundary operator'' on a mesh.
  Nevertheless, this does not exclude special choices of relative orientations
  in some particular cases like compatibly orientable meshes or regular grids.
\end{remark}


\section{Cochains and coboundary operator on meshes}
\label{section:cochains_and_coboundary_operator_on_meshes}
\begin{definition}
  \label{idec/mesh/quasi_cubical/hodge_star/concept-definition}
  Let
    $d \in \N$,
    $K$ be a compatibly oriented quasi-cubical
    \hyperref[idec:mesh:definition]{mesh} of dimension $d$,
    $[K] := \sum_{c_d \in K_d} c^d$ be the fundamental class of $K$
    $\inner{\cdot}{\cdot}$ be an inner product on $K$,
    $p \in \{0, ..., d\}$.
  The \textbf{Hodge star operator on $p$-cochains}
  $\star_p \colon C^p K \to C^{d - p} K$
  is defined as the unique map satisfying the following equation:
  for any $\pi^p \in C^p K$ and $\rho^{d - p} \in C^{d - p} K$,
  \begin{equation}
    \inner{\rho^{d - p}}{\star_p \pi^p}_{d - p}
    = (\rho^{d - p} \smile \pi^p)[K].
  \end{equation}
  The operator $\star_p$ has physical dimension $[L^{d - 2 p}]$.
\end{definition}

\begin{definition}
  \label{idec/mesh/quasi_cubical/hodge_star/concept-definition}
  Let
    $d \in \N$,
    $K$ be a compatibly oriented quasi-cubical
    \hyperref[idec:mesh:definition]{mesh} of dimension $d$,
    $[K] := \sum_{c_d \in K_d} c^d$ be the fundamental class of $K$
    $\inner{\cdot}{\cdot}$ be an inner product on $K$,
    $p \in \{0, ..., d\}$.
  The \textbf{Hodge star operator on $p$-cochains}
  $\star_p \colon C^p K \to C^{d - p} K$
  is defined as the unique map satisfying the following equation:
  for any $\pi^p \in C^p K$ and $\rho^{d - p} \in C^{d - p} K$,
  \begin{equation}
    \inner{\rho^{d - p}}{\star_p \pi^p}_{d - p}
    = (\rho^{d - p} \smile \pi^p)[K].
  \end{equation}
  The operator $\star_p$ has physical dimension $[L^{d - 2 p}]$.
\end{definition}

\begin{proposition}
  Let
    $d \in \N$,
    $M$ be a \hyperref[cmc:mesh:definition]{mesh} of dimension $d$,
    $\partial$ be a boundary operator on $M$m
    $D$ be the discrete differential on $M$.
  Then $(\Omega^\bullet M, D)$ is a cochain complex, i.e.,
  \begin{equation}
    D^2 = 0.
  \end{equation}
\end{proposition}

\begin{proof}
  A straightforward computation using the fact that $\partial^2 = 0$.
  Indeed, let $\omega \in \Omega^p M$.
  Then
  \begin{equation}
    \begin{split}
      D^2(\omega)
      & = D(D \omega) \\
      & = D(\omega \circ \partial - (-1)^p \partial \circ \omega) \\
      & =
      \omega \circ \partial \circ \partial
        - (-1)^{p + 1} \partial \circ \omega \circ \partial
        - (-1)^p \partial \circ \omega \circ \partial
        - (-1)^p (-1)^{p + 1} \partial \circ \partial \circ \omega \\
      & =
      0
      - (-1)^{p + 1}\partial \circ \omega \circ \partial
      + (-1)^{p + 1}\partial \circ \omega \circ \partial
      + 0 \\
      & = 0.
    \end{split}
  \end{equation}
\end{proof}


\section{Combinatorial differential forms and Forman subdivision}
\label{section:combinatorial_differential_forms_and_forman_subdivision}
\begin{definition}
  \label{idec/mesh/quasi_cubical/hodge_star/concept-definition}
  Let
    $d \in \N$,
    $K$ be a compatibly oriented quasi-cubical
    \hyperref[idec:mesh:definition]{mesh} of dimension $d$,
    $[K] := \sum_{c_d \in K_d} c^d$ be the fundamental class of $K$
    $\inner{\cdot}{\cdot}$ be an inner product on $K$,
    $p \in \{0, ..., d\}$.
  The \textbf{Hodge star operator on $p$-cochains}
  $\star_p \colon C^p K \to C^{d - p} K$
  is defined as the unique map satisfying the following equation:
  for any $\pi^p \in C^p K$ and $\rho^{d - p} \in C^{d - p} K$,
  \begin{equation}
    \inner{\rho^{d - p}}{\star_p \pi^p}_{d - p}
    = (\rho^{d - p} \smile \pi^p)[K].
  \end{equation}
  The operator $\star_p$ has physical dimension $[L^{d - 2 p}]$.
\end{definition}

\begin{definition}
  Let
    $d \in \N$,
    $M$ be a \hyperref[idec:mesh:definition]{mesh} of dimension $d$
    $\partial$ be a boundary operator on $M$.
  The \textbf{discrete differential} on $M$ is the linear map
  \begin{equation}
    D \colon \Omega^\bullet \to \Omega^\bullet
  \end{equation}
  which maps a $p$-form $\omega$ to a $(p + 1)$-form by the formula
  \begin{equation}
    D \omega := \omega \circ \partial - (-1)^p \partial \circ \omega.
  \end{equation}
\end{definition}

\begin{proposition}
  Let
    $d \in \N$,
    $M$ be a \hyperref[cmc:mesh:definition]{mesh} of dimension $d$,
    $\partial$ be a boundary operator on $M$m
    $D$ be the discrete differential on $M$.
  Then $(\Omega^\bullet M, D)$ is a cochain complex, i.e.,
  \begin{equation}
    D^2 = 0.
  \end{equation}
\end{proposition}

\begin{proof}
  A straightforward computation using the fact that $\partial^2 = 0$.
  Indeed, let $\omega \in \Omega^p M$.
  Then
  \begin{equation}
    \begin{split}
      D^2(\omega)
      & = D(D \omega) \\
      & = D(\omega \circ \partial - (-1)^p \partial \circ \omega) \\
      & =
      \omega \circ \partial \circ \partial
        - (-1)^{p + 1} \partial \circ \omega \circ \partial
        - (-1)^p \partial \circ \omega \circ \partial
        - (-1)^p (-1)^{p + 1} \partial \circ \partial \circ \omega \\
      & =
      0
      - (-1)^{p + 1}\partial \circ \omega \circ \partial
      + (-1)^{p + 1}\partial \circ \omega \circ \partial
      + 0 \\
      & = 0.
    \end{split}
  \end{equation}
\end{proof}

\begin{definition}
  \label{idec/mesh/quasi_cubical/hodge_star/concept-definition}
  Let
    $d \in \N$,
    $K$ be a compatibly oriented quasi-cubical
    \hyperref[idec:mesh:definition]{mesh} of dimension $d$,
    $[K] := \sum_{c_d \in K_d} c^d$ be the fundamental class of $K$
    $\inner{\cdot}{\cdot}$ be an inner product on $K$,
    $p \in \{0, ..., d\}$.
  The \textbf{Hodge star operator on $p$-cochains}
  $\star_p \colon C^p K \to C^{d - p} K$
  is defined as the unique map satisfying the following equation:
  for any $\pi^p \in C^p K$ and $\rho^{d - p} \in C^{d - p} K$,
  \begin{equation}
    \inner{\rho^{d - p}}{\star_p \pi^p}_{d - p}
    = (\rho^{d - p} \smile \pi^p)[K].
  \end{equation}
  The operator $\star_p$ has physical dimension $[L^{d - 2 p}]$.
\end{definition}

\begin{definition}
  Let
    $d \in \N$,
    $M$ be a \hyperref[cmc:mesh:definition]{mesh} of dimension $d$,
    $K$ be the Forman subdivision of $M$,
    $\varepsilon_M$ be the relative orientation on $M$.
  We construct the relative orientation $\varepsilon_K$ as follows.
  Let
    $p_f \in [1, d]$,
    $p \in [p_f, d]$,
    $s = p - p_f$,
    $c_K(p_f, i_f)$ be a $p_f$-cell on $K$,
  \begin{equation}
    c_K(p_f, i_f) = (c(p, i), c(s, l))\
    \text{for some $c(p, i) \in C_p M$ and $c(s, l) \in C_s M$}.
  \end{equation}
  Let $c_K(p_f - 1, j_f)$ be a hyperface of $c_K(p_f, i_f)$.
  Then there exist $q, r \in \N$ such that
  $p \geq q \geq r \geq s$ and $q - r = p_f - 1$,
  such that
  \begin{equation}
    c_K(p_f - 1, j_f) = (c(q, j), c(r, k))\
    \text{for some $c(q, j) \in C_p M$ and $c(r, k) \in C_r M$}.
  \end{equation}
  There are two possibilities for $q$ and $r$:
  $(q, r) = (p - 1, s)$ or $(q, r) = (p, s + 1)$.
  \begin{enumerate}
    \item
      If $(q, r) = (p - 1, s)$, then
      \begin{equation}
        c_K(p_f - 1, j_f) = (c(p - 1, j), c(s, l)),\
        \text{where $c(p, i) \succ c(p - 1, j) \succeq c(s, l)$}.
      \end{equation}
      In this case
      \begin{equation}
        \varepsilon_K(c_K(p_f, i_f), c_K(p_f - 1, j_f))
        = \varepsilon_M(c_M(p, i), c_M(p - 1, j)).
      \end{equation}
    \item
      If $(q, r) = (p, s + 1)$, then
      \begin{equation}
        c_K(p_f - 1, j_f) = (c(p, i), c(s + 1, k)),\
        \text{where $c(p, i) \succeq c(s + 1, k) \succ c(s, l)$}.
      \end{equation}
      In this case
      \begin{equation}
        \varepsilon_K(c_K(p_f, i_f), c_K(p_f - 1, j_f))
        = (-1)^{p_f} \varepsilon_M(c_M(s + 1, k), c_M(s, l)).
      \end{equation}
  \end{enumerate}
\end{definition}

\begin{theorem}
  Let
    $d \in \N$,
    $M$ be a \hyperref[cmc:mesh:definition]{mesh} of dimension $d$,
    $\varepsilon_M$ be a relative orientation on $M$
      with corresponding boundary operator $\partial_M$
      and discrete differential $D_M$.
  Let $K$ be the Forman subdivision of $M$,
  $\varepsilon_K$ be the orientation on $K$ constructed above,
  $d_K$ be the corresponding coboundary operator on $K$.
  Then
  \begin{equation}
    (\Omega^p M, D_M) \cong (C^p K, d_K).
  \end{equation}
  with the isomorphism being the mapping of the basis forms to basis cochains
  introduced in the construction of $K$.
\end{theorem}

\begin{definition}
  \label{idec/mesh/quasi_cubical/hodge_star/concept-definition}
  Let
    $d \in \N$,
    $K$ be a compatibly oriented quasi-cubical
    \hyperref[idec:mesh:definition]{mesh} of dimension $d$,
    $[K] := \sum_{c_d \in K_d} c^d$ be the fundamental class of $K$
    $\inner{\cdot}{\cdot}$ be an inner product on $K$,
    $p \in \{0, ..., d\}$.
  The \textbf{Hodge star operator on $p$-cochains}
  $\star_p \colon C^p K \to C^{d - p} K$
  is defined as the unique map satisfying the following equation:
  for any $\pi^p \in C^p K$ and $\rho^{d - p} \in C^{d - p} K$,
  \begin{equation}
    \inner{\rho^{d - p}}{\star_p \pi^p}_{d - p}
    = (\rho^{d - p} \smile \pi^p)[K].
  \end{equation}
  The operator $\star_p$ has physical dimension $[L^{d - 2 p}]$.
\end{definition}

\begin{example}
  Let $R$ be a commutative ring with unity.
  The following are examples of modules over $R$.
  \begin{enumerate}
    \item
      For any $n \in \N$, the space $R^n$ is a module over $R$ under
      component-wise addition and multiplication with a scalar.
    \item
      For any $m, n \in \N$, the space $M_{m \times n}(R)$ of $m \times n$
      matrices with elements in $R$ is a module over $R$ under under
      component-wise addition and multiplication with a scalar.
    \item
      For any set $X$, the ring $R^X$ can also be considered as a module over
      $R$ with pointwise addition and multiplication with a scalar.
      It generalises the previous two cases when $X = \{1, ..., n\}$ and
      $X = \{1, ..., m\} \times \{1, ..., n\}$ respectively.
  \end{enumerate}
\end{example}

\begin{definition}
  \label{idec/mesh/quasi_cubical/hodge_star/concept-definition}
  Let
    $d \in \N$,
    $K$ be a compatibly oriented quasi-cubical
    \hyperref[idec:mesh:definition]{mesh} of dimension $d$,
    $[K] := \sum_{c_d \in K_d} c^d$ be the fundamental class of $K$
    $\inner{\cdot}{\cdot}$ be an inner product on $K$,
    $p \in \{0, ..., d\}$.
  The \textbf{Hodge star operator on $p$-cochains}
  $\star_p \colon C^p K \to C^{d - p} K$
  is defined as the unique map satisfying the following equation:
  for any $\pi^p \in C^p K$ and $\rho^{d - p} \in C^{d - p} K$,
  \begin{equation}
    \inner{\rho^{d - p}}{\star_p \pi^p}_{d - p}
    = (\rho^{d - p} \smile \pi^p)[K].
  \end{equation}
  The operator $\star_p$ has physical dimension $[L^{d - 2 p}]$.
\end{definition}

\begin{definition}
  Let
    $(P, \preceq)$ be a partially ordered set,
    $a, b \in P$ with $a \preceq b$.
  The \textbf{closed interval} $[a, b]$ is defined as
  \begin{equation}
    [a, b] := \set{x \in P}{a \preceq x\ \&\ x \preceq b}.
  \end{equation}
\end{definition}

\begin{definition}
  We say that a mesh $M$ is \textbf{interval-simplicial} if for any two cells
  $a, b \in M$ with $a \preceq b$, the interval $[a, b]$ is an abstract simplex.
\end{definition}

\begin{proposition}
  Let
    $M$ be a \hyperref[idec:mesh:definition]{mesh}
    and $K$ be its Forman subdivision.
  Then $M$ is interval-simplicial if and only if $K$ is quasi-cubical.
\end{proposition}

\begin{proposition}
  Let $M$ be a \hyperref[cmc:mesh:definition]{mesh} of dimension at most $2$.
  Then $M$ is interval-simplicial.
  In particular, its Forman subdivision is quasi-cubical.
\end{proposition}

\begin{proposition}
  Let $M$ be a \hyperref[cmc:mesh:definition]{mesh} of dimension at most $2$.
  Then $M$ is interval-simplicial.
  In particular, its Forman subdivision is quasi-cubical.
\end{proposition}

\begin{definition}
  Let $d \in \N$ and $P$ be a polytope of dimension $d$.
  We say that $P$ is a \textbf{simple polytope}
  if any of its nodes is connected to exactly $d$ edges.
\end{definition}

\begin{proposition}
  Let $M$ be a \hyperref[idec:mesh:definition]{mesh} of dimension $3$.
  Then $M$ is interval-simplicial if and only if all $3$-cells of $M$
  are simple polytopes.
  In particular, if $K$ is the Forman subdivision of $M$,
  then $K$ is quasi-cubical if and only if all $3$-cells of $M$
  are simple polytopes.
\end{proposition}

\begin{proposition}
  Let
    $D \in \N$,
    $K$ be a quasi-cubical mesh of dimension $D$,
    $\partial$ be the topological (unsigned) boundary operator on $K$,
    $\delta$ be the topological (unsigned) coboundary operator on $K$,
    $\perp$ be the perpendicularity operator on $K$,
    $p \in \{0, ..., D\}$.
  Then
  \begin{equation}
    \partial_{D - p} \circ \perp_p = \perp_{p + 1} \circ \delta_p.
  \end{equation}
\end{proposition}


\section{Metric-dependent calculus on quasi-cubical meshes}
\label{section:metric_dependent_calculus_on_quasi_cubical_meshes}
\begin{discussion}
  As we saw, the Forman subdivision of an interval-simplicial mesh leads to a
  quasi-cubical mesh.
  Interval-simplicial meshes are not that uncommon:
  \begin{enumerate}
    \item
      all meshes of dimension at most $2$ are interval-simplicial;
    \item
      all $3$D meshes of simple polytopes are interval-simplicial;
    \item
      all simplicial and quasi-cubical meshes are interval-simplicial;
    \item
      the product of interval-simplicial meshes is an interval-simplicial mesh.
  \end{enumerate}
  For that reason we will build our calculus on quasi-cubical meshes thought as
  the Forman subdivision of an interval-simplicial mesh.
\end{discussion}

\begin{definition}
  Let
    $d \in \N$,
    $K$ be a quasi-cubical mesh of dimension $d$,
    $a_d \in M_d$,
    $p \in \{0, ..., d\}$,
    $b_p \in M_p$,
    $c_{d - p} \in M_{d - p}$.
  We say that $b_p$ and $c_{d - p}$ are \textbf{topologically orthogonal}
  with respect to $a_d$ if $b_p, c_{d - p} \preceq a_d$, and the intersection of
  $b_p$ with $c_{d - p}$ is a node in $a_d$.
  In this case we write
  \begin{equation}
    b_p \oplus c_{d - p} = a_p\
    \text{and}\
    b_p \perp c_{d - p}.
  \end{equation}
\end{definition}

\begin{notation}
  Let $d \in \N$,
  $P$ be a polytope of dimension $d$.
  The \textbf{(Euclidean) measure} of $P$ is denoted by
  \begin{equation}
    \mu_d(P) (= \mu(P)).
  \end{equation}
  If $P$ is with standard physical dimensions
  (e.g., it has not been non-dimensionalised),
  then $\mu_d(P)$ is of physical dimension $[L^d]$.
\end{notation}

\begin{definition}
  \label{idec/mesh/quasi_cubical/hodge_star/concept-definition}
  Let
    $d \in \N$,
    $K$ be a compatibly oriented quasi-cubical
    \hyperref[idec:mesh:definition]{mesh} of dimension $d$,
    $[K] := \sum_{c_d \in K_d} c^d$ be the fundamental class of $K$
    $\inner{\cdot}{\cdot}$ be an inner product on $K$,
    $p \in \{0, ..., d\}$.
  The \textbf{Hodge star operator on $p$-cochains}
  $\star_p \colon C^p K \to C^{d - p} K$
  is defined as the unique map satisfying the following equation:
  for any $\pi^p \in C^p K$ and $\rho^{d - p} \in C^{d - p} K$,
  \begin{equation}
    \inner{\rho^{d - p}}{\star_p \pi^p}_{d - p}
    = (\rho^{d - p} \smile \pi^p)[K].
  \end{equation}
  The operator $\star_p$ has physical dimension $[L^{d - 2 p}]$.
\end{definition}

\begin{example}
  Let
    $d \in \N$,
    $h \in \R^+$
    $K$ be a regular cubical mesh of dimension $d$ with size $h$,
    $p \in \{0, ..., d\}$,
    $b_p$ be an internal $p$-cell in $K$.
  Then
  \begin{equation}
    \inner{b^p}{b^p}_p = h^{d - 2 p}.
  \end{equation}
\end{example}

\begin{definition}
  \label{idec/mesh/quasi_cubical/hodge_star/concept-definition}
  Let
    $d \in \N$,
    $K$ be a compatibly oriented quasi-cubical
    \hyperref[idec:mesh:definition]{mesh} of dimension $d$,
    $[K] := \sum_{c_d \in K_d} c^d$ be the fundamental class of $K$
    $\inner{\cdot}{\cdot}$ be an inner product on $K$,
    $p \in \{0, ..., d\}$.
  The \textbf{Hodge star operator on $p$-cochains}
  $\star_p \colon C^p K \to C^{d - p} K$
  is defined as the unique map satisfying the following equation:
  for any $\pi^p \in C^p K$ and $\rho^{d - p} \in C^{d - p} K$,
  \begin{equation}
    \inner{\rho^{d - p}}{\star_p \pi^p}_{d - p}
    = (\rho^{d - p} \smile \pi^p)[K].
  \end{equation}
  The operator $\star_p$ has physical dimension $[L^{d - 2 p}]$.
\end{definition}

\begin{proposition}
  Let
    $d \in \N$,
    $K$ be a compatibly oriented quasi-cubical
    \hyperref[cmc:mesh:definition]{mesh} of dimension $d$,
    $[K] := \sum_{c_d \in K_d} c^d$ be the fundamental class of $K$
    $\inner{\cdot}{\cdot}$ be an orthogonal inner product on $K$,
    $p \in \{0, ..., d\}$.
  The \hyperref[cmc/mesh/quasi_cubical/hodge_star/concept-definition]
               {Hodge star operator}
  $\star_p \colon C^p K \to C^{d - p} K$ has the following closed form:
  for any $\pi^p \in C^p K$ and any $c^{d - p} \in C^{d - p} K$,
  \begin{equation}
    (\star_p \pi^p)(b_{d - p})
    = \sum_{a_p \perp b_{d - p}}
      \frac{(a^p \smile b^{d - p})[K]}{\inner{b^{d - p}}{b^{d - p}}} \pi^p(a_p).
  \end{equation}
\end{proposition}

\begin{corollary}
  Let
    $d \in \N$,
    $h \in \R^+$
    $K$ a cubical grid of dimension $d$ with size $h$
      with its standard orientation,
    $p \in \{0, ..., d\}$,
    $\pi^p \in C^p K$,
    $b_{p - 1} \in K_{p - 1}$ be an internal cell.
  Then
  \begin{equation}
    (\delta^\star_p \pi^p) b_{p - 1} =
    \frac{1}{h^2}
    \sum_{b_{p - 1} \prec a_p}
      \varepsilon(a_p, b_{p - 1})
      \pi^p(a_p).
  \end{equation}
\end{corollary}


\section{Approximating vector fields with 1-cochains}
\label{section:approximating_vector_fields_with_1_cochains}
\begin{definition}
  \label{idec/mesh/quasi_cubical/hodge_star/concept-definition}
  Let
    $d \in \N$,
    $K$ be a compatibly oriented quasi-cubical
    \hyperref[idec:mesh:definition]{mesh} of dimension $d$,
    $[K] := \sum_{c_d \in K_d} c^d$ be the fundamental class of $K$
    $\inner{\cdot}{\cdot}$ be an inner product on $K$,
    $p \in \{0, ..., d\}$.
  The \textbf{Hodge star operator on $p$-cochains}
  $\star_p \colon C^p K \to C^{d - p} K$
  is defined as the unique map satisfying the following equation:
  for any $\pi^p \in C^p K$ and $\rho^{d - p} \in C^{d - p} K$,
  \begin{equation}
    \inner{\rho^{d - p}}{\star_p \pi^p}_{d - p}
    = (\rho^{d - p} \smile \pi^p)[K].
  \end{equation}
  The operator $\star_p$ has physical dimension $[L^{d - 2 p}]$.
\end{definition}

\begin{remark}
  Let $A$ be a matrix that has a physical dimension [X],
    $B$ be a Moore-Penrose inverse of $A$.
  Then $B\ [X^{-1}]$.
\end{remark}

\begin{theorem}
  Let $m, n \in \N$, $A$ be a real $m \times n$ matrix.
  Then $A$ has a unique Moore-Penrose inverse, denoted by $A^*$.
\end{theorem}

\begin{remark}
  If the matrix $A$ is of full rank there exists a closed formula for $A^*$.
  \begin{enumerate}
    \item
      If $A$ is a square matrix with full rank, i.e., an invertible one, then
      $A^* = A^{-1}$.
    \item
      If $m > n$ and $A$ is an $m \times n$ matrix of full rank, then its
      columns are linearly independent which means that $A^T A$ is
      symmetric and positive definite and hence invertible.
      (Its inverse $(A^T A)^{-1}$ is also symmetric and positive definite.)
      It is then easy to check that
      \begin{equation}
        B := (A^T A)^{-1} A^T
      \end{equation}
      is the Moore-Penrose inverse of $A$.
      Indeed, obviously $B$ is a left inverse of $A$, and
      \begin{subequations}
        \begin{alignat}{2}
          & A B A && = A (B A) = A I_n = A, \\
          & B A B && = (B A) B = I_n B = B, \\
          & (A B)^T && = (A (A^T A)^{-1} A^T)^T
            = (A^T)^T ((A^T A)^{-1})^T A^T = A (A^T A)^{-1} A^T = A B, \\
          & (B A)^T && = I_n^T = I_n = B A.
        \end{alignat}
      \end{subequations}
    \item
      If $m < n$ and $A$ is an $m \times n$ matrix of full rank, then an
      analogous reasoning to the previous point shows that
      \begin{equation}
        A^* = A^T (A A^T)^{-1}.
      \end{equation}
  \end{enumerate}
\end{remark}

\begin{definition}
  \label{idec/mesh/quasi_cubical/hodge_star/concept-definition}
  Let
    $d \in \N$,
    $K$ be a compatibly oriented quasi-cubical
    \hyperref[idec:mesh:definition]{mesh} of dimension $d$,
    $[K] := \sum_{c_d \in K_d} c^d$ be the fundamental class of $K$
    $\inner{\cdot}{\cdot}$ be an inner product on $K$,
    $p \in \{0, ..., d\}$.
  The \textbf{Hodge star operator on $p$-cochains}
  $\star_p \colon C^p K \to C^{d - p} K$
  is defined as the unique map satisfying the following equation:
  for any $\pi^p \in C^p K$ and $\rho^{d - p} \in C^{d - p} K$,
  \begin{equation}
    \inner{\rho^{d - p}}{\star_p \pi^p}_{d - p}
    = (\rho^{d - p} \smile \pi^p)[K].
  \end{equation}
  The operator $\star_p$ has physical dimension $[L^{d - 2 p}]$.
\end{definition}

\begin{definition}
  Let:
  \begin{enumerate}
    \item
      $d \in \N,\ d \geq 1$;
    \item
      $K$ be a flat mesh of dimension $d$
      (with a chosen embedding in $\R^d$);
    \item
      $\epsilon$ be a relative orientation on $K$;
    \item
      $\mathcal{N}_i$ be a node in $K$ connected to $n > 0$ edges;
    \item
      $\mathcal{E}_{j_0}, ..., \mathcal{E}_{j_{n - 1}}$ be all the edges
      containing $\mathcal{N}_i$ as a node;
    \item
      $\mathcal{N}_{i_0}, ..., \mathcal{N}_{i_{n - 1}}$ be the other than
      other $\mathcal{N}_i$ of
      $\mathcal{E}_{j_0}, ..., \mathcal{E}_{j_{n - 1}}$ respectively.
  \end{enumerate}
  We define the \textbf{node matrix} $\mathcal{L}_{\mathcal{N}_i}$ of
  $\mathcal{N}_i$ by
  \begin{equation}
    \mathcal{L}_{\mathcal{N}_i} :=
    \begin{pmatrix}
      x_{i_0, 0} - x_{i, 0} & \cdots & x_{i_0, d - 1} - x_{i, d - 1} \\
      \vdots & \ddots & \vdots \\
      x_{i_{n - 1}, 0} - x_{i, 0} & \cdots & x_{i_{n - 1}, d - 1} - x_{i, d - 1}
    \end{pmatrix}
    \in \R^{n \times d}.
  \end{equation}
  The physical dimension of $\mathcal{L}_{\mathcal{N}_i}$ is $[L]$.
\end{definition}

\begin{definition}
  Let
    $d \in \N\ \text{with}\ d \geq 1$,
    $K$ be a mesh of dimension $d$,
    $\pi^1 \in C^1 K$,
    $\mathcal{N}_i \in K_0$.
  Define the \textbf{neighbor representation} $\widehat{\pi^1}_{\mathcal{N}_i}$
  of $\pi^1$ at $\mathcal{N}_i$ by
  \begin{equation}
    \widehat{\pi^1}_{\mathcal{N}_i} :=
    ( \epsilon_K(\mathcal{E}_{j_0}, \mathcal{N}_{i_0}) \pi^1(\mathcal{E}_{j_0}),
      \cdots,
        \epsilon_K(\mathcal{E}_{j_{k - 1}}, \mathcal{N}_{i_{k - 1}})
        \pi^1(\mathcal{E}_{j_{k - 1}})
    ) \in \R^n.
  \end{equation}
  The neighbor is a dimensionless operator.
\end{definition}

\begin{definition}
  Let
    $d \in \N\ \text{with}\ d \geq 1$,
    $K$ be a flat mesh of dimension $d$,
    $\pi^1 \in C^1 K$,
    $\mathcal{N}_i \in K_0$ with corresponding Euclidean coordinates $x_i$.
  Define the \textbf{1-cochain embedding} $\overline{\pi^1}(x_i)$
  of $\pi^1$ at $\mathcal{N}_i$ by
  \begin{equation}
    \overline{\pi^1}(x_i) :=
    \left(\mathcal{L}_{\mathcal{N}_i}\right)^* \widehat{\pi^1}_{\mathcal{N}_i}
    \in \R^d.
  \end{equation}
  The $1$-cochain embedding operator has physical dimension $[L^{-1}]$.
\end{definition}

\begin{example}
  Let $h \in \R^+$ and $K$ be a regular subdivision with size $h$
  of some interval,
  all the edges in $K$ are oriented from left to right,
  $\pi^1 \in C^1 K$.
  \begin{enumerate}
    \item
      Consider an interior point $\mathcal{N}_i$ with neighboring edges
      $\mathcal{E}_{i - 1}$ and $\mathcal{E}_i$ and corresponding neighboring
      nodes $\mathcal{N}_{i - 1}$ and $\mathcal{N}_{i + 1}$.
      Then
      \begin{equation}
        \mathcal{L}_{\mathcal{N}_i} =
        \begin{pmatrix}
          -h \\
          h
        \end{pmatrix}
        \Rightarrow
        (\mathcal{L}_{\mathcal{N}_i})^* =
          \frac{1}{2 h}
          \begin{pmatrix}
            -1 & 1
          \end{pmatrix}
      \end{equation}
      and
      \begin{equation}
        \widehat{\pi^1}_{\mathcal{N}_i} =
        \begin{pmatrix}
          - \pi^1 \mathcal{E}_{i - 1} \\
          \pi^1 \mathcal{E}_i
        \end{pmatrix}
        \Rightarrow
        \overline{\pi^1}(x_i) =
          \frac{1}{2 h}
          \begin{pmatrix}
            -1 & 1
          \end{pmatrix}
          \begin{pmatrix}
            - \pi^1 \mathcal{E}_{i - 1} \\
            \pi^1 \mathcal{E}_i
          \end{pmatrix}
        = \frac{1}{2 h}
          \left(\pi^1 \mathcal{E}_{i - 1} + \pi^1 \mathcal{E}_i\right).
      \end{equation}
  \end{enumerate}
\end{example}

\begin{definition}
  \label{idec/vector_field_to_1_cochain/definition}
  Let
    $K$ be an embedded flat mesh,
    $X$ be the manifold enclosed by $K$,
    $u$ be a vector field on $X$.
  Define the \textbf{approximation} $\underline{u} \in C^1 K$ as follows.
  Let $\mathcal{E}_i$ be an edge with endpoints $\mathcal{N}_{i_0}$ and
  $\mathcal{N}_{i_1}$, oriented from $\mathcal{N}_{i_0}$ to $\mathcal{N}_{i_1}$.
  Let $p_0$ and $p_1$ be such that for $j = 0, 1$,
  the $p_j$-th edge adjacent to $N_{i_j}$ is $E_i$.
  Then
  \begin{equation}
    \underline{u}(\mathcal{E}_i) :=
      \frac{1}{2}
      ( - (\mathcal{L}_{\mathcal{N}_{i_0}} u(x_{i_0}))_{p_0}
        + (\mathcal{L}_{\mathcal{N}_{i_1}} u(x_{i_1}))_{p_1}
      ).
  \end{equation}
  The approximation operator is of physical dimension $[L]$.
\end{definition}

\begin{example}
  With the mesh of the previous example, we have
  \begin{equation}
    \label{cmc/vector_field_to_1_cochain/1d_example:exact_value}
    \underline{u}(\mathcal{E}_i)
    = \frac{1}{2} (h u(x_i) + h u(x_{i + 1}))
    = h \frac{u(x_i) + u(x_{i + 1})}{2}.
  \end{equation}
  Let's calculate the consecutive application of approximation and embedding
  (and vice versa).
  \begin{equation}
    \underline{\left(\overline{\pi^1}\right)}(\mathcal{E}_i)
    = h
      \left(
        \frac{\overline{\pi^1}(x_i) + \overline{\pi^1}(x_{i + 1})}{2}
      \right)
    = \frac{h}{2}
      \frac{1}{2 h}
      ((\pi^1 \mathcal{E}_{i - 1} + \pi^1 \mathcal{E}_i)
       + (\pi^1 \mathcal{E}_i + \pi^1 \mathcal{E}_{i + 1}))
    = \frac
    {\pi^1 \mathcal{E}_{i - 1} + 2 \pi^1 \mathcal{E}_i
      + \pi^1 \mathcal{E}_{i + 1}}
    {4}.
  \end{equation}
  \begin{equation}
    \overline{\left(\underline{u}\right)}(x_i)
    = \frac{1}{2 h}
      \left(
        \underline{u} \mathcal{E}_{i - 1} + \underline{u} \mathcal{E}_i
      \right)
    = \frac{1}{2 h}
      \frac{h}{2}
      ((u(x_{i - 1}) + u(x_i)) + (u(x_i) + u(x_{i + 1})))
    = \frac{u(x_{i - 1}) + 2 u(x_i) + u(x_{i + 1})}{4}.
  \end{equation}
  In both cases of composition of embedding and approximation the final result
  is the identity operator when $\pi^1$ (respectively $u$) is linear with
  respect to the index $i$.
\end{example}

\begin{discussion}
  Let me summarize the operations relating cochains and embedding.
  We will use notation coming from dependent type theory for functions whose
  codomain depends on the domain.
  Namely, if
    $X$ is a type (set),
    $\{Y(x)\}_{x \in X}$ is a family of sets and
    $\{f(x) \in Y(x)\}_{x \in X}$,
  we will write
  \begin{equation}
    f \colon \prod_{x \in X} Y(x).
  \end{equation}
  Let
    $d \in \N$,
    $K$ be a quasi-cubical flat mesh of dimension $d$,
    $X$ be the manifold it encompasses.
  We define the following data:
  \begin{itemize}
    \item
      $n \colon K_0 \to \N$ denotes the number of node neighbors of a $0$-cell;
    \item
      $\widehat{\phantom{T}} \colon C^1 K \to
        \displaystyle \prod_{x_0 \in K_0} \R^{n(x_0)}$
      denotes the neighbor representation of a $1$-cochain,
      $\widehat{\phantom{T}}\ [1]$;
    \item
      $\displaystyle
        \mathcal{L} \colon \prod_{x_0 \in K_0} M_{n(x_0), d}(\R)$
      denotes the node neighbors matrix,
      $\mathcal{L}\ [L]$;
    \item
      $\displaystyle
        \star \colon \prod_{(m, n) \in \N^2} M_{m, n}(\R) \to M_{n, m}(\R)$
      denotes the Moore-Penrose inverse of a rectangular matrix,
      ($\star$ reverses physical dimensions);
    \item
      $\overline{\phantom{T}} \colon C^1 K \to \Hom_\R(C^0 K, \R^d)$
      denotes the approximation of a $1$-cochain as a Euclidean vector-valued
      $0$-cochain,
      \begin{equation}
        \overline{\pi^1} c_0 :=
        (\mathcal{L}_{c_0})^\star \cdot (\widehat{\pi^1})_{c_0},
      \end{equation}
      $\overline{\phantom{T}}\ [L^{-1}]$;
    \item
      $\underline{\phantom{T}} \colon \chi X \to C^1 K$
      denotes the discretization of a continuum vector field as a $1$-cochain,
      $\underline{\phantom{T}}\ [L]$;
  \end{itemize}
\end{discussion}


\section{Continuous heat transport}
\label{section:continuous_heat_transport}
\begin{discussion}
  In this section we will consider the heat transport phenomenon in both
  transient and steady-state form.
  Our formulation will be represented in the language of differential forms
  because they better represent the meaning of physical quantities.
  Various (weak) reformulations will be presented -- those reformulations will
  give us hints on how to construct purely discrete formulations.
\end{discussion}

\begin{discussion}
  \label{idec/heat_transport/continuous/model_with_differential_forms-discussion}
  Let:
  \begin{itemize}
    \item
      $D$ be a positive integer (space dimension);
    \item
      $X$ be an open region in $\R^D$ (the space region);
    \item
      $t_0 [T] \in \R$ be the initial time;
    \item
      $I := [t_0, \infty)$;
  \end{itemize}
  The main physical quantities in our model are:
  \begin{itemize}
    \item
      temperature $u^0 [\Theta] \colon I \to \Omega^0 X$:
      for any moment $t \in I$ and any point $x \in X$,
      \begin{equation}
        \text{``temperature $[\Theta]$ on $x$ at time $t$''}
        = u^0(t)(x) := u^0(t, x);
      \end{equation}
    \item
      heat energy $Q^D [E] \colon I \to \Omega^D X$:
      for any moment $t \in I$ and any volume $V_D \subseteq X$,
      \begin{equation}
        \text{``total heat energy $[E]$ of the system on $V$ at time $t$''}
        = \int_{V_D} Q^D(t);
      \end{equation}
    \item
      heat flow $q^{D - 1} [E T^{-1}] \colon I \to \Omega^{D - 1} X$:
      for any time interval $[t_1, t_2] \subset I$
      and any hypersurface $S_{D - 1} \subset X$,
      \begin{equation}
        \text{``total flow $[E]$ through $S_{D - 1}$ in $[t_1, t_2]$''}
        = \int_{t_1}^{t_2}\left(\int_{S_{D - 1}} q^{D - 1}(t)\right)\, d t.
      \end{equation}
      (Here we assume that $S_{D - 1}$ is oriented.
       Let $U_D$ and $V_D$ be adjacent regions having $S_{D - 1}$ as a common
       boundary, such that $\varepsilon(U_D, S_{D - 1}) = -1$,
       $\varepsilon(V_D, S_{D - 1}) = 1$.
       Then the above equation measures the total flow from $U_D$ to $V_D$.)
  \end{itemize}
  We will also need the dual variables of heat energy, temperature and flow.
  \begin{itemize}
    \item
      temperature-volume $\tilde{u}^D [\Theta L^D] \colon I \to \Omega^D X$
      defined by
      \begin{equation}
        \tilde{u}^D := \star_0 u^0
      \end{equation}
      (althiugh using non-zero based temperature scale might make $\star_0$ not
      well defined, this will not cause problems as we will always take
      temperature differences when substituting in equations);
    \item
      heat energy density $\tilde{Q}^0 [E L^{-D}] \colon I \to \Omega^0 X$
      defined by
      \begin{equation}
        \tilde{Q}^0 := \star_D Q^D;
      \end{equation}
    \item
      orthogonal (edge) flux
      $\tilde{q}^1 [E T L^{2 - D}] \colon I \to \Omega^1 X$
      defined by
      \begin{equation}
        \tilde{q} := \star_{D - 1} q;
      \end{equation}
  \end{itemize}
  The governing laws are formulated as follows.
  \begin{itemize}
    \item
      Let $f^D [E T^{-1}] \colon I \to \Omega^D X$ be an external heat source:
      for any time interval $[t_1, t_2] \subset I$
      and any volume $V_D \subseteq X$,
      \begin{equation}
        \text{``total net heat production $[E]$ in $V_D$ in $[t_1, t_2]$''}
        = \int_{t_1}^{t_2} \left(\int_{V_D} f^D(t) \right)\, d t.
      \end{equation}
      \textbf{Conservation of heat energy} is given by the following relation:
      for any time interval $[t_1, t_2] \subset I$
      and any volume $V_D \subseteq X$,
      \begin{equation}
        \begin{split}
        \text{``heat difference on $V_D$ between moments $t_2$ and $t_1$''}
        & =
          \text{``heat inflow through the boundary of $V_D$
          in $[t_1, t_2]$''} \\
        & +
          \text{``heat production inside $V_D$ in $[t_1, t_2]$''}.
        \end{split}
      \end{equation}
      In symbolic terms, the last equation is written as
      \begin{equation}
        \int_{V_D} (Q^D(t_2) - Q^d(t_1))
        = \int_{t_1}^{t_2}
          \left(\int_{\partial V_D} q^{D - 1}(t) \right)\, d t
        + \int_{t_1}^{t_2} \left(\int_{V_D} f^D(t) \right)\, d t.
      \end{equation}
      Using Stokes' theorem twice, we get the equation
      \begin{equation}
        \int_{t_1}^{t_2}
          \left(\int_{V_D} \frac{\partial Q^d}{\partial t}\right)\, d t =
          \int_{t_1}^{t_2} \left(\int_{V_D} d_{D - 1} q^{D - 1} \right)\, d t
        + \int_{t_1}^{t_2} \left(\int_{V_D} f^D \right)\, d t.
      \end{equation}
      Since the time interval $[t_1, t_2]$ and the volume $V_D$ are arbitrary,
      we can drop integrals and arrive at the differential form
      \begin{equation}
        \frac{\partial Q^D}{\partial t} = d_{D - 1} q^{D - 1} + f^D.
      \end{equation}
    \item
      Let
        $u_0 [\Theta] \in \Omega^0 X$ be the initial temperature.
      The \textbf{initial condition} is the prescribed initial temperature:
      \begin{equation}
        u^0(t_0) = u_0.
      \end{equation}
    \item
      Let $\pi_0 [E L^{-D} \Theta^{-1}] \colon \Omega^0 X \to \Omega^0 X$
      be the volumetric heat capacity.
      The \textbf{relation between temperature change and heat energy change}
      is given by
      \begin{equation}
        \frac{\partial Q^D}{\partial t}
        = \star_0 \left(\frac{\partial \tilde{Q}^0}{\partial t}\right).
        = \star_0 \left(\pi_0 \frac{\partial u^0}{\partial t}\right).
      \end{equation}
    \item
      Consider two adjacent volumes $U_D$ and $V_D$
      with a common surface $S_{D - 1}$, such that
      $\varepsilon(U_D, S_{D - 1}) = -1$ and
      $\varepsilon(V_D, S_{D - 1}) = 1$.
      According to the second law of thermodynamics, heat flows from regions of
      higher temperature to regions of lower temperatures.
      Therefore, the net flow through $S_{D - 1}$ is in the negative direction
      of the temperature difference between $U_D$ and $V_D$.

      Let
      $\kappa_{D - 1} [E L^{2 - D} T^{-1} \Theta^{-1}]
      \colon \Omega^{D - 1} X \to \Omega^{D - 1} X$
      be the thermal conductivity.
      The \textbf{Fourier's constitutive relation}
      quantifies the above relation by using $\kappa_{D - 1}$
      as a proportionality factor:
      \begin{equation}
        q^{D - 1}
        = - \kappa_{D - 1} \delta_D^\star \tilde{u}^D
        = - \kappa_{D - 1} \delta_D^\star \star_0 u^0
        = (-1)^{D - 1} \kappa_{D - 1} \star_1 \delta_0 u^0
        = (-1)^{D - 1} \star_1 \tilde{\kappa}_1 \delta_0 u,
      \end{equation}
      where we have denoted
      \begin{equation}
        \tilde{\kappa}_1
        :=\star_1^{-1} \kappa_2 \star_1 [E L^{2 - D} T^{-1} \Theta^{-1}]
        \colon \Omega^1 X \to \Omega^1 X.
      \end{equation}
  \end{itemize}
  We complete our model with boundary conditions.
  Let $\Gamma_D, \Gamma_N$ form a partition of $\partial X$
  into Dirichlet and Neumann boundary.
  \begin{itemize}
    \item
      Let $g_D^0 [\Theta] \colon I \to \Omega^0 \Gamma_D$
      be the prescribed temperature on the Dirichlet boundary $\Gamma_D$.
      The \textbf{Dirichlet boundary condition} is given by
      \begin{equation}
        \tr_{\Gamma_D, 0} u^0 := \restrict{u}{\Gamma_D} = g_D^0.
      \end{equation}
    \item
      Let $g_N^{D - 1} [E T^{-1}] \colon I \to \Omega^2 \Gamma_N$
      be the prescribed flow on the Neumann boundary $\Gamma_N$.
      The \textbf{Neumann boundary condition} is given by
      \begin{equation}
        \tr_{\Gamma_N, D - 1} q^{D - 1} = g_N^{D - 1}.
      \end{equation}
  \end{itemize}
\end{discussion}

\begin{formulation}
  \label{idec/continuous_heat_transport/strong-formulation}
  The \textbf{strong differential formulation for heat transport}is obtained by
  representing heat energy and heat flow in terms of temperature.

  Let:
  \begin{itemize}
    \item
      $X$ be a bounded open region in $3$-dimensional Euclidean space;
    \item
      $t_0 \in \R$ be the initial time, $t_0\ [T]$;
    \item
      $I := [t_0, \infty)$;
    \item
      $u_0 \in \Omega^0 X$ be the initial temperature, $u_0\ [\Theta]$;
    \item
      $f \in \Omega^0 X$ be the external heat source, $f\ [E L^{-3} T^{-1}]$;
    \item
      $\Gamma_D, \Gamma_N$ form a partition of $(\partial K)_0$;
    \item
      $g_D \colon I \to \Omega^0 \Gamma_D$
      be the Dirichlet boundary condition, $g_D\ [\Theta]$;
    \item
      $g_N \colon I \to \Omega^0 \Gamma_N$
      be the Neumann boundary condition, $g_N\ [E L^{-2} T^{-1}]$;
    \item
      $\pi_0 \colon \Omega^0 M \to \Omega^0 M$ be the volumetric heat capacity,
      $\pi_0\ [E L^{-3} \Theta^{-1}]$;
    \item
      $\pi_1 \colon \Omega^1 X \to \Omega^1 X$ be the thermal conductivity,
      $\pi_1\ [E L^{-1} T^{-1} \Theta^{-1}]$.
  \end{itemize}
  We are solving the following problem.
  \begin{equation}
    \begin{split}
      & \text{Find $u \colon I \to \Omega^0 X$, $u\ [\Theta]$, such that} \\
      &
      \begin{cases}
        \frac{\partial (\pi_0 u)}{\partial t} =
        - (d_1^\star \circ \pi_1 \circ d_0) u + f
        & (\text{point-wise conservation of heat energy},
          \ [E L^{-3} T^{-1}]), \\
%
        u(t_0, \cdot) = u_0
        & (\text{initial condition},\ [\Theta]), \\
%
        \restrict{u}{\Gamma_D} = g_D
        & (\text{Dirichlet boundary condition},\ [\Theta]), \\
%
        (\tr_{\Gamma_N} \circ \star_1\circ \pi_1 \circ d_0) u
        = \star^{\Gamma_N}_0 g_N
        & (\text{Neumann boundary condition},\ [E T^{-1}]).
      \end{cases}
    \end{split}
  \end{equation}
\end{formulation}

\begin{discussion}
  Using the variables from
  \Cref{idec/continuous_heat_transport/strong-formulation}
  we are going to introduce an alternative (primal weak) formulation.
  Let $v \in \Ker \tr_{\Gamma_D}$ be a test function
  (later on the differentiablity assumptions on $v$ can be weakened).
  Multiply the conservation of energy with $v$ and integrate over $X$:
  \begin{equation}
    \begin{split}
      \int_X v \wedge \frac{\partial Q}{\partial t}
      & = \int_X (v \wedge d q) + \int_X (v \wedge f) \\
      & = \int_{\partial X} \tr_{\partial X} (v \wedge q)
        - \int_X (d v \wedge q)
        + \int_X (v \wedge f) \\
      & = \int_{\Gamma_D} (\tr_{\Gamma_D} v \wedge \tr_{\Gamma_D} q)
        + \int_{\Gamma_N} (\tr_{\Gamma_N} v \wedge \tr_{\Gamma_N} q)
        - \int_X (d v \wedge \star_1 \pi_1 d_0 u)
        + \int_X (v \wedge \star_0 \tilde{f}) \\
      & = 0
        + \int_{\Gamma_N}
          (\tr_{\Gamma_N} v \wedge \star^{(\Gamma_N)}_0 \widetilde{g_N})
        - \int_X (d v \wedge \star_1 \pi_1 d_0 u)
        + \int_X (v \wedge \star_0 \tilde{f}) \\
      & = \inner{\tr_{\Gamma_N} v}{\widetilde{g_N}}_{(\Gamma_N)}
        - \inner{d v}{\pi_1 d_0 u}_{(X)}
        + \inner{v}{\tilde{f}}_{(X)}.
    \end{split}
  \end{equation}
  We also have:
  \begin{equation}
    \int_X v \wedge \frac{\partial Q}{\partial t}
    = \frac{\partial}{\partial t} \left(\int_X v \wedge Q\right)
    = \frac{\partial}{\partial t}
      \left(\int_X v \wedge \star_3 (\pi_0 u)\right)
    = \frac{\partial}{\partial t} \inner{v}{\pi_0 u}_X.
  \end{equation}
  Hence, we get the following (primal weak) formulation.
\end{discussion}
\begin{formulation}
  [Primal weak formulation for the transient continuous heat equation
    with differential forms]
  Let:
  \begin{itemize}
    \item
      $X$ be an open region in $3$D, representing a material body;
    \item
      $t_0 \in \R$ be the initial time;
    \item
      $I = [t_0, \infty)$ be the time-interval where the process occurs;
    \item
      $u_0 [\Theta] \in \Omega^0 X$ be the initial temperature;
    \item
      $\pi_1 [E L^{-1} T^{-1} \Theta^{-1}] \colon \Omega^1 X \to \Omega^1 X$
      be the thermal conductivity of the material;
    \item
      $\pi_3 [E \Theta^{-1}] \colon \Omega^3 X \to \Omega^0 X$
      be the heat capacity of the material;
    \item
      $\partial X = \Gamma_D \cup \Gamma_N$ be the partition of the boundary of
      $X$ into Dirichlet ($\Gamma_D$) and Neumann ($\Gamma_N$) regions;
    \item
      $g_D [\Theta] \colon I \to \Omega^0 \Gamma_D$
      be the prescribed temperature on the Dirichlet boundary;
    \item
      $g_N [E T^{-1}] \colon I \to \Omega^2 \Gamma_N$
      be the prescribed flow on the Neumann boundary.
  \end{itemize}
  Our unknowns is temperature $u [\Theta] \colon I \to \Omega^0 X$.
  We are solving the following problem for $u$:
  \begin{subequations}
    \begin{alignat}{4}
      & \forall v [\Theta] \in \Ker \tr_{\Gamma_D}, \quad
      && \inner{v}{\pi_0 \frac{\partial u} {\partial t}}_{X, 0}
        + \inner{d_0 v}{\pi_1 d_0 u}_{X, 0}
      && = \int_{\Gamma_N} (\tr_{\Gamma_N} v \wedge g_N)
        + \int_X (v \wedge f) \qquad
      && [E T^{-1} \Theta], \\
      &
      && \tr_{\Gamma_D, 0} u
      && = g_G \qquad
      && [\Theta], \\
      &
      && u(t_0)
      && = u_0 \qquad
      && [\Theta].
    \end{alignat}
  \end{subequations}
\end{formulation}
\begin{formulation}
  [Primal weak formulation for the steady-state continuous heat equation
    with differential forms]
  Let:
  \begin{itemize}
    \item
      $X$ be an open region in $3$D, representing a material body;
    \item
      $\pi_1 [E L^{-1} T^{-1} \Theta^{-1}] \colon \Omega^1 X \to \Omega^1 X$
      be the thermal conductivity of the material;
    \item
      $\partial X = \Gamma_D \cup \Gamma_N$ be the partition of the boundary of
      $X$ into Dirichlet ($\Gamma_D$) and Neumann ($\Gamma_N$) regions;
    \item
      $g_D [\Theta] \colon I \to \Omega^0 \Gamma_D$
      be the prescribed temperature on the Dirichlet boundary;
    \item
      $g_N [E T^{-1}] \colon I \to \Omega^2 \Gamma_N$
      be the prescribed flow on the Neumann boundary.
  \end{itemize}
  Our unknowns is temperature $u [\Theta] \in \Omega^0 X$.
  We are solving the following problem for $u$:
  \begin{subequations}
    \begin{alignat}{4}
      & \forall v \in \Ker \tr_{\Gamma_D}, \quad
      && \inner{d_0 v}{\pi_1 d_0 u}_{(X)}
      && = \int_{\Gamma_N} (\tr_{\Gamma_N} v \wedge g_N)
        + \int_X (v \wedge f) \qquad
      && [E T^{-1} \Theta], \\
%
      &
      && \tr_{\Gamma_D, 0} u
      && = g_G \qquad
      && [\Theta].
    \end{alignat}
  \end{subequations}
\end{formulation}

\begin{discussion}
  We are going to formulate the \textbf{mixed weak formulation for continuous
  heat transport with differential forms}.
  Consider the model
  \Cref{idec/heat_transport/continuous/model_with_differential_forms-discussion}
  with the same domains and variable names.
  Let $v \in \Omega^0 X$ and $r \in \Ker \tr_{\Gamma_N, 2}$ be test functions.
  Define
  \begin{equation}
    \pi_2 :=
    \star_2^{-1} \circ \pi_1 \circ \star_2 \colon \Omega^2 X \to \Omega^2 X.
  \end{equation}
  Then
  \begin{equation}
    \pi_2^{-1} q = \star_2^{-1} (d_0 u),
  \end{equation}
  and therefore
  \begin{equation}
    \begin{split}
      \inner{r}{\pi_2^{-1} q}
      & = \int_X (r \wedge \star_2 (\pi_2^{-1} q)) \\
      & = \int_X (r \wedge d_0 u) \\
      & = \int_{\partial X} (\tr_{X, 2} r \wedge \tr_{X, 0} u)
        - \int_{X} (d_2 r \wedge u) \\
      & = \int_{\Gamma_D} (\tr_{\Gamma_D, 2} r \wedge g_D)
        - \int_{X} (d_2 r \wedge u).
    \end{split}
  \end{equation}
  Multiplying the balance equation with $v$ and integrating over $X$ gives
  \begin{equation}
    \frac{d}{d t} \int_X (v \wedge \star_0 (\pi_0 u))
    = \int_X v \wedge d_2 q + \int_X (v \wedge f).
  \end{equation}
  We get the following formulation.
  \begin{equation}
    \begin{split}
      & \text{Find $u [\Theta] \colon I \to \Omega^0 X$ and
        $q [E T^{-1}] \colon I \to \Omega^2 X$ such that} \\
      &
      \begin{cases}
        \forall r [E T^{-1}] \in \Ker \tr_{\Gamma_N, 2},\
          \inner{r}{\pi_2^{-1} q}_{X, 2} + \int_{X} (u \wedge d_2 r)
          = \int_{\Gamma_D} (\tr_{\Gamma_D, 2} r \wedge g_D)
        & [E T^{-1} \Theta], \\
        \forall v [\Theta] \in \Omega^0 X,\
          \int_X (v \wedge d_2 q)
          - \inner{v}{\pi_0 \frac{\partial u}{\partial t}}_{X, 0}
          = - \int_X (v \wedge f)
        & [E T^{-1} \Theta], \\
        u(t_0, \cdot) = u_0
        & [\Theta], \\
        \tr_{\Gamma_N, 2} q = g_N
        & [E T^{-1}].
      \end{cases}
    \end{split}
  \end{equation}
  The steady-state model reads as follows.
  \begin{equation}
    \begin{split}
      & \text{Find $u [\Theta] \in \Omega^0 X$ and
        $q [E T^{-1}] \in \Omega^2 X$ such that} \\
      &
      \begin{cases}
        \forall r [E T^{-1}] \in \Ker \tr_{\Gamma_N, 2},\
          \inner{r}{\pi_2^{-1} q}_{X, 2} + \int_{X} (u \wedge d_2 r)
          = \int_{\Gamma_D} (\tr_{\Gamma_D, 2} r \wedge g_D)
        & [E T^{-1} \Theta], \\
        \forall v [\Theta] \in \Omega^0 X,\
          \int_X (v \wedge d_2 q) = - \int_X (v \wedge f)
        & [E T^{-1} \Theta], \\
        \tr_{\Gamma_N, 2} q = g_N
        & [E T^{-1}].
      \end{cases}
    \end{split}
  \end{equation}
\end{discussion}


\section{Discrete heat transport}
\label{section:discrete_heat_transport}
\begin{notation}
  Let
    $S$ be a set,
    $T$ be a subset of $S$,
    $V$ be a real vector space,
    $u \in \Hom_\R({\rm Free}_\R(S), V)$.
  We will denote by
  \begin{equation}
    \restrict{u}{T} \in \Hom_\R({\rm Free}_\R(T), V)
  \end{equation}
  the map defined in the same way of $u$ but acting on formal linear
  combinations of the elements in $T$.
\end{notation}

\begin{notation}
  Let
    $D \in \N$,
    $K$ be a flat mesh of dimension $D$,
    $c_0 \in C_0(\partial K)$ with corresponding point $x \in \R^D$.
  By ${\bf n}_{c_0}$ we will denote the exterior unit normal at $x$ to $K$.
  When $c_{0}$ has more than one non-parallel adjacent hyperfaces
  (for instance, in $3$D, it can lie on an edge or at a corner), we will take
  some average of the normals to those faces.
  The easiest one is to sum all exterior unit normals and divide by the length
  of the sum.
  This is the approach taken in the software implementation.

  When $\Gamma \subseteq (\partial K)_0$, we will understand ${\bf n}$ as
  a function from $\Gamma$ to $\R^D$ or as a linear map in
  $\Hom({\rm Free}_\R(\Gamma), \R^D)$.
\end{notation}

\begin{discussion}
  We are going to state the governing laws for the discrete heat transport
  phenomenon.

  Let:
  \begin{itemize}
    \item $M$ be a manifold-like flat mesh of dimension $3$;
    \item $K$ be the Forman subdivision of $M$;
    \item $t_0 \in \R$ be the initial time, $I = [t_0, \infty)$.
  \end{itemize}
  Physical quantities in our model are:
  \begin{itemize}
    \item
      temperature $u \colon I \to C^0 K$ of physical dimension $[\Theta]$;
    \item
      heat energy density $Q \colon I \to C^0 K$ of physical dimension
      $[E L^{-3}]$;
    \item
      heat flux $q \colon I \to C^1 K$ of physical dimension $[E L^{-1} T^{-1}]$.
  \end{itemize}
  The governing laws are the following.
  \begin{itemize}
    \item
      Let
        $K' := K \setminus \partial K$ be the interior of $K$,
        $f \in C^0 K'$ be the external heat source of physical dimension
          $[E L^{-3} T^{-1}]$.
      \textbf{Conservation of heat energy} is modeled by the equation
      \begin{equation}
        \restrict{\frac{\partial Q}{\partial t}}{K'_0} =
        \restrict{(\delta_1^\star q)}{K'_0} + f.
      \end{equation}
    \item
      Let
        $u_0 \in C^0 K$ be the initial temperature, $u_0\ [\Theta]$.
      The \textbf{initial condition} is prescribed initial temperature:
      \begin{equation}
        u(t_0, \cdot) = u_0.
      \end{equation}
    \item
      Let $\pi_0 \colon C^0 K \to C^0 K$
      of physical dimension $[E L^{-3} \Theta^{-1}]$
      be a material property of the nodes of $K$,
      playing the role of volumetric heat capacity
      (its matrix in the standard basis is diagonal).
      The \textbf{relation between temperature and heat energy} is given by
      \begin{equation}
        Q = \pi_0 u.
      \end{equation}
    \item
      Let $\pi_1 \colon C^1 K \to C^1 K$
      of physical dimension $[E L^{-1} T^{-1} \Theta^{-1}]$
      be a material property of the edges of $K$,
      playing the role of thermal conductivity
      (its matrix in the standard basis is diagonal).
      The \textbf{Fourier's constitutive relation} is given by
      \begin{equation}
        q = \pi_1 (\delta_0 u).
      \end{equation}
  \end{itemize}
  We complete our model with boundary conditions.
  Let $\Gamma_D, \Gamma_N$ form a partition of $(\partial K)_0$
  into Dirichlet and Neumann boundary.
  \begin{itemize}
    \item
      Let $g_D \colon [t_0, \infty) \to ({\rm Free}_\R(\Gamma_D))^*$
      of physical dimension $[\Theta]$
      be the prescribed temperature on the Dirichlet boundary $\Gamma_D$.
      The \textbf{Dirichlet boundary condition} is given by
      \begin{equation}
        \restrict{u}{\Gamma_D} = g_D.
      \end{equation}
    \item
      Let
        ${\bf n} \colon \Gamma_N \to \R^d$
          be the dimensionless generalized exterior unit normal,
        $g_N \colon [t_0, \infty) \to ({\rm Free}_\R(\Gamma_N))^*$
          of physical dimension $[E L^{-1} T^{-1}]$
          be the prescribed flux on the Neumann boundary $\Gamma_N$.
      The \textbf{Neumann boundary condition} is given by
      \begin{equation}
        \restrict{\overline{q}}{\Gamma_N} \cdot {\bf n} = g_N.
      \end{equation}
  \end{itemize}
\end{discussion}

\begin{formulation}
  \label{idec/discrete_heat_transport/phenomenon-formulation}
  By substituting $q$ with $\pi_1 \delta_0 u$ and $Q$ with $\pi_0 u$,
  we arrive at the following formulation with only one unknown
  (the temperature $u$).
  Let:
  \begin{itemize}
    \item
      $d \in \N$;
    \item
      $M$ be a manifold-like flat mesh of dimension $d$;
    \item
      $K$ be the Forman subdivision of $M$;
    \item
      $K' := K \setminus \partial K$ be the interior of $K$;
    \item
      $t_0 \in \R$ be the initial time, $t_0\ [T]$;
    \item
      $u_0 \in C^0 K$ be the initial temperature, $u_0\ [\Theta]$;
    \item
      $f \in C^0 K'$ be the external heat source, $f\ [E L^{-3} T^{-1}]$;
    \item
      $\Gamma_D, \Gamma_N$ form a partition of $(\partial K)_0$;
    \item
      ${\bf n} \colon \Gamma_N \to \R^d$
      be the generalized exterior unit normal, ${\bf n}\ [1]$;
    \item
      $g_D \colon [t_0, \infty) \to ({\rm Free}_\R(\Gamma_D))^*$
      be the Dirichlet boundary condition, $g_D\ [\Theta]$;
    \item
      $g_N \colon [t_0, \infty) \to ({\rm Free}_\R(\Gamma_N))^*$
      be the Neumann boundary condition, $g_N\ [E L^{-1} T^{-1}]$;
    \item
      $\pi_0 \colon C^0 K \to C^0 K$ be a material property of the nodes of $K$
      (its matrix in the standard basis is diagonal),
      $\pi_0\ [E L^{-3} \Theta^{-1}]$;
    \item
      $\pi_1 \colon C^1 K \to C^1 K$ be a material property of the edges of $K$
      (its matrix in the standard basis is diagonal),
      $\pi_1\ [E L^{-1} T^{-1} \Theta^{-1}]$.
  \end{itemize}
  We are solving the following problem.
  \begin{equation}
    \begin{split}
      & \text{Find $u \in [t_0, \infty) \to C^0 K$, $u\ [\Theta]$, such that} \\
      &
      \begin{cases}
        \restrict{\frac{\partial (\pi_0 u)}{\partial t}}{K'_0} =
        \restrict{((\delta_1^\star \circ \pi_1 \circ \delta_0) u)}{K'_0} + f
        & (\text{conservation of heat energy},\ [E L^{-3} T^{-1}]), \\
%
        u(t_0, \cdot) = u_0
        & (\text{initial condition},\ [\Theta]), \\
%
        \restrict{u}{\Gamma_D} = g_D
        & (\text{Dirichlet boundary condition},\ [\Theta]), \\
%
        \restrict{\overline{(\pi_1 \circ \delta_0) u}}{\Gamma_N}
        \cdot {\bf n} = g_N
        & (\text{Neumann boundary condition},\ [E L^{-1} T^{-1}]).
      \end{cases}
    \end{split}
  \end{equation}
\end{formulation}

\begin{discussion}
  Consider \Cref{idec/discrete_heat_transport/phenomenon-formulation}.
  This formulation is discrete in space but continuous in time.
  In order to numerically solve it we need to discretize the time variable.
  We will use the trapezoidal (Crank-Nicolson) method.

  Let
    $\tau \in \R^+$ of physical dimension $[T]$ be the time step,
    $i \in \N$,
    $t_i := t_0 + i \tau$,
    $y_i := u(t_i, \cdot) \in C^0 K$ of physical dimension $[\Theta]$,
    $\rho_0 := \delta_1^\star \circ \kappa_1 \circ \delta_0$
      of physical dimension $[E L^{-3} T^{-1} \Theta^{-1}]$.
  Integrating the conservation of heat energy in $[t_i, t_{i + 1}]$, we get
  \begin{equation}
    \restrict{(\pi_0 y_{i + 1} - \pi_0 y_i)}{K'_0}
    =   \int_{t_i}^{t_{i + 1}} \restrict{(\rho_0 u(t, \cdot))}{K'_0}\, d t
      + \int_{t_i}^{t_{i + 1}} f\, d t
    \approx
    \frac{\tau}{2} \restrict{(\rho_0 y_i + \rho_0 y_{i + 1})}{K'_0} + \tau f.
  \end{equation}
  Rearranging, we get the discretized equation
  \begin{equation}
    \restrict{((\pi_0 - \frac{\tau}{2} \rho_0) y_{i + 1})}{K'_0}
    = \restrict{((\pi_0 + \frac{\tau}{2} \rho_0) y_i)}{K'_0} + \tau f.
  \end{equation}
  Define the $\dim C^0 K \times \dim C^0 K$ matrices of physical dimension
  $[E L^{-3} \Theta^{-1}]$
  \begin{equation}
    A := \pi_0 - \frac{\tau}{2} \rho_0,\ B := \pi_0 + \frac{\tau}{2} \rho_0.
  \end{equation}
  The discretized in time (space is already discrete) formulation reads as
  follows.
  \begin{equation}
    \begin{split}
      & \text{Find $y \colon \N \to C^0 K$ such that} \\
      &
      \begin{cases}
        y_0 = u_0 \\
        \text{for any $i \in \N^+$, $y_i$ solves the system} \\
        \begin{cases}
          \restrict{(A y_i)}{K'_0} = \restrict{(B y_{i - 1})}{K'_0} + \tau f, \\
          \restrict{y_i}{\Gamma_D} = g_D, \\
          \restrict{\overline{(\kappa_1 \circ \delta_0) y_i}}{\Gamma_N}
            \cdot {\bf n} = g_N.
        \end{cases}
      \end{cases}
    \end{split}
  \end{equation}
  Of course, in practice we solve it for a finite number of time steps.
  Usually, we compare tho adjacent solutions $y_i$ and $y_{i + 1}$ and stop when
  the relative error is sufficiently small (we have reached a steady state).
  % Moreover, if $\pi_0$ and $\kappa_1$ are constants, and $K$ has mesh size
  % $h$, then $h^2 / \tau$ should be close to the thermal diffusivity
  % $\kappa_1 / \pi_0$ (both of physical dimension $[L^2 T^{-1}]$).
\end{discussion}

\begin{formulation}
  Let
  \begin{itemize}
    \item
      $M$ be a manifold-like flat mesh of dimension $3$;
    \item
      $K$ be the Forman subdivision of $M$;
    \item
      $K' := K \setminus \partial K$ be the interior of $K$;
    \item
      $t_0 \in \R$ be the initial time, $t_0\ [T]$;
    \item
      $I := [t_0, \infty)$
    \item
      $u_0 \in C^0 K$ be the initial temperature, $u_0\ [\Theta]$;
    \item
      $\tilde{f} \in C^0 K'$ be the density form of the heat source,
      $\tilde{f}\ [E L^{-3} T^{-1}]$;
    \item
      $\Gamma_D, \Gamma_N$ form a partition of $(\partial K)$;
    \item
      $g_D \colon I \to C^0 \Gamma_D$
      be the Dirichlet boundary condition, $g_D\ [\Theta]$;
    \item
      $\widetilde{g_N} \colon I \to C^0 \Gamma_N$
      be the density form of the Neumann boundary condition,
      $\widetilde{g_N}\ [E L^{-2} T^{-1}]$;
    \item
      $\pi_0 \colon C^0 K \to C^0 K$ be the volumetric heat capacity
      (its matrix in the standard basis is diagonal),
      $\pi_0\ [E L^{-3} \Theta^{-1}]$;
    \item
      $\pi_1 \colon C^1 K \to C^1 K$ be the thermal conductivity
      (its matrix in the standard basis is diagonal),
      $\pi_1\ [E L^{-1} T^{-1} \Theta^{-1}]$.
  \end{itemize}
  We are solving the following problem:
  \begin{equation}
    \begin{split}
      & \text{Find $u \colon I \to C^0 K$, $u\ [\Theta]$, such that} \\
      &
      \begin{cases}
        \forall v \in \Ker \tr_{\Gamma_D},\
          \frac{\partial}{\partial t} \inner{v}{\pi_0 u}_{X, 0} =
            \inner{\tr_{\Gamma_N} v}{\widetilde{g_N}}_{\Gamma_N, 0}
          - \inner{\delta_0 v}{\pi_1 \delta_0 u}_{X, 1}
          + \inner{v}{\tilde{f}}_{X, 0} 
        & [E T^{-1} \Theta], \\
        u(t_0, \cdot) = u_0 & [\Theta], \\
        \tr_{\Gamma_D} u = g_D & [\Theta].
      \end{cases}
    \end{split}
  \end{equation}
  The steady-state model reads as follows:
  \begin{equation}
    \begin{split}
      & \text{Find $u \in C^0 K$, $u\ [\Theta]$, such that} \\
      &
      \begin{cases}
        \forall v \in \Ker \tr_{\Gamma_D},\
          \inner{\delta_0 v}{\pi_1 \delta_0 u}_{X, 1}
          = \inner{\tr_{\Gamma_N} v}{\widetilde{g_N}}_{\Gamma_N, 0}
          + \inner{v}{\tilde{f}}_{X, 0}
      & [E T^{-1} \Theta], \\
        \tr_{\Gamma_D} u = g_D & [\Theta].
      \end{cases}
    \end{split}
  \end{equation}
  In both cases in the post-processing phase we have:
  \begin{subequations}
    \begin{alignat}{2}
      & \tilde{q} && := \pi_1 \delta_0 u, \\
      & q && \approx \star_1 \tilde{q}, \\
      & \tilde{Q} && := \pi_0 u, \\
      & Q && \approx \star_0 \tilde{Q}.
    \end{alignat}
  \end{subequations}
  If we are given $g_N$ or $f$, then we define
  \begin{subequations}
    \begin{alignat}{2}
      & \tilde{f} && := \star_3 f, \\
      & \tilde{g_N} && := \star_2^{(\Gamma_N)} g_N.
    \end{alignat}
  \end{subequations}
\end{formulation}

\begin{formulation}
  \label{cmc/diffusion/discrete/transient/mixed_weak-formulation}
  [Mixed weak formulation for the discrete transient heat equation]
  The following formulation is a discrete version of
  \Cref{cmc/diffusion/continuous/transient/mixed_weak-formulation}.
  Let:
  \begin{itemize}
    \item
      $d$ be a positive integer (space dimension);
    \item
      $K$ be an oriented quasi-cubical \hyperref[cmc:mesh:definition]{mesh} of
      dimension $d$ representing the material body;
    \item
      $[K]$ be the fundamental class of $K$;
    \item
      $t_0 \in \R$ be the initial time;
    \item
      $I = [t_0, \infty)$ be the time-interval where the process occurs;
    \item
      $f [E T^{-1}] \colon I \to C^d K$ be the internal production rate;
    \item
      $u_0 [\Theta] \in C^0 K$ be the initial temperature;
    \item
      $\kappa [E L^{2 - D} T^{-1} \Theta^{-1}]
      \colon C^{D - 1} K \to C^{D - 1} K$
      be the thermal conductivity of the material;
    \item
      $\pi [E L^{-D} \Theta^{-1}] \colon C^d K \to C^d K$
      be the heat capacity of the material;
    \item
      $\partial K = \Gamma_D \cup \Gamma_N$ be the partition of the boundary of
      $K$ into Dirichlet ($\Gamma_D$) and Neumann ($\Gamma_N$) regions;
    \item
      $[\Gamma_D]$ be the fundamental class of $\Gamma_D$, where $\Gamma_D$
      has the boundary orientation induced from $K$;
    \item
      $g_D [\Theta] \colon I \to C^0 \Gamma_D$
      be the prescribed temperature on the Dirichlet boundary;
    \item
      $g_N [E T^{-1}] \colon I \to C^{D - 1} \Gamma_N$
      be the prescribed flow rate on the Neumann boundary.
  \end{itemize}
  define the following operators:
  \begin{subequations}
    \begin{alignat}{3}
      & A \colon C^{D - 1} K \times (I \to C^{D - 1} K) \to \R,
        \enspace
      && A(r, s^{D - 1})
        := \inner{r}{\kappa^{-1} s^{D - 1}}_{K, d - 1} \enspace
      && [E^{-1} T \Theta], \\
      %
      & B \colon C^d K \times (I \to C^{D - 1} K) \to \R, \enspace
      && B(v^d, r)
        := \inner{\delta_{D - 1} r}{v^d}_{K, d} \enspace
      && [L^{-D}], \\
      %
      & C \colon C^d K \times (I \to C^d K) \to \R, \enspace
      && C(v^d, w^d) := \inner{\pi w^d}{v^d}_{K, d} \enspace
      && [E L^{-2 d} \Theta^{-1}], \\
      %
      & G \colon C^{D - 1} K \to \R, \enspace
      && G(r) :=(\tr_{\Gamma_D, d - 1} r \smile g_D)[\Gamma_D]
        \enspace
      && [\Theta], \\
      %
      & F \colon C^d K \to \R, \enspace
      && F(v^d) := \inner{f}{v^d}_{K, d} \enspace
      && [E T^{-1} L^{-D}].
    \end{alignat}
  \end{subequations}
  Our unknowns are:
  \begin{itemize}
    \item
      $q [E T^{-1}] \colon I \to C^{D - 1} K$ (heat flow rate);
    \item
      $\tilde{u} [\Theta L^d] \colon I \to C^d K$ (dual temperature).
  \end{itemize}
  We are solving the following problem for $q$ and $\tilde{u}$:
  \begin{subequations}
    \begin{alignat}{4}
      & \forall r [E T^{-1}] \in \Ker \tr_{\Gamma_N, d - 1}, \enspace
      && A(r, q) - B^T(r, \tilde{u})
      && = - G(r) \enspace
      && [E T^{-1} \Theta], \\
      %
      & \forall v^d [\Theta L^d] \in C^d K, \enspace
      && - B(v^d, q) - C(v^d, \frac{\partial \tilde{u}}{\partial t})
      && = - F(v^d) \enspace
      && [E T^{-1} \Theta], \\
      %
      &
      && \tr_{\Gamma_N, d - 1} q
      && = g_N \enspace
      && [E T^{-1}], \\
      %
      &
      && \tilde{u}(t_0)
      && = \star_{K, 0} u_0 \enspace
      && [\Theta L^d].
    \end{alignat}
  \end{subequations}
  The temperature $u [\Theta] \colon I \to C^0 K$ is calculated in the
  post-processing phase by the formula
  \begin{equation}
    u(t, c_0) :=
    \begin{cases}
      u_0(c_0), & t = t_0 \\
      (\star_d \tilde{u})(t, x),
        & t > t_0\ \text{and}\ c_0 \notin (\Gamma_D)_0 \\
      g_D(t, c_0), & t_0 > 0\ \text{and}\ c_0 \in (\Gamma_D)_0
    \end{cases}.
  \end{equation}
\end{formulation}


\section{Discrete elasticity}
\label{section:discrete_elasticity}
\begin{discussion}[Discrete elasticity]
  Let $M$ be a mesh of dimension $3$, $K$ be the Forman subdivision of $M$.
  Let $L$ and $F$ denote length and force measures respectively.

  Discrete displacement is represented by
  \begin{equation}
    \eta^1 [L^2] \in C^1 K.
  \end{equation}
  Displacement gradient is represented by
  \begin{equation}
    \epsilon^0 [1] \in C^0 K,\ \omega^2 [L^2] \in C^2 K.
  \end{equation}
  Stress (force) is represented by
  \begin{equation}
    \tau^0 [F] \in C^0 K,\ \tau^2 [F L^2] \in C^2 K.
  \end{equation}
  Body force is represented by
  \begin{equation}
    \mathfrak{b}^1 [F] \in C^1 K.
  \end{equation}
  Let $\lambda, \mu [F] \in \R$ be the Lam{\'e} parameters.
  Our model is the following.
  \begin{subequations}
    \begin{align}
      & \epsilon^0 = \delta_1^\star \eta^1
      & (\text{volumetric displacement gradient}), \\
      & \omega^2 = \delta_1 \eta^1
      & (\text{deviatoric displacement gradient}), \\
      & \tau^0 = \lambda \epsilon^0
      & (\text{hydrostatic force}), \\
      & \tau^2 = \mu \omega^2
      & (\text{deviatoric force}), \\
      & \delta_0 \tau^0 + \delta_2^\star \tau^2 + \mathfrak{b}^1 = 0
      & (\text{conservation of linear momentum}).
    \end{align}
  \end{subequations}
\end{discussion}

\begin{example}
  Consider the problem of a twist of a cylindrical bar described
  in Section 9.1 of
  (\href
    {https://www.sciencedirect.com/science/article/pii/S0020768311001727}
    {Hadjesfandiari 2011}).
  Let
    $\theta$ be the constant angle of twist per unit length,
    $\lambda$ and $\mu$ be the Lame parameters.
  Let
    ${\bf u}$ be the displacement vector,
    $\boldsymbol{\epsilon}$ be the strain tensor,
    $\boldsymbol{\omega}$ be the rotation tensor,
    $\boldsymbol{\sigma}$ be the stress tensor.
  Then at any point $x = (x_1, x_2, x_3)$ we have:
  \begin{equation}
    {\bf u} =
    \begin{pmatrix}
      -\theta x_2 x_3 \\
      \theta x_1 x_3 \\
      0
    \end{pmatrix},\
    \boldsymbol{\epsilon} =
      \frac{\theta}{2}
      \begin{pmatrix}
        0 & 0 & - x_2  \\
        0 & 0 & x_1 \\
        -x_2 & x_1 & 0
      \end{pmatrix},\
    \boldsymbol{\omega} =
      \frac{\theta}{2}
      \begin{pmatrix}
        0 & -2 x_3 & - x_2  \\
        2 x_3 & 0 & x_1 \\
        x_2 & -x_1 & 0
      \end{pmatrix},\
    \boldsymbol{\sigma} =
      \mu \theta
      \begin{pmatrix}
        0 & 0 & - x_2  \\
        0 & 0 & x_1 \\
        -x_2 & x_1 & 0
      \end{pmatrix}.
  \end{equation}
  Note that the skew-symmetric matrix $\boldsymbol{\omega}$ corresponds to the
  vector
  \begin{equation}
    \label{idec/discrete_elasticity/example_9_1:equation:rotation_vector}
    \boldsymbol{\omega} \mapsto \frac{\theta}{2} (-x_1, -x_2, 2 x_3)^T.
  \end{equation}
  Let $h \in \R^+$ and consider a $3$D regular grid $K$ of size $h$.
  For integers $i, j, k$, nodes in $K$ have coordinates
  \begin{equation}
    {\bf x}_{(i, j, k)} := (i h, j h, k h).
  \end{equation}
  Nodes in $K$ will be denoted by $\mathcal{N}_{(i, j, k)}$.

  There are three type of edges in $K$ (parallel to the $3$ axes)
  constructed as follows.
  Let $p \in \{1, 2, 3\}$ and $e_p$ be the $p$-th unit vector.
  Denote by $\mathcal{E}^{(p)}_{(i, j, k)}$ the edge starting at
  $\mathcal{N}_{(i, j, k)}$ and ending at $\mathcal{N}_{(i, j, k) + e_p}$.
  In particular, the oriented boundary of $\mathcal{E}^{(1)}_{(i, j, k)}$ is
  \begin{equation}
    \partial_1 \mathcal{E}^{(1)}_{(i, j, k)} =
    - \mathcal{N}_{(i, j, k)} + \mathcal{N}_{(i + 1, j, k)}.
  \end{equation}
  Similar computation holds for $p = 2$ and $p = 3$.

  For $p, q \in \{1, 2, 3\},\ p < q$ faces in $K$ are denoted by
  $\mathcal{F}^{(p, q)}_{(i, j, k)}$ and represent squares starting at
  $\mathcal{N}_{(i, j, k)}$ with basis vectors going in directions
  $e_p$ and $e_q$.
  Also use the identification of cochains
  \begin{equation}
    \mathcal{F}^{(q, p)}_{(i, j, k)} := -\mathcal{F}^{(p, q)}_{(i, j, k)}.
  \end{equation}
  For instance, the oriented boundary of $\mathcal{F}^{(1, 2}_{(i, j, k)}$ is
  \begin{equation}
    \partial_2 \mathcal{F}^{(1, 2)}_{(i, j, k)} =
    - \mathcal{E}^{(1)}_{(i, j + 1, k)}
    + \mathcal{E}^{(1)}_{(i, j, k)}
    + \mathcal{E}^{(2)}_{(i + 1, j, k)}
    - \mathcal{E}^{(2)}_{(i, j, k)}.
  \end{equation}
  We will work with the approximation $\eta^1 := \underline{u}$.

  Let $p \in \{1, 2, 3\}$.
  Then for $\epsilon^0 := \delta_1^\star \eta^1$, using the fact that
  \begin{equation}
    \eta^1 \mathcal{E}^{(p)}_{(i, j, k) + e_p} =
    \eta^1 \mathcal{E}^{(p)}_{(i, j, k)},
  \end{equation}
  we calculate:
  \begin{equation}
    \epsilon^0 \mathcal{N}_{(i, j, k)}
    =
      \frac{1}{h^2}
      \sum_{p = 1}^3 (
          \eta^1 \mathcal{E}^{(p)}_{(i, j, k)}
        - \eta^1 \mathcal{E}^{(p)}_{(i, j, k) + e_p}
      )
    = 0.
  \end{equation}
  Hence,
  \begin{equation}
    \tau^0 = \lambda \epsilon^0 = 0.
  \end{equation}
  Using the computation from
  \Cref{idec/vector_field_to_1_cochain/1d_example:exact_value}
  in each direction, we get
  \begin{subequations}
    \begin{alignat}{2}
      & \eta^1 \mathcal{E}^{(1)}_{(i, j, k)} &&
        = \frac{h}{2} \theta (-x_2 x_3 - x_2 x_3)
        = - \theta h x_2 x_3
        = - \theta j k h^3, \\
      & \eta^1 \mathcal{E}^{(2)}_{(i, j, k)} &&
        = \frac{h}{2} \theta (x_1 x_3 + x_1 x_3)
        = \theta h x_1 x_3
        = \theta i k h^3, \\
      & \eta^1 \mathcal{E}^{(3)}_{(i, j, k)} &&
        = \frac{h}{2} \theta (0 + 0)
        = 0.
    \end{alignat}
  \end{subequations}
  For $\omega^2 := \delta_1 \eta^1$, using that
  $\epsilon(c_2) = \eta^1(\partial c_2)$, we get:
  \begin{subequations}
    \begin{alignat}{3}
      & \omega^2 \mathcal{F}^{(2, 3)}_{(i, j, k)}
      && =
        ( - \eta^1 \mathcal{E}^{(2)}_{(i, j, k + 1)}
          + \eta^1 \mathcal{E}^{(2)}_{(i, j, k)}
        )
      + ( \eta^1 \mathcal{E}^{(3)}_{(i, j + 1, k)}
          - \eta^1 \mathcal{E}^{(3)}_{(i, j, k)}
        )
      = - \theta i h^3 + 0
      && = - \theta i h^3, \\
%
      & \omega^2 \mathcal{F}^{(3, 1)}_{(i, j, k)}
      && =
        ( - \eta^1 \mathcal{E}^{(3)}_{(i + 1, j, k)}
          + \eta^1 \mathcal{E}^{(3)}_{(i, j, k)}
        )
      + ( \eta^1 \mathcal{E}^{(1)}_{(i, j, k + 1)}
          + \eta^1 \mathcal{E}^{(1)}_{(i, j, k)}
        )
      = 0 - \theta j h^3
      && = - \theta j h^3, \\
%
      & \omega^2 \mathcal{F}^{(1, 2)}_{(i, j, k)}
      && =
        ( - \eta^1 \mathcal{E}^{(1)}_{(i, j + 1, k)}
          + \eta^1 \mathcal{E}^{(1)}_{(i, j, k)}
        )
      + ( \eta^1 \mathcal{E}^{(2)}_{(i + 1, j, k)}
          - \eta^1 \mathcal{E}^{(2)}_{(i, j, k)}
        )
      = \theta k h^3 + \theta k h^3
      && = 2 \theta k h^3.
    \end{alignat}
  \end{subequations}
  We see the clear correspondence (by a factor of $2 h^2$) to the ``flattened''
  version of $\boldsymbol{\omega}$,
  \Cref{idec/discrete_elasticity/example_9_1:equation:rotation_vector}.

  We have $\tau^2 = \mu \omega^2$ and hence
  $\delta_2^\star \tau^2 = \mu \delta_2^\star \omega^2$.
  Then
  \begin{subequations}
    \begin{alignat}{3}
      & (\delta_2^\star \tau^2) \mathcal{E}^{(1)}_{(i, j, k)}
      && =
        \frac{\mu}{h^2}
        (
          ( \omega^2 \mathcal{F}^{(1, 2)}_{(i, j, k)}
            - \omega^2 \mathcal{F}^{(1, 2)}_{(i, j - 1, k)}
          )
          -
          ( \omega^2 \mathcal{F}^{(1, 3)}_{(i, j, k)}
            - \omega^2 \mathcal{F}^{(1, 3)}_{(i, j, k - 1)}
          )
        )
      = \frac{\mu}{h^2} (0 - 0)
      && = 0, \\
%
      & (\delta_2^\star \tau^2) \mathcal{E}^{(2)}_{(i, j, k)}
      && =
        \frac{\mu}{h^2}
        (
          ( \omega^2 \mathcal{F}^{(1, 2)}_{(i - 1, j, k)}
            - \omega^2 \mathcal{F}^{(1, 2)}_{(i, j, k)}
          )
          -
          ( \omega^2 \mathcal{F}^{(2, 3)}_{(i, j, k)}
            - \omega^2 \mathcal{F}^{(2, 3)}_{(i, j, k - 1)}
          )
        )
      = \frac{\mu}{h^2} (0 - 0)
      &&  = 0, \\
%
      & (\delta_2^\star \tau^2) \mathcal{E}^{(3)}_{(i, j, k)}
      && =
        \frac{\mu}{h^2}
        (
          ( \omega^2 \mathcal{F}^{(1, 3)}_{(i - 1, j, k)}
            - \omega^2 \mathcal{F}^{(1, 3)}_{(i, j, k)}
          )
          -
          ( \omega^2 \mathcal{F}^{(2, 3)}_{(i, j - 1, k)}
            - \omega^2 \mathcal{F}^{(2, 3)}_{(i, j, k)}
          )
        )
      = \frac{\mu}{h^2} (0 - 0)
      && = 0.
    \end{alignat}
  \end{subequations}
  Hence, $\delta_2^\star \tau^2 = 0$.
  With zero body force $\mathfrak{b}^1$, we get:
  \begin{equation}
    \delta_0 \tau^0 + \delta_2^\star \tau^2 + \mathfrak{b}^1 = 0 + 0 + 0 = 0.
  \end{equation}
\end{example}


\section{Discrete vector bundles and covariant exterior derivative}
\label{section:discrete_vector_bundles_and_covariant_exterior_derivative}
\begin{definition}
  \label{idec/mesh/quasi_cubical/hodge_star/concept-definition}
  Let
    $d \in \N$,
    $K$ be a compatibly oriented quasi-cubical
    \hyperref[idec:mesh:definition]{mesh} of dimension $d$,
    $[K] := \sum_{c_d \in K_d} c^d$ be the fundamental class of $K$
    $\inner{\cdot}{\cdot}$ be an inner product on $K$,
    $p \in \{0, ..., d\}$.
  The \textbf{Hodge star operator on $p$-cochains}
  $\star_p \colon C^p K \to C^{d - p} K$
  is defined as the unique map satisfying the following equation:
  for any $\pi^p \in C^p K$ and $\rho^{d - p} \in C^{d - p} K$,
  \begin{equation}
    \inner{\rho^{d - p}}{\star_p \pi^p}_{d - p}
    = (\rho^{d - p} \smile \pi^p)[K].
  \end{equation}
  The operator $\star_p$ has physical dimension $[L^{d - 2 p}]$.
\end{definition}

\begin{definition}
  Let
    $K$ be a quasi-cubical mesh,
    $V$ be a vector space,
    $p \in \N$,
    $q \in \N$.
  Define the \textbf{cup product of a vector-valued cochain with a cochain}
  \begin{equation}
    \usmile \colon C^p(K, V) \times C^q K \to C^{p + q}(K, V)
  \end{equation}
  as follows: for any $v \in V$, $\tau^p \in C^p K$, $\sigma^q \in C^q K$,
  \begin{equation}
    (v \otimes \tau^p) \usmile \sigma^q := v \otimes (\tau^p \smile \sigma^q),
  \end{equation}
  and extend it by linearity on $C^p(K, V) \times C^q K$.
\end{definition}

\begin{remark}
  Let
    $K$ be a quasi-cubical mesh,
    $V$ be a vector space,
    $v \in V$,
    $p \in \N$,
    $\sigma^p \in C^p K$.
  Denote by $1$ the identity zero-cochain on $K$.
  Then
  \begin{equation}
  \label{cmc:tensor_to_cup:equation}
    v \otimes \sigma^p
    = v \otimes (1 \smile \sigma^p)
    = (v \otimes 1) \usmile \sigma^p.
  \end{equation}
  Abuse the notation and identify $v \in V$ with $v \otimes 1 \in C^0(K, V)$.
  Then \Cref{cmc:tensor_to_cup:equation} reads as
  \begin{equation}
    v \otimes \sigma^p = v \usmile \sigma^p.
  \end{equation}
\end{remark}

\begin{definition}
  \label{idec/mesh/quasi_cubical/hodge_star/concept-definition}
  Let
    $d \in \N$,
    $K$ be a compatibly oriented quasi-cubical
    \hyperref[idec:mesh:definition]{mesh} of dimension $d$,
    $[K] := \sum_{c_d \in K_d} c^d$ be the fundamental class of $K$
    $\inner{\cdot}{\cdot}$ be an inner product on $K$,
    $p \in \{0, ..., d\}$.
  The \textbf{Hodge star operator on $p$-cochains}
  $\star_p \colon C^p K \to C^{d - p} K$
  is defined as the unique map satisfying the following equation:
  for any $\pi^p \in C^p K$ and $\rho^{d - p} \in C^{d - p} K$,
  \begin{equation}
    \inner{\rho^{d - p}}{\star_p \pi^p}_{d - p}
    = (\rho^{d - p} \smile \pi^p)[K].
  \end{equation}
  The operator $\star_p$ has physical dimension $[L^{d - 2 p}]$.
\end{definition}

\begin{proposition}
  Let
    $K$ be a quasi-cubical mesh,
    $V$ be a vector space,
    $v \in V$,
    $p \in \N$,
    $\sigma^p \in C^p K$.
  Then
  \begin{equation}
    \nabla_p(v \usmile \sigma^p)
    = \nabla_0 v \usmile \sigma^p + v \usmile \delta_p \sigma^p.
  \end{equation}
\end{proposition}

\begin{proof}
  \begin{equation}
    \nabla_p(v \usmile \sigma^p)
    := \nabla_p(v \otimes \sigma^p)
    = \nabla_p((v \otimes 1) \usmile \sigma^p)
    = \nabla_0(v \otimes 1) \usmile \sigma^p
      + (v \otimes 1) \usmile \delta_p \sigma^p
    =: \nabla_0 v \usmile \sigma^p + v \usmile \delta_p \sigma^p.
  \end{equation}
\end{proof}

\begin{proposition}
  Let
    $K$ be a quasi-cubical mesh,
    $V$ be a vector space,
    $\nabla \colon C^\bullet(K, V) \to C^\bullet(K, V)$ be
      a discrete exterior covariant derivative.
  Then $\nabla$ satisfies the graded Leibniz rule:
  for any $p, q \in \N$, $\sigma_V^p \in C^p(K, V)$, $\tau^q \in C^q K$,
  \begin{equation}
    \nabla_{p + q}(\sigma_V^p \usmile \tau^q)
    = \nabla_p \sigma_V^p \usmile \tau^q
      + (-1)^p \sigma_V^p \usmile \delta_q \tau^q.
  \end{equation}
\end{proposition}

\begin{proof}
  It is enough to prove the proposition for a product element $\sigma_V^p$.
  Let $v \in V$, $\theta^p \in C^p K$ and $\sigma_V^p = v \usmile \theta^p$.
  Then
  \begin{equation}
    \begin{split}
      \nabla_{p + q}(\sigma_V^p \usmile \tau^q)
      & = \nabla_{p + q}(v \usmile (\theta^p \smile \tau^q)) \\
      & = \nabla_0 v \usmile (\theta^p \smile \tau^q)
          + v \usmile (\delta_p \theta^p \smile \tau^q)
          + (-1)^p v \usmile (\theta^p \smile \delta_q \tau^q) \\
      & = \nabla_p(v \usmile \theta^p) \usmile \tau^q
          + (-1)^p (v \usmile \theta^p) \usmile \delta_q \tau^q \\
      & = \nabla_p \sigma_V^p \usmile \tau^q
          + (-1)^p \sigma_V^p \usmile \delta_q \tau^q. \qedhere
    \end{split}
  \end{equation}
\end{proof}

%% \begin{example}
%   Let
%     $d \in \N$,
%     $K$ be a cubical grid of dimension $d$ with the standard grid orientation,
%     $V = C_1 K$.
%   Define $\nabla \colon C^\bullet(K, C_1 K) \to C^\bullet(K, C_1 K)$ as follows.
%   Let $p \in \{0, ..., d - 1\}$ and $e_p$ be the sum of all basis $1$-chains
%   whose cells are parallel to the $p$-th basis vector in the standard basis of
%   the embedding of $K$.
%   Then we require that for each $\sigma^\bullet \in C^\bullet K$,
%   \begin{equation}
%     \nabla_p(e_p \usmile \sigma^\bullet) = e_p \usmile \delta \sigma^\bullet.
%   \end{equation}
%   Denote by
%   \begin{equation}
%     \Omega^\bullet(\abs{K}, \R^d)
%   \end{equation}
%   the space of $\R^d$-valued smooth differential forms on $\abs{K}$.
%   Let
%   \begin{equation}
%     S \colon \chi \abs{K} \to C^1 K
%   \end{equation}
%   be the discretization of vector fields into $1$-forms.
%   We define the discretization of vector-valued smooth $0$-forms,
%   \begin{equation}
%     {\bf S} \colon \Omega^0(\abs{K}, \R^d) \to C^0(K, C_1 K),
%   \end{equation}
%   as follows: if for $p = 0, ..., d - 1$,
%   $u_p \colon \abs{K} \to \R$ is smooth with discretization
%   $\eta_{(p)}^0 \in C^1 K = \widetilde{u_p}$, then
%   \begin{equation}
%     {\bf S}\left(\sum_{p = 0}^{D - 1} u_p \frac{\partial}{\partial x_p}\right)
%     := \sum_{p = 0}^{D - 1} e_p \smile \eta_{(p)}^0.
%   \end{equation}
%   Then if we take the vector gradient ${\boldsymbol \nabla}$
%   (which comes from the Levi-Civitta connection on the flat metric on
%   $\abs{K}$), then for any smooth function $f \colon \abs{K} \to \R$,
%   \begin{equation}
%     {\bf S}({\boldsymbol \nabla} f)
%     = \sum_{p = 0}^{D - 1}
%       e_p \usmile S\left(\frac{\partial}{\partial x_p} f\right).
%   \end{equation}
% \end{example}

% We will summarize the operations relating cochains and embedding.
% \begin{discussion}
%   Let
%     $d \in \N$,
%     $K$ be a quasi-cubical flat mesh of dimension $d$,
%     $X$ be the manifold it encompasses.
%   We define the following data:
%   \begin{itemize}
    % \item
    %   $R \colon \Omega^\bullet X \to C^\bullet K$ denotes the de Rham map,
    %   \begin{equation}
    %     (R_p \omega^p) c_p := \int_{\abs{c_p}} \omega^p;
    %   \end{equation}
    % \item
    %   $S \colon \Omega^\bullet(X, \chi X) \to C^\bullet(K, C^1 K)$
    %   denotes the discretization of vector-valued smooth forms into
    %   $1$-cochain valued cochains,
    %   \begin{equation}
    %     S_p(u \otimes \omega^p) := \underline{u} \otimes R_p \omega^p;
    %   \end{equation}
    % \item
    %   $\nabla \colon \chi X \to \Omega^1(X, \chi X)$ is the vector gradient
    %   (connection on vector fields);
    % \item
    %   $D \colon \Omega^\bullet(X, \chi X) \to \Omega^\bullet(X, \chi X)$
    %   is the respective covariant exterior derivative on vector-valued forms,
    %   \begin{equation}
    %     D_p(u \otimes \omega^p) :=
    %     \nabla u \uwedge \omega^p + u \otimes d_p \omega^p;
    %   \end{equation}
    % \item
    %   $\nabla^{(K)} \colon C^1 K \to C^1(K, C^1 K)$
    %   is the discrete vector gradient (discrete connection on $1$-cochains);
    % \item
    %   $D^{(K)} \colon C^\bullet(K, C^1 K) \to C^\bullet(K, C^1 K)$
    %   is the respective discrete covariant exterior derivative on
    %   $1$-cochain valued cohains,
    %   \begin{equation}
    %     D^{(K)}_p(\pi^1 \otimes \rho^p) :=
    %     \nabla^{(K)} \pi^1 \usmile \rho^p + \pi^1 \otimes \delta_p \rho^p.
    %   \end{equation}
%   \end{itemize}
% \end{discussion}


\end{document}
